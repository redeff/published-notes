\documentclass{article}
\usepackage{amssymb}
\usepackage{mathtools}

\everymath{\displaystyle}
\setlength{\parskip}{3mm}
\setlength{\parindent}{0mm}

\def\R{\mathbb{R}}
\def\K{\mathbb{K}}
\def\V{\mathbb{V}}
\def\W{\mathbb{W}}
\def\WW{\mathcal{W}}
\def\PP{\mathcal{P}}
\def\FF{\mathcal{F}}
\def\U{\mathbb{U}}
\def\C{\mathbb{C}}
\def\N{\mathbb{N}}
\def\Q{\mathbb{Q}}
\def\Z{\mathbb{Z}}

\def\BB{\mathcal{B}}
\def\AA{\mathcal{A}}
\def\EE{\mathcal{E}}
\def\CC{\mathcal{C}}
\def\OO{\mathcal{O}}
\def\DD{\mathcal{D}}

\def\e{\varepsilon}
\def\inn{\subseteq}

\def\S{\mathbb{S}}
\def\T{\mathbb{T}}
\def\l{\lambda}

\def\To{\Rightarrow}
\def\from{\leftarrow}
\def\From{\Leftarrow}

\DeclareMathOperator{\cl}{cl}
\DeclareMathOperator{\li}{li}
\DeclareMathOperator{\tl}{tl}
\DeclareMathOperator{\Id}{Id}
\DeclareMathOperator{\tr}{tr}
\DeclareMathOperator{\oo}{o}
\DeclareMathOperator{\spec}{spec}
\DeclareMathOperator{\mult}{mult}

\DeclareMathOperator{\iso}{iso}
\DeclareMathOperator{\mono}{mono}
\DeclareMathOperator{\epi}{epi}
\DeclareMathOperator{\adj}{adj}

\DeclareMathOperator{\Nu}{Nu}
\DeclareMathOperator{\Ima}{Im}
\DeclareMathOperator{\id}{id}
\DeclareMathOperator{\ze}{ze}

\DeclareMathOperator{\rg}{rg}

\DeclareMathOperator{\Hom}{Hom}
\DeclareMathOperator{\GL}{GL}
\DeclareMathOperator{\cont}{cont}
\DeclareMathOperator{\Hs}{H}

\DeclareMathOperator{\D}{D}
\DeclareMathOperator{\lcm}{lcm}

\DeclareMathOperator{\ev}{ev}
\DeclareMathOperator{\sg}{sg}

\date{}
\author{}

\begin{document}
\section*{Autovalores de Autoadjuntas}
\begin{teo}
    Si $f : \V \to \V$, $\V$ finito autoadjunta, luego $f$ tienen algún autovalor.
\end{teo}
\begin{demo}
    Si $\K = \C$ es trivial.
    Si $\K = \R$, tomemos $\BB$ base ortonormal.

    Sea $A = [f]_\BB$. Como $f$ es autoadjunta, $A$ es simétrica.

    Vamos a construirnos una nueva transformación $C^n \to \C^n$ con la misma matriz. Tomemos $g : \C^n \to \C^n$ tal que:
    \[
        [g]_E = A
    \]
    Que es autoadjunta ya que $A$ es hermitiana, por lo que $g$ tiene un autovalor $\l$, que por ser autoadjunta debe ser real.
    Luego $\l \in \R$ es raíz de $\chi_g$, pero:
    \[\chi_g = \chi_A = \chi_f\]
    Luego $\l$ es autovalor de $g$.
\end{demo}
\begin{teo}
    Si $\dim \V = n$ y $f : \V \to \V$ autoadjunta, luego $f$ es diagonalizable, y con una base ortonormal de autovectores.
\end{teo}
\begin{demo}
    Vamos a hacer inducción. Si $n = 0$ ya está.

    Si no, tomemos algún autovalor $\l \in \R$ de $f$, y su autovector $v$. WLOG $v$ es unitario.

    Si tomamos $\W = \langle v \rangle ^\perp$, nos gustaría que $\W$ fuese $f$-invariante para hacer inducción. Si $w \in \W$ tomemos:
    % \[w \in \W \To\]
    \[0 = \langle w, v\rangle \To\]
    % \[0 = \l\langle w, v\rangle \To\]
    \[0 = \langle w, \l v\rangle \To\]
    \[0 = \langle w, fv\rangle \To\]
    \[0 = \langle fw, v\rangle \To\]
    \[fw \in \W\]
    Luego $\W$ es $f$-invariante. Tomemos entonces $\tilde f : \W \to \W$ la restricción, que tendrá de dominio un espacio de dimensión $\dim \W = n-1$, y será claramente autoadjunta.

    Por hipótesis inductiva, tomemos una base de autovectores de $\tilde f$. Sabemos que los de distinto autovalor son ortogonales, y dentro de cada autovalor los ortonormalizamos, luego tenemos una base ortonormal de autovalores de $\tilde f$. La extendemos con el vector $v$, que es ortogonal a todos los otros y es una autovector, luego tenemos una base ortonormal de autovectores de $f$.
\end{demo}
\begin{teo}
    Si $f : \V \to \V$, $f$ es autoadjunta sii existe una bon $\BB$ tal que $[f]_\BB$ es diagonal y real.
\end{teo}
\begin{demo}
    Un lado está por el teorema anterior. Y la vuelta es fácil, ya que si hay una bon que tiene como matriz una diagonal real $A$, sabemos que $A$ es hemitiana, luego $f$ es autoadjunta.
\end{demo}
\section*{Transformaciones Normales}
\begin{defi}
    Una transfomración lineal $f : \V \to \V$ es \emph{normal} sii tiene adjunta y conmuta con ella: $f \circ f^* = f^* \circ f$.

    Notemos que las autoadjuntas son normales.
\end{defi}
\begin{ej}
    Si Tengo $f : \R^2 \to \R^2$, luego:
    \[
        [f] =
        \begin{bmatrix}
            \cos \alpha & \sin \alpha \\
            -\sin \alpha & \cos \alpha \\
        \end{bmatrix}
    \]
    No es autoadjunta, al menos que $\sin \alpha = -\sin \alpha$. Sin embargo, $f \circ f^* = f^* \circ f = \id$, por lo que $f$ es normal.
\end{ej}
\section*{Propiedades de las Normales}
\begin{teo}
    Sea $f : \V \to \V$ normal, con $v \in \V$ y $\l \in \K$. Son equivalentes:
    \begin{enumerate}
        \item $v$ es $\l$-autovector de $f$.
        \item $v$ es $\overline \l$-autovector de $f^*$.
    \end{enumerate}
\end{teo}
\begin{demo}
    Tomemos $g = f - \l \id$, con $g^* = f^* - \overline \l \id$, que claramente será normal.

    Tenemos que:
    \[v \in \ker g\]
    \[\iff \langle gv, gv\rangle = 0\]
    \[\iff \langle v, g^*gv\rangle = 0\]
    \[\iff \langle v, gg^*v\rangle = 0\]
    \[\iff \langle g^*v, g^*v\rangle = 0\]
    \[v \in \ker g^*\]
    Luego $v \in \ker g \iff v \in \ker g^*$, luego estamos.
\end{demo}
\begin{teo}
    Dada $f : \V \to \V$ con $\V$ un $\C$-evpi (no anda con $\R$) y $f$ normal, se tiene una bon $\BB$ de autovectores de $f$.
\end{teo}
\begin{demo}
    (Es la misma demo que para autoadjuntas)

    Vamos a hacer inducción. Si $n = 0$ ya está.

    Si no, tomemos algún autovalor $\l$ de $f$, que sabemos que existe ya que estamos en $\C$, y su autovector $v$. WLOG $v$ es unitario.

    Si tomamos $\W = \langle v \rangle ^\perp$, nos gustaría que $\W$ fuese $f$-invariante para hacer inducción. Si $w \in \W$ tomemos:
    % \[w \in \W \To\]
    \[0 = \langle w, v\rangle \To\]
    % \[0 = \l\langle w, v\rangle \To\]
    \[0 = \langle w, \overline \l v\rangle \To\]
    \[0 = \langle w, f^*v\rangle \To\]
    Ya que $v$ es $\overline \l$-autovector de $f^*$.
    \[0 = \langle fw, v\rangle \To\]
    \[fw \in \W\]
    Luego $\W$ es $f$-invariante, y análogamente $\W$ es $f^*$-invariante. Tomemos entonces $\tilde f : \W \to \W$ la restricción, que tendrá de dominio un espacio de dimensión $\dim \W = n-1$, y será claramente normal con adjunta $\tilde {f^*} : \W \to \W$.

    Por hipótesis inductiva, tomemos una base de autovectores de $\tilde f$. Sabemos que los de distinto autovalor son ortogonales, y dentro de cada autovalor los ortonormalizamos, luego tenemos una base ortonormal de autovalores de $\tilde f$. La extendemos con el vector $v$, que es ortogonal a todos los otros y es una autovector, luego tenemos una base ortonormal de autovectores de $f$.
\end{demo}
\begin{teo}
    Una dada una $f : \V \to \V$, con $\V$ un $\C$-evpi, luego es lo mismo:
    \begin{enumerate}
        \item $f$ normal
        \item Existe base bon de autovectores.
    \end{enumerate}
\end{teo}
\begin{demo}
    $1) \To 2)$ ya está.

    $2) \To 1)$ sale ya que tomamos $[f]_\BB$ con $\BB$ bon de autovectores, que será diagonal, luego:
    \[[f^*]_\BB =
        \overline {[f]_\BB^t} =
        \overline {[f]_\BB}
    \]
    Luego la matriz de la adjunta también es diagonal, por lo que conmutan.
\end{demo}
\section*{Teorema Espectral para Normales}
\begin{teo}
    Dado $\V$ un $\C$-evpi, $\dim \V = n < \infty$, con $f : \V \to \V$ normal.

    Luego si $\l_i$ son los autovectores (sin repetidos), definimos:
    \begin{itemize}
        \item $\V_i = \ker f - \l_i \id$
        \item $p_i$ proyección ortogonal hacia $\V_i$
    \end{itemize}
    Luego para todo $i \neq j$, tenemos que:
    \begin{enumerate}
        \item $
        \V_i \perp \V_j \quad \text{y} \quad p_i \circ p_j = 0
    $
\item $
        \V = \bigoplus \V_i
    $
\item $
        \id = \sum p_i
    $
\item $
        f = \sum \l_i p_i
    $
    \end{enumerate}
\end{teo}
\begin{demo}
    Tomemos $i \neq j$, luego sean $v_i$ y $v_j$ autovectores correspondientes tenemos:
    \[\langle v_i, f^* v_j \rangle = \langle fv_i, v_j \rangle\]
    \[\langle v_i, \overline \l_j v_j \rangle = \langle \l_i v_i, v_j \rangle\]
    \[\l_j \langle v_i, v_j \rangle = \l_i \langle v_i, v_j \rangle\]
    Luego como $\l_i \neq \l_j$, tenemos que $\langle v_i , v_j\rangle = 0$, luego estamos para $1)$ (la otra parte es inmediata).

    Parte $2)$ es inmediata por la existencia de una bon de autovectores.

    Para la parte $3)$, notemos que $p_iv_j = 0$ si $i \neq j$, y $p_ix_j = x_j$ si $i = j$, luego como existe una bon de autovalores, todo $v \in \V$ se puede expresar como:
    \[
        v = \sum v_i
    \]
    Por $2)$, con $v_i$ autovector de autovalor $\l_i$, luego:
    \[
        \left(\sum p_i\right) \left(\sum v_i\right) =
    \]
    \[
        \left(\sum p_i \left(\sum v_j\right) \right)=
    \]
    \[
        \left(\sum_{i,j} p_i v_j \right)=
    \]
Como se anula para ditintos,
    \[
        \left(\sum p_i v_i \right) =
    \]
    \[
        \left(\sum v_i \right) = v
    \]
    Luego $\sum _pi = \id$. Similarmente:
    \[
        \left(\sum \l_i p_i\right) \left(\sum v_i\right) =
    \]
    \[
        \left(\sum \l_i p_i \left(\sum v_j\right) \right)=
    \]
    \[
        \left(\sum_{i,j} \l_i p_i v_j \right)=
    \]
Como se anula para ditintos,
    \[
        \left(\sum \l_i p_i v_i \right) =
    \]
    \[
        \left(\sum \l_i v_i \right) =
    \]
Como son autovectores,
    \[
        \left(\sum f v_i \right) = f(v)
    \]
Luego $\sum \l_i p_i = f$
\end{demo}
\section*{Caracterización de las Normales}
\begin{teo}
    Dado $\V$ un $\C$-evpi de $\dim$ finita, con $f: \V \to \V$, luego son equivalentes:
    \begin{enumerate}
        \item $f$ normal
        \item Hay un polinomio $h(x) \in \C[x]$ tal que $h(f) = f^*$.
    \end{enumerate}
\end{teo}
\end{document}
