\documentclass{article}
\usepackage{amssymb}
\usepackage{mathtools}

\everymath{\displaystyle}
\setlength{\parskip}{3mm}
\setlength{\parindent}{0mm}

\def\R{\mathbb{R}}
\def\K{\mathbb{K}}
\def\V{\mathbb{V}}
\def\W{\mathbb{W}}
\def\WW{\mathcal{W}}
\def\PP{\mathcal{P}}
\def\FF{\mathcal{F}}
\def\U{\mathbb{U}}
\def\C{\mathbb{C}}
\def\N{\mathbb{N}}
\def\Q{\mathbb{Q}}
\def\Z{\mathbb{Z}}

\def\BB{\mathcal{B}}
\def\AA{\mathcal{A}}
\def\EE{\mathcal{E}}
\def\CC{\mathcal{C}}
\def\OO{\mathcal{O}}
\def\DD{\mathcal{D}}

\def\e{\varepsilon}
\def\inn{\subseteq}

\def\S{\mathbb{S}}
\def\T{\mathbb{T}}
\def\l{\lambda}

\def\To{\Rightarrow}
\def\from{\leftarrow}
\def\From{\Leftarrow}

\DeclareMathOperator{\cl}{cl}
\DeclareMathOperator{\li}{li}
\DeclareMathOperator{\tl}{tl}
\DeclareMathOperator{\Id}{Id}
\DeclareMathOperator{\tr}{tr}
\DeclareMathOperator{\oo}{o}
\DeclareMathOperator{\spec}{spec}
\DeclareMathOperator{\mult}{mult}

\DeclareMathOperator{\iso}{iso}
\DeclareMathOperator{\mono}{mono}
\DeclareMathOperator{\epi}{epi}
\DeclareMathOperator{\adj}{adj}

\DeclareMathOperator{\Nu}{Nu}
\DeclareMathOperator{\Ima}{Im}
\DeclareMathOperator{\id}{id}
\DeclareMathOperator{\ze}{ze}

\DeclareMathOperator{\rg}{rg}

\DeclareMathOperator{\Hom}{Hom}
\DeclareMathOperator{\GL}{GL}
\DeclareMathOperator{\cont}{cont}
\DeclareMathOperator{\Hs}{H}

\DeclareMathOperator{\D}{D}
\DeclareMathOperator{\lcm}{lcm}

\DeclareMathOperator{\ev}{ev}
\DeclareMathOperator{\sg}{sg}

\date{}
\author{}

\title{Determinantes}
\begin{document}
	\maketitle
\section*{Definición}
	$\det$ es la única $\det : \K^{n \times n} \to \K$ que es multilineal y alternada y con $\det \Id = 1$


\section*{Permutaciones}
Decimos que , dado $n \in \N$, $[[n]] = \{x \in \N : x < n\}$.

Decimos que $\sigma : [[n]] \to [[n]]$ biyectiva se dice \emph{permutación}, y decimos $\sigma \in S_n$.

Notemos que:
\begin{itemize}
	\item $\id \in S_n$
	\item $|S_n| = n!$
	\item Si $\sigma, \tau \in S_n$, luego $\sigma \circ \tau \in S_n$
	\item Si $\sigma \in S_n$, luego $\sigma^{-1} \in S_n$
\end{itemize}

\subsection*{Transposición}
una transposición $(rs) \in S_n$ es una permutación que deja todo igual, excepto $r$ y $s$, que los swappea.

\subsection*{Matriz de Permutación}
Dada $\sigma \in S_n$ Decimos $A(\sigma) \in \K^{n \times n}$ es:
\[A(\sigma)_{ij} = \delta_{\sigma_ij}\]
Notemos que producto de matrices corresponde con composición de permutaciones, y que además el inverso coincide con la transpuesta para matrices de permutación.

\subsection*{Signo de una Permutación}
Dada $\sigma \in S_n$, decimos que $\sg \sigma = \det A_\sigma$.

\section*{Fórmula de Leibniz}
Tenemos
\[\det A = \sum_{\sigma \in S_n} \left(\sg \sigma \cdot \prod_i a_{\sigma_ii}\right)\]

Demo:
La $i$-ésima columna de la matriz $A$ está dada por:
\[A_i = \sum_j a_{ji} e_j\]
Luego por multilinearidad:
\[
	\det A = \sum_{i \in [[n]]^n} \det (\dots | e_{i_j} | \dots)_j \prod a_{i_jj}
\]
Pero cuando agarro $i_{a} = i_{b}$, se me anula, luego me restrinjo a $S_n \inn [[n]]^n$, y queda
\[
	\det A = \sum_{\sigma \in S_n} \det (\dots | e_{\sigma_j} | \dots)_j \prod a_{\sigma_jj} =
\]
\[
	\det A = \sum_{\sigma \in S_n} \sg \sigma \cdot \prod a_{\sigma_jj} =
\]

\section*{Determinantes en $\Z$}
Si tenemos $A \in \Q^{n \times n}$, con $a_{ij} \in \Z$, tenemos $\det A \in \Z$.

\section*{Adjunta}
La adjunta está dada por:
\[
	(\adj A)_{ij} = (-1)^{i+j} \det A(j|i)
\]
Que cumple $A \cdot \adj A = \Id \cdot \det A$
\section*{Regla de Cramer}
Si tenemos una ecuación de la forma $Ax = b$ con $A$ inversible, luego:
\[
	x_i = \frac{\det (a_1 | \dots | a_{i-1} | b | a_{i+1} | \dots )}{\det A}
\]
Esto sale tomando $x = \frac{\adj A}{\det A} \cdot b$ y expandiendo el producto.

\section*{Rango de matrices no cuadradas}
Si tenemos $A \in \K^{n \times m}$, decimos que una submatriz $B \in \K^{r \times s}$ de $A$ es el resultado de eliminar algunas filas y columnas de $A$.
Son equivalentes:
\begin{itemize}
	\item $\rg A \geq r$
	\item $\exists B \in \K^{r \times r}$ una submatriz de $A$ con $\det B \neq 0$
\end{itemize}
Si tengo $\rg A \geq r$, me tomo $r$ filas li, y luego en la submatriz inducida por esas filas (que se que tiene rango $r$), me tomo $r$ columnas li, y ya estoy.

La vuelta es trivial.

\section*{Determinante de una TL}
Dado un $\K$-ev $\V$ y un endo $f : \V \to_{\tl} \V$, con $\BB$ una base, decimos que \[\det f = \det [f]_\BB\]

Demostrar que anda bien para la composición, a identidad, y que cambios de base no cambian el determinante.
\end{document}
