\documentclass{article}
\usepackage{amssymb}
\usepackage{mathtools}

\everymath{\displaystyle}
\setlength{\parskip}{3mm}
\setlength{\parindent}{0mm}

\def\R{\mathbb{R}}
\def\K{\mathbb{K}}
\def\V{\mathbb{V}}
\def\W{\mathbb{W}}
\def\WW{\mathcal{W}}
\def\PP{\mathcal{P}}
\def\FF{\mathcal{F}}
\def\U{\mathbb{U}}
\def\C{\mathbb{C}}
\def\N{\mathbb{N}}
\def\Q{\mathbb{Q}}
\def\Z{\mathbb{Z}}

\def\BB{\mathcal{B}}
\def\AA{\mathcal{A}}
\def\EE{\mathcal{E}}
\def\CC{\mathcal{C}}
\def\OO{\mathcal{O}}
\def\DD{\mathcal{D}}

\def\e{\varepsilon}
\def\inn{\subseteq}

\def\S{\mathbb{S}}
\def\T{\mathbb{T}}
\def\l{\lambda}

\def\To{\Rightarrow}
\def\from{\leftarrow}
\def\From{\Leftarrow}

\DeclareMathOperator{\cl}{cl}
\DeclareMathOperator{\li}{li}
\DeclareMathOperator{\tl}{tl}
\DeclareMathOperator{\Id}{Id}
\DeclareMathOperator{\tr}{tr}
\DeclareMathOperator{\oo}{o}
\DeclareMathOperator{\spec}{spec}
\DeclareMathOperator{\mult}{mult}

\DeclareMathOperator{\iso}{iso}
\DeclareMathOperator{\mono}{mono}
\DeclareMathOperator{\epi}{epi}
\DeclareMathOperator{\adj}{adj}

\DeclareMathOperator{\Nu}{Nu}
\DeclareMathOperator{\Ima}{Im}
\DeclareMathOperator{\id}{id}
\DeclareMathOperator{\ze}{ze}

\DeclareMathOperator{\rg}{rg}

\DeclareMathOperator{\Hom}{Hom}
\DeclareMathOperator{\GL}{GL}
\DeclareMathOperator{\cont}{cont}
\DeclareMathOperator{\Hs}{H}

\DeclareMathOperator{\D}{D}
\DeclareMathOperator{\lcm}{lcm}

\DeclareMathOperator{\ev}{ev}
\DeclareMathOperator{\sg}{sg}

\date{}
\author{}

\begin{document}
\section*{Descomposición Primaria}
Si tenemos $f : \V \to \V$, y $m_f = \prod p_i$, con $\deg p_i > 0$ coprimos dos a dos, luego:
\begin{enumerate}
	\item $\V = \V_1 + \V_2 + \dots + \V_n$ si $\V_i = \ker p_i(f)$.
	\item $\V_i$ es $f$ invariante.
	\item $m_{f_{\V_i}} = p_i$
	\item Sea $\pi_i$ una proyección sobre $\V_i$, luego $\pi_i = g_i(f)$ para algún polinomio $g$.
\end{enumerate}

Es claro que $m_{f_{\V_i}} \mid p_i$, ya que por definición $p_i(f)|_{\V_i} = 0$. Además, como son todos coprimos, y sabemos que $\V = \sum \V_i$, entonces:
\[
	m_f = \prod m_{f_{\V_i}}
\]
Pero entonces $m_{f_{\V_i}} = p_i$

\section*{Notación}
Una factorización $m_f = \prod p_i$ con $p_i$ coprimo, se llama descomposición primaria, y el subespacio $\V_i = \ker p_i(f)$ se llama componente $p_i$-primaria.

\section*{Condición de Diagonalizabilidad}
Una transformación lineal $f : \V \to \V$ es diagonalizable $\iff$ $m_f = \prod (x-a_i)$ con todos coprimos.

\subsection*{($\To$)}
Sean $\{\l_{i \leq r}\}$ los autovalores sin repetición de $f$. Además $\V = \bigoplus_i \V_{\l_i}$

Sabemos además que $p_i = m_{f_{\V_{\l_i}}} = x - \l_i$. Luego
\[m_f = \lcm p_i = \prod p_i = \prod (x - \l_i)\].
Ya estamos.

\subsection*{($\From$)}
Supongamos que $m_f = \prod (x - \l_i)$, con $\l_i \neq \l_j$. Con lo anterior, sabmos que \[\V = \bigoplus \V_i\] con
\[\V_i = \ker p_i(f) = \ker (f - \l_if) = \V_{\l_i}\]
Luego $\V$ se escribe como suma de sus autoespacios. Ya estamos.

\section*{Nilpotentes}
Índice de nilpotencia. Dada $f : \V \to \V$, el índice de nilpotencia es el menor $n$ tal que $f^n = 0$.

Notemos que si $f$ es nilpotente, entonces $m_f = x^i$ donde $i$ es el índice de nilpotencia de $f$. Además $\chi_f = x^n$.

Esto implica que la única nilpotente diagonalizable es la nula.

\section*{Caracterización de Nilpotentes}
Si tenemos $f : \V \to \V$, luego es equivalente:
\begin{enumerate}
	\item $f$ nilpotente
	\item $f^n = 0$
	\item $m_f = x^k$
	\item $\chi_f = x^n$
\end{enumerate}
Son todas triviales.

\section*{Propiedad}
Sea $f : \V \to \V$ con índice de nilpotencia $n = \dim V$. Luego tenemos:
\begin{itemize}
	\item $\exists v : \langle v \rangle_f = \V$
	\item $\exists \BB : f_\BB = 
		\begin{bmatrix}
			0 & 0 & 0 & 0 \\
			1 & 0 & 0 & 0 \\
			0 & 1 & 0 & 0 \\
			0 & 0 & 1 & 0 \\
		\end{bmatrix}$, que se llama bloque de Jordan nilpotente de tamaño $n$.
\end{itemize}
Tomemos $v \notin \ker f^{n-1}$. Luego si tomamos $\BB = \{v, fv, f^2v, \dots, f^{n-1}v\}$. Luego esto es base y estamos.

\section*{Descomposición Jordan de Nilpotentes}
Dada $f : \V \to \V$ nilpotente, luego va a existir una base $\BB$ tal que:
\[
	[f]_\BB = 
	\begin{bmatrix}
		J_1 & 0 & 0 & 0 \\
		0 & J_2 & 0 & 0 \\
		0 & 0 & J_3 & 0 \\
		0 & 0 & 0 & J_4 \\
	\end{bmatrix}
\]
Dónde $J_i$ va a ser un bloque de Jordan nilpotente.

\subsection*{Lema}
Si tenemos $\{v_i\}$ li, con $f$ nilpotente. y $\langle v_i \rangle \cap \ker f^i = \{0\}$, luego $\{fv_i\}$ li y $\langle fv_i \rangle \cap \ker f^{i-1} = \{0\}$.

Sea $v \in \langle fv_i \rangle \cap \ker f^{i-1}$. Luego
\[
	f^{i-1} v = 0 =
\]
\[
	f^{i-1} \left(\sum \alpha_i fv_i\right) =
\]
\[
	f^i \left(\sum \alpha_i v_i\right)
\]
Luego $\alpha_i = 0$ , luego $v = 0$.

Supongamos $\sum \beta_i fv_i = 0$. Notemos que:
\[
	\sum \beta_i fv_i = 0
\]
\[
	f \left(\sum \beta_i v_i \right) = 0
\]
\[
	\sum \beta_i v_i \in \ker f^i
\]
\[
	\sum \beta_i v_i = 0
\]
\end{document}
