\documentclass{article}
\usepackage{amssymb}
\usepackage{mathtools}

\everymath{\displaystyle}
\setlength{\parskip}{3mm}
\setlength{\parindent}{0mm}

\def\R{\mathbb{R}}
\def\K{\mathbb{K}}
\def\V{\mathbb{V}}
\def\W{\mathbb{W}}
\def\WW{\mathcal{W}}
\def\PP{\mathcal{P}}
\def\FF{\mathcal{F}}
\def\U{\mathbb{U}}
\def\C{\mathbb{C}}
\def\N{\mathbb{N}}
\def\Q{\mathbb{Q}}
\def\Z{\mathbb{Z}}

\def\BB{\mathcal{B}}
\def\AA{\mathcal{A}}
\def\EE{\mathcal{E}}
\def\CC{\mathcal{C}}
\def\OO{\mathcal{O}}
\def\DD{\mathcal{D}}

\def\e{\varepsilon}
\def\inn{\subseteq}

\def\S{\mathbb{S}}
\def\T{\mathbb{T}}
\def\l{\lambda}

\def\To{\Rightarrow}
\def\from{\leftarrow}
\def\From{\Leftarrow}

\DeclareMathOperator{\cl}{cl}
\DeclareMathOperator{\li}{li}
\DeclareMathOperator{\tl}{tl}
\DeclareMathOperator{\Id}{Id}
\DeclareMathOperator{\tr}{tr}
\DeclareMathOperator{\oo}{o}
\DeclareMathOperator{\spec}{spec}
\DeclareMathOperator{\mult}{mult}

\DeclareMathOperator{\iso}{iso}
\DeclareMathOperator{\mono}{mono}
\DeclareMathOperator{\epi}{epi}
\DeclareMathOperator{\adj}{adj}

\DeclareMathOperator{\Nu}{Nu}
\DeclareMathOperator{\Ima}{Im}
\DeclareMathOperator{\id}{id}
\DeclareMathOperator{\ze}{ze}

\DeclareMathOperator{\rg}{rg}

\DeclareMathOperator{\Hom}{Hom}
\DeclareMathOperator{\GL}{GL}
\DeclareMathOperator{\cont}{cont}
\DeclareMathOperator{\Hs}{H}

\DeclareMathOperator{\D}{D}
\DeclareMathOperator{\lcm}{lcm}

\DeclareMathOperator{\ev}{ev}
\DeclareMathOperator{\sg}{sg}

\date{}
\author{}

\begin{document}
\section*{Polinomio Característico}
Dada $A \in \K^{n \times n}$, decimos que el polinomio caracerístico está dado por:
\[
    \chi_A(\lambda) = \det (\lambda I - A)
\]
Y sus raices son los autovalores de $A$.

\section*{Ejercicio 1}
decidir si
\[
    A = 
    \begin{bmatrix}
        0 & -2 & -3 \\
        -1 & 1 & -1 \\
        2 & 2 & 5
    \end{bmatrix}
\]
Es diagonalizable. el polinomio característico queda:
\[
    \det \lambda I - A = 
    \det \begin{bmatrix}
        \lambda & 2 & 3 \\
        1 & \lambda - 1 & 1 \\
        -2 & - 2 & \lambda -5
    \end{bmatrix} =\]\[
x((x-1)(x-5) + 2) - 2(x - 5+ 2) + 3(-2 + 2x - 2) =\]\[
(x-1)(x-2)(x+3)
\]
Luego la matriz es diagonalizable.

\section*{Dimensión del Autoespacio}
Se tiene $\dim \V_\lambda \leq \mult_\lambda(\chi_A)$, es decir.

\section*{$\chi$ de semejantes}
Tenemos
\[\det (\lambda I - B) = \]
\[\det (\lambda I - CAC^{-1}) = \]
\[\det (\lambda CIC^{-1} - CAC^{-1}) = \]
\[\det C \cdot \det (\lambda I - A) \cdot \det C^{-1} = \]
\[\det (\lambda I - A)\]
Luego $A$ y $B$ tienen el mismo característico.

\section*{Autovalores bajo aplicación de polinomios}
Dado cualquier poly $P \in \K[x]$, con $A \in \K^{n \times n}$, y $\lambda$ autovalor de $A$ con autovector $v$, luego tenemos $P \lambda$ es autovalor de $PA$ con autovector $v$. esto es ya que:
\[
    (PA)v = \left(\sum a_iA^i\right) v =
\]
\[
    \sum a_iA^iv =
\]
\[
    \sum a_i\lambda^iv =
\]
\[
    \left(\sum a_i\lambda^i\right)v = (P\lambda)v
\]

\subsection*{Vuelta}
Si estoy en $\K = \C$, con $P \in \C[x]$ con $\mu \in \C$ autovalor de $PA$, luego existe algún $\lambda$ autovalor de $A$, tal que $P\lambda = \mu$

Tomemos el polinomio $Q(x) = P(x) - \mu = a\prod x - \lambda_i$, y aplicandolo a la matrix $A$ tenemos:
\[
    QA = PA - I\mu = a \prod A - \lambda_i I
\]
Sabemos que $QA = PA - I \mu$ no es inversible, ya que osinó $\mu$ no es autovalor. Entonces no pueden ser todas las $A - \lambda_i I$ inversibles, ya que producto de inversibles es inversible. Luego ya está.
\end{document}
