\documentclass{article}
\usepackage{amssymb}
\usepackage{mathtools}

\everymath{\displaystyle}
\setlength{\parskip}{3mm}
\setlength{\parindent}{0mm}

\def\R{\mathbb{R}}
\def\K{\mathbb{K}}
\def\V{\mathbb{V}}
\def\W{\mathbb{W}}
\def\WW{\mathcal{W}}
\def\PP{\mathcal{P}}
\def\FF{\mathcal{F}}
\def\U{\mathbb{U}}
\def\C{\mathbb{C}}
\def\N{\mathbb{N}}
\def\Q{\mathbb{Q}}
\def\Z{\mathbb{Z}}

\def\BB{\mathcal{B}}
\def\AA{\mathcal{A}}
\def\EE{\mathcal{E}}
\def\CC{\mathcal{C}}
\def\OO{\mathcal{O}}
\def\DD{\mathcal{D}}

\def\e{\varepsilon}
\def\inn{\subseteq}

\def\S{\mathbb{S}}
\def\T{\mathbb{T}}
\def\l{\lambda}

\def\To{\Rightarrow}
\def\from{\leftarrow}
\def\From{\Leftarrow}

\DeclareMathOperator{\cl}{cl}
\DeclareMathOperator{\li}{li}
\DeclareMathOperator{\tl}{tl}
\DeclareMathOperator{\Id}{Id}
\DeclareMathOperator{\tr}{tr}
\DeclareMathOperator{\oo}{o}
\DeclareMathOperator{\spec}{spec}
\DeclareMathOperator{\mult}{mult}

\DeclareMathOperator{\iso}{iso}
\DeclareMathOperator{\mono}{mono}
\DeclareMathOperator{\epi}{epi}
\DeclareMathOperator{\adj}{adj}

\DeclareMathOperator{\Nu}{Nu}
\DeclareMathOperator{\Ima}{Im}
\DeclareMathOperator{\id}{id}
\DeclareMathOperator{\ze}{ze}

\DeclareMathOperator{\rg}{rg}

\DeclareMathOperator{\Hom}{Hom}
\DeclareMathOperator{\GL}{GL}
\DeclareMathOperator{\cont}{cont}
\DeclareMathOperator{\Hs}{H}

\DeclareMathOperator{\D}{D}
\DeclareMathOperator{\lcm}{lcm}

\DeclareMathOperator{\ev}{ev}
\DeclareMathOperator{\sg}{sg}

\date{}
\author{}

\title{Clase 2 de Álgebra Lineal}
\begin{document}
\maketitle
\section{Subespacio Propio}
Los subespacios de $\V$ dintintos de $\V$ y de $\{0\}$ (el subespacio nulo)
\subsection{Ej}
Si $\V = \R^{[0, 1]}$, y $S = \{f \in \V \mid f \; cont\}$, $S$ es sub. propio
\section{Intersección de Subespacios}
Si $S$ y $T$ son subespacios de $\V$, luego $S \cap T$ es sub. de $\V$, y en gral, si $\WW$
es un conjunto de subespacios, luego $\U = \bigcap_{\W \in \WW} \W$ es subespacio

Demo

Notemos que $0 \in \U$ ya que $0 \in \W \; \forall \W \in \WW$

Notemos que si $v, w \in \U$, luego para todo $\W \in \WW$, tenemos $v, w \in \W$, que
implica $v + w \in \W$ ya que es subespacio

Notemos que si $v \in \U$ y $\lambda \in \K$,
luego para todo $\W \in \WW$ tenemos $\lambda v \in \W$,
ya que es subespacio
\section{Unión de subespacios}
La unión de dos subespacios $\S \cap \T$ de $\V$ es sub
$\iff$ $\S \subseteq \T$ ó $\T \subseteq \S$

Demo

Si se cumple la pertenencia claramente es subespacio

Si no, tomemos $v \in \S - \T$, y $u \in \T - \S$, luego para que sea subespacio tenemos
que $v + u \in \S \cup \T$, wlog $u + v \in \S$,
pero $v \in \S$, luego $u \in \S$, contradicción.
\section{Combinaciones lineales}
Sea $\V$ un $\K$-espac vectorial, y sea $\emptyset \neq X \in \V$, luego
$\langle X \rangle = \left\{\sum_{i \leq n}\lambda_ix_i \;\middle|\;
\begin{matrix}
x_i \in X\\ \lambda_i \in \K\\ n \in \N
\end{matrix}
\right\}$
Nótese que todas las sumas son de finitos términos
\subsection{$\langle X \rangle$ es subespacio}
dado $X \in \V$, el generado por $X$ ($\langle X \rangle$) es un subespacio de $\V$ que contiene a $X$

Demo

Notemos que el $0 \in X$

Si $u, v \in \langle X \rangle$, luego $u = \sum \lambda_ix_i$, y
$v = \sum \mu_ix_i$ (los $x$s que aparecían en una y no en otra le pongo coeficiente $0$),
entonces $u + v = \sum (\lambda_i + \mu_i)x_i \in \langle X\rangle$

Si $v \in \langle X \rangle$, luego $v = \sum \lambda_i x_i$, luego para todo $\mu \in \K$
tenemos $\mu v = \sum (\mu\lambda_i) x_i \in \langle X \rangle$

\subsection{$\langle X \rangle$ es el subespacio más chico}
Supongamos que $X \subseteq \T$, $\T$ sub. Notemos que
$\langle X \rangle \subseteq \T$ ya que por ser subespacio, $\T$
contiene todas las combinaciones lineales de cosas en $X$

Decimos que $\langle\{\}\rangle = \{0\}$

\subsection{Otra definición de generador}
Sea 
$U = \bigcap_{
	\begin{matrix}
		S \text{ sub. } V \\
		X \subseteq S
	\end{matrix}
} S$, vamos a ver $U = \langle X\rangle$

Es claro que $X$ está contenido en la intersección pq está contenido en cada uno,
y además la intersección es subespacio pq es intersección de subespacios, luego $\langle X \rangle \subseteq U$

Además $\langle X \rangle$ es uno de los elementos de la intersección, luego
$U \subseteq \langle X\rangle$
\section{Dependencia Lineal}
Dado $X \subseteq \V$, $X = \{x_i\}$ es linearmente independiente si cuando
$\sum \lambda_i x_i = 0$, entonces $\lambda_i = 0$

Para $X$ infinito, se dice que $X$ es LI si todo subconjunto finito de $X$ es LI

Notemos que $\emptyset$ es LI, y que si $0 \in X$, luego X es LD

\subsection{Prop}
Si $X \subseteq Y$, entonces $X$ ld implica $Y$ ld, y $Y$ li implica $X$ li

\subsection{Independencia $\iff$ unicidad de representación}
Tenemos que $X$ es li $\iff$ para todos
$\sum \lambda_ix_i = \sum \mu_ix_i$ se tiene $\lambda_i = \mu_i$
\end{document}
