\documentclass{article}
\usepackage{amssymb}
\usepackage{mathtools}

\everymath{\displaystyle}
\setlength{\parskip}{3mm}
\setlength{\parindent}{0mm}

\def\R{\mathbb{R}}
\def\K{\mathbb{K}}
\def\V{\mathbb{V}}
\def\W{\mathbb{W}}
\def\WW{\mathcal{W}}
\def\PP{\mathcal{P}}
\def\FF{\mathcal{F}}
\def\U{\mathbb{U}}
\def\C{\mathbb{C}}
\def\N{\mathbb{N}}
\def\Q{\mathbb{Q}}
\def\Z{\mathbb{Z}}

\def\BB{\mathcal{B}}
\def\AA{\mathcal{A}}
\def\EE{\mathcal{E}}
\def\CC{\mathcal{C}}
\def\OO{\mathcal{O}}
\def\DD{\mathcal{D}}

\def\e{\varepsilon}
\def\inn{\subseteq}

\def\S{\mathbb{S}}
\def\T{\mathbb{T}}
\def\l{\lambda}

\def\To{\Rightarrow}
\def\from{\leftarrow}
\def\From{\Leftarrow}

\DeclareMathOperator{\cl}{cl}
\DeclareMathOperator{\li}{li}
\DeclareMathOperator{\tl}{tl}
\DeclareMathOperator{\Id}{Id}
\DeclareMathOperator{\tr}{tr}
\DeclareMathOperator{\oo}{o}
\DeclareMathOperator{\spec}{spec}
\DeclareMathOperator{\mult}{mult}

\DeclareMathOperator{\iso}{iso}
\DeclareMathOperator{\mono}{mono}
\DeclareMathOperator{\epi}{epi}
\DeclareMathOperator{\adj}{adj}

\DeclareMathOperator{\Nu}{Nu}
\DeclareMathOperator{\Ima}{Im}
\DeclareMathOperator{\id}{id}
\DeclareMathOperator{\ze}{ze}

\DeclareMathOperator{\rg}{rg}

\DeclareMathOperator{\Hom}{Hom}
\DeclareMathOperator{\GL}{GL}
\DeclareMathOperator{\cont}{cont}
\DeclareMathOperator{\Hs}{H}

\DeclareMathOperator{\D}{D}
\DeclareMathOperator{\lcm}{lcm}

\DeclareMathOperator{\ev}{ev}
\DeclareMathOperator{\sg}{sg}

\date{}
\author{}

\begin{document}
\section{Dependencia lineal}
\fbox{\begin{minipage}{\textwidth}
		\subsection{Lema 1}
Sea $X \subseteq \V$
Son equivalentes:
\begin{enumerate}
	\item $X$ es ld
	\item Exite un $x \in X$ tal que $\langle X\rangle = \langle X - x \rangle$
\end{enumerate}
\end{minipage}}
\subsection{1 $\to$ 2}
Tenemos $\sum \lambda_ix_i = 0$, y wlog $\lambda_0 \neq 0$, luego tenemos
$x_0 = -\lambda_0^{-1}\sum_{i \geq 1} \lambda_ix_i
= \sum_{i \geq 1} (-\lambda_0^{-1}\lambda_i)x_i$

Notemos que $\langle X - x_0 \rangle\subseteq \langle X \rangle$, ahora veamos para
el otro lado

Sea $v \in \langle X \rangle$, luego $v = \sum \mu_iw_i$, $w_i \in X$ wlog $w_0 = x_0$ ó
ninguno es $x_0$.

Si ninguno es $x_0$, luego trivialmente $v \in \langle X - x_0 \rangle$

Si $w_0 = x_0$, entonces tenemos $v = \mu_0w_0  + \sum_{i \geq 1} \mu_iw_i
= 
\mu_0\left(\sum_{i \geq 1} (-\lambda_0^{-1}\lambda_i)x_i\right)
+ \sum_{i \geq 1} \mu_iw_i
=
\sum_{i \geq 1} (-\mu_0\lambda_0^{-1}\lambda_i)x_i
+ \sum_{i \geq 1} \mu_iw_i
$, con $x_{i \geq 1}$, $w_{i \geq 1} \in X - x_0$, luego $v \in \langle X - x_0 \rangle$

\subsection{2 $\to$ 1}
Supongamos que $\langle X \rangle = \langle X - x \rangle$, luego en particular
$x \in \langle X - x \rangle$, luego se puede escribir $x = \sum \lambda_ix_i$, 
con $_i \neq x$ es decir $0 = x + \sum\lambda_ix_i$, osea que $X$ es ld

\section{Lema 2}
\fbox{\begin{minipage}{\textwidth}
sea $X \subseteq \V$, $X$ li; y sea $v \in \V$. Luego $X + v$ es li $\iff$
$v \nsubseteq \langle X \rangle$.
\end{minipage}}

Notemos que $v \in \langle X \rangle$ claramente implica $X + v$ ld.

Si $X + v$ es ld, luego tenemos $\sum \lambda_ix_i = 0$ con $\lambda_i \neq 0$ y
$x_i \in X + v$.

Notemos que no puede ser que $x_i \neq v$, ya que luego $X$ ld. Luego wlog $x_0 = v$,
entonces $\lambda_0v + \sum_{i \geq 1} \lambda_i x_i$, entonces $v \in \langle X \rangle$.

\section{Bases}
\fbox{\begin{minipage}{\textwidth}
Dado $X \in \V$, $X$ es base de $\V$ $\iff$ $\langle X \rangle = \V$ y $X$ li.
\end{minipage}}

\subsection{Prop}
Son equivalentes:
\begin{enumerate}
	\item $B$ es base de $\V$.
	\item $B$ es maximal bajo independencia lineal.
	\item $B$ es minimal bajo generación de $\V$.
\end{enumerate}

\subsubsection{(1) $\to$ (2)}
Claramente $B$ base implica $B$ li.
Además supongamos que $B \subsetneq C$. Queremos probar ld.

Tenemos entonces algún $c \in C, c \notin B$. Además como $B$ base tenemos $c$ es
combinación lineal de elementos de $B$, luego $B + c$ no puede ser li. Entonces como
$B + c \subseteq C$, tenemos $C$ ld.

\subsubsection{(2) $\to$ (3)}
Probemos que $\langle B \rangle = \V$. Supongamos que hay $v \in V, v \notin B$, luego
tenemos por Lema 1, tenemos que $B + v$ es li, contradicción porque era maximal.

Supongamos que $C \subsetneq B$ tal que $\langle C \rangle = V$, luego sea $v \in B - C$.
Notemos que $v \notin \langle C \rangle$ ya que osinó $B$ no sería li, entonces
$\langle C \rangle \neq \V$.

\subsubsection{(3) $\to$ (1)}
Ya sabemos que $\langle B \rangle = \V$, osea que hace falta probar $B$ es li
supongamos que no, luego por Lema 1, podemos sacar un vector $v \in B$ tal que
$\langle B - v \rangle = \langle B \rangle = \V$, que contradice la minimalidad
$\square$.

\section{Espacios Finitamente Generado}
un espacio vectorial $\V$ es finitamente generado si tiene un conjunto de
genradores finito.
\subsection{Lema}
Si $X$ finito y $\langle X \rangle = \V$, luego existe $B \subseteq X$ tal que $B$ es
base de $\V$

Sea $\Gamma = \{X' \subseteq X \mid \langle X' \rangle = \V \}$. 
Sabemos que $\Gamma$ es finito, y además es no vacío porque $X \in \Gamma$

Entonces existe $m = \min \{\#Y \mid Y \in \Gamma\} \in \N$, y además tomemos
$Y \in \Gamma$ tal que $\#Y = m$. Claramente es minimal bajo generación de $\V$,
entonces ya sabemos que es base.

\subsection{Clase que viene}
Si $\langle A \rangle = \V$ y $B \subseteq \V$ li, entonces $\# A \geq \#B$.

Esto implica que todas las bases tienen mismo tamaño.
\end{document}
