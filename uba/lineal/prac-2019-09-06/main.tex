\documentclass{article}
\usepackage{amssymb}
\usepackage{mathtools}

\everymath{\displaystyle}
\setlength{\parskip}{3mm}
\setlength{\parindent}{0mm}

\def\R{\mathbb{R}}
\def\K{\mathbb{K}}
\def\V{\mathbb{V}}
\def\W{\mathbb{W}}
\def\WW{\mathcal{W}}
\def\PP{\mathcal{P}}
\def\FF{\mathcal{F}}
\def\U{\mathbb{U}}
\def\C{\mathbb{C}}
\def\N{\mathbb{N}}
\def\Q{\mathbb{Q}}
\def\Z{\mathbb{Z}}

\def\BB{\mathcal{B}}
\def\AA{\mathcal{A}}
\def\EE{\mathcal{E}}
\def\CC{\mathcal{C}}
\def\OO{\mathcal{O}}
\def\DD{\mathcal{D}}

\def\e{\varepsilon}
\def\inn{\subseteq}

\def\S{\mathbb{S}}
\def\T{\mathbb{T}}
\def\l{\lambda}

\def\To{\Rightarrow}
\def\from{\leftarrow}
\def\From{\Leftarrow}

\DeclareMathOperator{\cl}{cl}
\DeclareMathOperator{\li}{li}
\DeclareMathOperator{\tl}{tl}
\DeclareMathOperator{\Id}{Id}
\DeclareMathOperator{\tr}{tr}
\DeclareMathOperator{\oo}{o}
\DeclareMathOperator{\spec}{spec}
\DeclareMathOperator{\mult}{mult}

\DeclareMathOperator{\iso}{iso}
\DeclareMathOperator{\mono}{mono}
\DeclareMathOperator{\epi}{epi}
\DeclareMathOperator{\adj}{adj}

\DeclareMathOperator{\Nu}{Nu}
\DeclareMathOperator{\Ima}{Im}
\DeclareMathOperator{\id}{id}
\DeclareMathOperator{\ze}{ze}

\DeclareMathOperator{\rg}{rg}

\DeclareMathOperator{\Hom}{Hom}
\DeclareMathOperator{\GL}{GL}
\DeclareMathOperator{\cont}{cont}
\DeclareMathOperator{\Hs}{H}

\DeclareMathOperator{\D}{D}
\DeclareMathOperator{\lcm}{lcm}

\DeclareMathOperator{\ev}{ev}
\DeclareMathOperator{\sg}{sg}

\date{}
\author{}

\begin{document}
\section{Ejercicio 1}
Sea $\K-\tl f: \V \to \W$. $\dim \V = n$, $\dim \W = m$. Demostrar que existen
$B, B'$ bases tal que
\[
	[f]_{BB'} = 
	\begin{pmatrix}
		0 & 0 & 0 & 0 & 0 \\
		0 & 0 & 0 & 0 & 0 \\
		0 & 0 & 1 & 0 & 0 \\
		0 & 0 & 0 & 1 & 0 \\
	\end{pmatrix}
\]

Tomo una base $v_{i \leq k}$ de $\Nu f$, y la extendemos a una base $v_{i \leq
n}$ de $\V$.

Notemos que si tomamos $w_i = f v_{k < i \leq n}$, estos son li, ya que sinó
habría una combinación lineal que dá 0, y por linearidad habría una combinación
lineal de los $v_{k < i \leq n}$ está en el núcleo $\Nu f$.

Entonces si tomamos $w_i = f v_i$ para $k < i \leq n$, y completamos la base
con cosas li, entonces ya estamos.

\section{Ejercicio 2}
Sea $p: \V \to \V$. Sea $S \inn \V$ sev. tal que $\Ima p \inn S$. Demostrar
que $\Ima p = S \iff \Nu p \cap S = \{0\}$

Notemos que la parte $(\To)$ es trivial. Tomemos $v \in \Nu f \cap \Ima f$.
Sea $x : fx = v$. Notemos $v = fx = f^2x = f(fx) = fv = 0$, luego $v = 0$.

Notemos además que para $(\From)$, tenemos que $\Nu p + \Ima p = \V$, pero
$\Ima f \inn S$, luego $\Nu p + S = \V$, pero como la intersección es $0$, por
lo que están en suma directa $\Nu f \oplus S$.

Por dimensión $\dim S = n - \dim \Nu f = \dim \Ima f$, luego $S = \Ima f$.

\section{Ejercicio 3}
Sea $S = \langle (1 \;0 \;-1 \;1), (0 \;1 \;0 \;0) \rangle$
Definir $p: \R^4 \to \R^4$, $p$ proyector tal que $p \neq 0$ y $\Ima p \inn S$
y $\Nu p \cap S \neq \{0\}$.

Tomemos $p (0\;1\;0\;0) = (0\;1\;0\;0)$, y $pE = 0$ donde $E$ es uno de los otros
cańonicos.

\section{Ejercicio 4}
Sea $\V$, $S, T \inn \V$, $S\cap T = \{0\}$, y tenemos:

$f : S \to \V$ mono

$g : \V \to T$ epi

$\Nu g = \Ima f$

Demostrar $\V = S \oplus T$.

Notemos que $\dim S = \dim S - \dim \Nu f = \dim \Ima f = \dim \Nu g =
n - \dim \Ima g = n - \dim T$, entonces ya estás.

\section{Ejercicio 5}
Sea $f : \R^9 \to \R^9 : \dim \Ima f = 6, \Nu f \inn \Ima f$.

Hallar $\dim \Ima f^2$

\section{Ejercicio 6}
Sea $f : \V \to \W$, $g : \V \to \U$, todos los espacios de dimensión finita.
Probar que son equivalentes:
\begin{itemize}
	\item Existe $h : \W \to \U$ tal que $h \circ f = g$
	\item $\Nu f \inn \Nu g$
\end{itemize}

$a \To b$ es trivial.

$b \To a$, tomemos base $B = B_f \cap B_g \cap B_v$ de $\V$, tal que $B_f$ es
base de $\Nu f$ y $B_f \cap B_g$ es base de $\Nu g$.

Notemos que $f (B_g \cap B_v)$ es base de $\W$. Luego tomemos $h(fB_g) = 0$ y
$h(fB_v) = gB_v$.
\end{document}
