\documentclass{article}
\usepackage{amssymb}
\usepackage{mathtools}

\everymath{\displaystyle}
\setlength{\parskip}{3mm}
\setlength{\parindent}{0mm}

\def\R{\mathbb{R}}
\def\K{\mathbb{K}}
\def\V{\mathbb{V}}
\def\W{\mathbb{W}}
\def\WW{\mathcal{W}}
\def\PP{\mathcal{P}}
\def\FF{\mathcal{F}}
\def\U{\mathbb{U}}
\def\C{\mathbb{C}}
\def\N{\mathbb{N}}
\def\Q{\mathbb{Q}}
\def\Z{\mathbb{Z}}

\def\BB{\mathcal{B}}
\def\AA{\mathcal{A}}
\def\EE{\mathcal{E}}
\def\CC{\mathcal{C}}
\def\OO{\mathcal{O}}
\def\DD{\mathcal{D}}

\def\e{\varepsilon}
\def\inn{\subseteq}

\def\S{\mathbb{S}}
\def\T{\mathbb{T}}
\def\l{\lambda}

\def\To{\Rightarrow}
\def\from{\leftarrow}
\def\From{\Leftarrow}

\DeclareMathOperator{\cl}{cl}
\DeclareMathOperator{\li}{li}
\DeclareMathOperator{\tl}{tl}
\DeclareMathOperator{\Id}{Id}
\DeclareMathOperator{\tr}{tr}
\DeclareMathOperator{\oo}{o}
\DeclareMathOperator{\spec}{spec}
\DeclareMathOperator{\mult}{mult}

\DeclareMathOperator{\iso}{iso}
\DeclareMathOperator{\mono}{mono}
\DeclareMathOperator{\epi}{epi}
\DeclareMathOperator{\adj}{adj}

\DeclareMathOperator{\Nu}{Nu}
\DeclareMathOperator{\Ima}{Im}
\DeclareMathOperator{\id}{id}
\DeclareMathOperator{\ze}{ze}

\DeclareMathOperator{\rg}{rg}

\DeclareMathOperator{\Hom}{Hom}
\DeclareMathOperator{\GL}{GL}
\DeclareMathOperator{\cont}{cont}
\DeclareMathOperator{\Hs}{H}

\DeclareMathOperator{\D}{D}
\DeclareMathOperator{\lcm}{lcm}

\DeclareMathOperator{\ev}{ev}
\DeclareMathOperator{\sg}{sg}

\date{}
\author{}


\begin{document}
\section{Rec}
$B \subseteq \V$ es base $\iff$ $B$ es li y $\langle B \rangle = \V$
\section{Ejercicio 1}
Dado $\V = \K_n[x] =  \{p \in \K[x] \mid p = 0 \text{ ó } \deg(p) \leq n\}$.

Si $B = \{p \mid \deg(p) = i | 0 \leq i \leq n\}$, demostrar que $B$ es base.
\subsection{Demo 1}
La matriz en $E = \{x^i \mid 0 \leq i \leq n\}$ es triangular.
\subsection{Demo 2}
Inducción. $n = 0$ es trivial.

Si $n \geq 1$, supongamos $\sum \lambda_iB_i = 0$. El coeficiente $n$ de ese polinomio
tiene que ser cero, luego $\lambda_n = 0$, luego $0 = \sum_{i < n} \lambda_iB_i = 0$,
pero esto no puede pasar por la hipótesis inductiva.

\section{Ejercicio 8}
Tomemos $S = \left\{A \in \K^{n \times n} \; \middle| \; \sum_j A_{ij} = 0 = \sum_i A_{ij}\right\}$

Calcular $\dim S$. Notemos que si dejamos libre los $A_{ij}$ con $i < n$ y $j < n$ y
tomamos $A_{nj} = -\sum_{i < n} A_{ij}$, y $A_{in} = -\sum_{j < n} A_{ij}$, y
$A_{nn} = -\sum_{i,j < n} A_{ij}$, esta matriz cumple, y son todas las que cumplen.

Luego da $\dim S = (n-1)^2$

\section{Intersección y suma de subespacios}
Dados $S$, $T$ sub. de $\V$, se tiene
\begin{itemize}
	\item $S \cap T = S \cap T$ 
	\item $S + T = \{s + t \mid s \in S, t \in T\}$
	\item $S \oplus T = S + T$ si están en suma directa, que se da cuando:
		\begin{itemize}
			\item $S \cap T = {0} \iff$
			\item $\dim (S + T) = \dim S + \dim T \iff$
			\item $\forall v \in S + T \; \exists! \; s \in S, t\in T \mid s + t = v$
		\end{itemize}
\end{itemize}

\section{Complemento}
Cuando $S \oplus T = \V$, decimos que $S$ es complemento de $T$. Se sabe que eso
implica $\dim S = n - \dim T$, es decir la dimensión de los complementos es toda la
misma
\end{document}
