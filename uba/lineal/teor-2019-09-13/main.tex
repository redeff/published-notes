\documentclass{article}
\usepackage{amssymb}
\usepackage{mathtools}

\everymath{\displaystyle}
\setlength{\parskip}{3mm}
\setlength{\parindent}{0mm}

\def\R{\mathbb{R}}
\def\K{\mathbb{K}}
\def\V{\mathbb{V}}
\def\W{\mathbb{W}}
\def\WW{\mathcal{W}}
\def\PP{\mathcal{P}}
\def\FF{\mathcal{F}}
\def\U{\mathbb{U}}
\def\C{\mathbb{C}}
\def\N{\mathbb{N}}
\def\Q{\mathbb{Q}}
\def\Z{\mathbb{Z}}

\def\BB{\mathcal{B}}
\def\AA{\mathcal{A}}
\def\EE{\mathcal{E}}
\def\CC{\mathcal{C}}
\def\OO{\mathcal{O}}
\def\DD{\mathcal{D}}

\def\e{\varepsilon}
\def\inn{\subseteq}

\def\S{\mathbb{S}}
\def\T{\mathbb{T}}
\def\l{\lambda}

\def\To{\Rightarrow}
\def\from{\leftarrow}
\def\From{\Leftarrow}

\DeclareMathOperator{\cl}{cl}
\DeclareMathOperator{\li}{li}
\DeclareMathOperator{\tl}{tl}
\DeclareMathOperator{\Id}{Id}
\DeclareMathOperator{\tr}{tr}
\DeclareMathOperator{\oo}{o}
\DeclareMathOperator{\spec}{spec}
\DeclareMathOperator{\mult}{mult}

\DeclareMathOperator{\iso}{iso}
\DeclareMathOperator{\mono}{mono}
\DeclareMathOperator{\epi}{epi}
\DeclareMathOperator{\adj}{adj}

\DeclareMathOperator{\Nu}{Nu}
\DeclareMathOperator{\Ima}{Im}
\DeclareMathOperator{\id}{id}
\DeclareMathOperator{\ze}{ze}

\DeclareMathOperator{\rg}{rg}

\DeclareMathOperator{\Hom}{Hom}
\DeclareMathOperator{\GL}{GL}
\DeclareMathOperator{\cont}{cont}
\DeclareMathOperator{\Hs}{H}

\DeclareMathOperator{\D}{D}
\DeclareMathOperator{\lcm}{lcm}

\DeclareMathOperator{\ev}{ev}
\DeclareMathOperator{\sg}{sg}

\date{}
\author{}

\begin{document}
	\section*{Espacio Dual}
	Dado $\V$ un $\K$-ev, llamo el \emph{dual} de $\V$ como $\V^* =
	\Hom_\K(\V,\K)$.
	Si $\dim \V = n$, luego tenemos $\dim \V^* = n \cdot 1 = n$.

	A los elementos $f \in \V^*$ se los llama \emph{funcionales lineales}.

	\section*{Base de $\V^*$}
	Dada una base $\BB$, definimos una familia de $n$ transformaciones
	lineales indejadas por $1 \leq j \leq n$, definidas por:
	$\phi_j = \BB_i \mapsto
	\begin{cases}
		1 & \text{Si $i = j$} \\
		0 & \text{osino}
	\end{cases}
	$. 

	Vamos a probar que $\phi$ es base, y vamos a llamarla la \emph{base dual
	de $\BB$}, o $\BB^*$.

	Demo: Como $\dim \V^* = n$, alcanza con probar que los $\phi$ son $\li$.

	Supongamos que $0 = \sum \lambda_i\phi_i$, luego evaluamos esta función nula
	en la base, y tenemos, para cada vector de la base $\BB_i$, se cumple

	\[
		0\;\BB_i = \sum \lambda_j \cdot \phi_j \BB_i
	\]
	Notemos que se anula para $i \neq j$, luego 
	\[
		0\;\BB_i = \lambda_i \cdot \phi_i \BB_i
	\]
	\[
		0 = \lambda_i \cdot 1
	\]
	\[
		0 = \lambda_i
	\]
	Entonces es li.

	\section*{Expresar elementos de $\V^*$ en una base}
	Dado un $\psi \in \V^*$. Se puede expresar como
	\[
		\psi = \sum \lambda_i \cdot \BB^*_i
	\]
	Para cada $\BB_j$, tomemos
	\[
		\psi = \sum \lambda_i \cdot \BB^*_i
	\]
	\[
		\psi\BB_j = \sum \lambda_i \cdot \BB^*_i \BB_j
	\]
	\[
		\psi\BB_j = \lambda_j \cdot \BB^*_j \BB_j
	\]
	\[
		\psi\BB_j = \lambda_j
	\]
	Luego los coeficientes en la base dual, son las imágenes de la base
	original.

	\section*{Bases diferentes}
	Sea $v$ y $w$ bases de $\V$, y sea $\phi = v^*$, $\psi = w^*$
	luego tenemos que:
	\[
		C_{\psi\phi} = (C_{vw})^T
	\]
	Sea $C = C_{vw}$. Tenemos que $v_j = \sum C_{ij}w_i$.

	Notemos además que $\psi_iv_j = C_{ij}$. Acordémonos que esto significa que
	$\psi_i$.

	Notemos además que $\psi_i = \sum (\psi_i v_j) \cdot \phi_j = \sum C_{ij}
	\phi_j$. Luego estamos.

	\section*{Dual del Dual}
	Dado un $\V$, si tomamos su doble dual $\V^{**}$, nos podemos definit la
	función:

	$\ev : \V \to (\V^* \to \K) = \V^{**}, \ev = v \mapsto \phi \mapsto
	\phi v$.

	Notemos que $\ev v$ es lineal ya que $(\alpha \cdot \phi + \beta \cdot \psi)v = 
	\alpha \cdot \phi v + \beta \cdot \psi v = \alpha \cdot \ev \phi + \beta
	\cdot \ev \psi
	$.

	Además notemos que $\ev$ es lineal ya que \[
		\ev (\alpha \cdot v + \beta \cdot w)
	\]\[
		= \phi \mapsto \phi (\alpha \cdot v + \beta \cdot w)
	\]\[
		= \phi \mapsto \alpha \cdot \phi v + \beta \cdot \phi w
	\]\[
		= \phi \mapsto \alpha \cdot \ev v \phi + \beta \cdot \ev w \phi
	\]\[
		= \alpha \cdot \ev v + \beta \cdot \ev w \]

	\section*{$\ev$ es mono}
	Supongamos que $\ev_v = 0$, es decir $\forall \phi: \ev_v \phi = 0$, es
	decir, $\forall \phi: \phi v = 0$.

	Si $v \neq 0$, luego $\li \{v\}$, luego existe una base $v = \BB_1$ de $\V$,
	luego tomemos $\BB^*_1$. Por definición $\BB^*_1v = 1$, pero contradicción ,
	ya que $\phi v = 0$ para todo $\phi$.

	\section*{$\ev$ es epi en finitos}
	Si $\dim \V = n$, luego $\ev$ es epi

	\section*{Anuladores}
	Dado $X \inn \V$, decimos que el anulador de $X$, notado $X^0 \inn \V^*$, es
	el dado por
	\[X^0 = \{\phi \in \V^* : \forall x \in X : fx = 0 \in \K \}\]

	\subsection*{Prop}
	\begin{enumerate}
		\item $\{0\}^0 = \V^*$ y $\V^0 = \{0\}$
		\item $X \inn Y \To Y^0 \inn X^0$
		\item $X^0 = \langle X \rangle ^0$
		\item $S, T \inn \V$ subespacios, luego:
			\begin{itemize}
				\item $(S + T)^0 = S^0 \cap T^0$
				\item $S^0 + T^0 \inn (S \cap T)^0$
			\end{itemize}
	\end{enumerate}
	\subsection*{Demo}
	\begin{enumerate}
		\item Trivial
		\item Sea $f \inn Y^0$, queremos probar que si $x \in X$, luego $fx = 0$,
			pero eso es verdad ya que $x \in X \inn Y$.
		\item Queremos probar que si $f \in X^0$, luego $f \in \langle X^0
			\rangle$. Sea $v = \sum \lambda_i \cdot x_i \in \langle X
			\rangle$, luego
			tenemos que $f v = \sum \lambda_i \cdot fx_i = 0$, entonces $f \in
			\langle X \rangle ^ 0$
		\item $S, T \inn \V$ subespacios, luego:
			\begin{itemize}
				\item Si tenemos $f \in S^0 \cap T^0$, queremos probar que $f \in
					(S+T)^0$. Sea $s + t \in S + T$, luego $f (s+t) = fs + ft = 0$,
					luego estamos.
				\item Es trivial por el punto $2$.
			\end{itemize}
	\end{enumerate}

	\section*{Dinsion $S^0$}
	Si tenemos $\dim \V = n$, luego tomemos $S$ subesp de $\V$. Demostremos que
	$\dim S + \dim S^0 = n$.

	Sea $r = \dim S$.

	Sea $v$ una base de $\V$ tal que $v_{i < r}$ es base de $S$, y sea $\phi$ se
	base dual, luego una $\psi = \sum \lambda_i \cdot \phi_i \in S^0 \iff
	\forall i < r : \psi v_i = 0 \iff \forall i < r : \lambda_i = 0$, luego
	tenemos que $S^0 = \langle \phi_{i \geq r} \rangle$.
\end{document}
