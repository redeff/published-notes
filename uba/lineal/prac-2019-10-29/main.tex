\documentclass{article}
\usepackage{amssymb}
\usepackage{mathtools}

\everymath{\displaystyle}
\setlength{\parskip}{3mm}
\setlength{\parindent}{0mm}

\def\R{\mathbb{R}}
\def\K{\mathbb{K}}
\def\V{\mathbb{V}}
\def\W{\mathbb{W}}
\def\WW{\mathcal{W}}
\def\PP{\mathcal{P}}
\def\FF{\mathcal{F}}
\def\U{\mathbb{U}}
\def\C{\mathbb{C}}
\def\N{\mathbb{N}}
\def\Q{\mathbb{Q}}
\def\Z{\mathbb{Z}}

\def\BB{\mathcal{B}}
\def\AA{\mathcal{A}}
\def\EE{\mathcal{E}}
\def\CC{\mathcal{C}}
\def\OO{\mathcal{O}}
\def\DD{\mathcal{D}}

\def\e{\varepsilon}
\def\inn{\subseteq}

\def\S{\mathbb{S}}
\def\T{\mathbb{T}}
\def\l{\lambda}

\def\To{\Rightarrow}
\def\from{\leftarrow}
\def\From{\Leftarrow}

\DeclareMathOperator{\cl}{cl}
\DeclareMathOperator{\li}{li}
\DeclareMathOperator{\tl}{tl}
\DeclareMathOperator{\Id}{Id}
\DeclareMathOperator{\tr}{tr}
\DeclareMathOperator{\oo}{o}
\DeclareMathOperator{\spec}{spec}
\DeclareMathOperator{\mult}{mult}

\DeclareMathOperator{\iso}{iso}
\DeclareMathOperator{\mono}{mono}
\DeclareMathOperator{\epi}{epi}
\DeclareMathOperator{\adj}{adj}

\DeclareMathOperator{\Nu}{Nu}
\DeclareMathOperator{\Ima}{Im}
\DeclareMathOperator{\id}{id}
\DeclareMathOperator{\ze}{ze}

\DeclareMathOperator{\rg}{rg}

\DeclareMathOperator{\Hom}{Hom}
\DeclareMathOperator{\GL}{GL}
\DeclareMathOperator{\cont}{cont}
\DeclareMathOperator{\Hs}{H}

\DeclareMathOperator{\D}{D}
\DeclareMathOperator{\lcm}{lcm}

\DeclareMathOperator{\ev}{ev}
\DeclareMathOperator{\sg}{sg}

\date{}
\author{}

\begin{document}
\section*{Subespacios $f$-invariantes}
Son los $S \in \V : f(S) \inn S$.

\section*{Imagen de un polinomio}
Sea $p \in k[x]$, luego la imagen y el núcleo de $p(f)$ son $f$-invariantes, ya que tenemos $f(p(f)v) = p(f)(fv) \in \Ima p(f)$

\section*{Diagonalizabilidad de las Restringidas}
Sea $S$ $f$-invariante, luego tenemos que $f_S$ es diagonalizable.

Por diagonalizabilidad, tenemos que $\V = \bigoplus \V_{\l_i}$
Tomemos entonces $\{v_i\}_{i \leq n}$ una base de autovectores.
Sea $w = \sum \alpha_iv_i \in S$, luego:
\[
	f(w) \in S
\]
\[
	\sum \alpha_i f(v_i) \in S
\]
\[
	\sum \alpha_i \lambda_i v_i \in S
\]

\section*{Ejercicio}
Son equivalentes:
\begin{enumerate}
	\item Los únicos $f$-invariantes son $\{0\}$ y $\V$.
	\item Todo vector $v \neq 0$ es cíclico.
	\item $\chi_f$ es irreducible.
\end{enumerate}
Para 3 $\To$ 1, tenemos $S \in \V$ $f$-invariante, luego $\chi_{f_S} \mid \chi_f$, pero entonces no es irreducible, estamos.

Para 2 $\To$ 3, tenemos que ara $v \neq 0$, $\deg m_{f_v} = n$, ya que $v$ es cíclico, luego $m_{f_v} \mid m_f \mid \chi_f$, y tienen todos el mismo grado. Luego son iguales.

Entonces como $m_f = \chi_f$, si asumimos $m_f = \chi_f = p \cdot q$, con $p$ y $q$ no triviales, luego ninguno de $\ker p(f)$ y $\ker q(f)$ son triviales, y ambos son $f$-invariantes. Luego estamos.
\end{document}
