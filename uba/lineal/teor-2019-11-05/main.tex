\documentclass{article}
\usepackage{amssymb}
\usepackage{mathtools}

\everymath{\displaystyle}
\setlength{\parskip}{3mm}
\setlength{\parindent}{0mm}

\def\R{\mathbb{R}}
\def\K{\mathbb{K}}
\def\V{\mathbb{V}}
\def\W{\mathbb{W}}
\def\WW{\mathcal{W}}
\def\PP{\mathcal{P}}
\def\FF{\mathcal{F}}
\def\U{\mathbb{U}}
\def\C{\mathbb{C}}
\def\N{\mathbb{N}}
\def\Q{\mathbb{Q}}
\def\Z{\mathbb{Z}}

\def\BB{\mathcal{B}}
\def\AA{\mathcal{A}}
\def\EE{\mathcal{E}}
\def\CC{\mathcal{C}}
\def\OO{\mathcal{O}}
\def\DD{\mathcal{D}}

\def\e{\varepsilon}
\def\inn{\subseteq}

\def\S{\mathbb{S}}
\def\T{\mathbb{T}}
\def\l{\lambda}

\def\To{\Rightarrow}
\def\from{\leftarrow}
\def\From{\Leftarrow}

\DeclareMathOperator{\cl}{cl}
\DeclareMathOperator{\li}{li}
\DeclareMathOperator{\tl}{tl}
\DeclareMathOperator{\Id}{Id}
\DeclareMathOperator{\tr}{tr}
\DeclareMathOperator{\oo}{o}
\DeclareMathOperator{\spec}{spec}
\DeclareMathOperator{\mult}{mult}

\DeclareMathOperator{\iso}{iso}
\DeclareMathOperator{\mono}{mono}
\DeclareMathOperator{\epi}{epi}
\DeclareMathOperator{\adj}{adj}

\DeclareMathOperator{\Nu}{Nu}
\DeclareMathOperator{\Ima}{Im}
\DeclareMathOperator{\id}{id}
\DeclareMathOperator{\ze}{ze}

\DeclareMathOperator{\rg}{rg}

\DeclareMathOperator{\Hom}{Hom}
\DeclareMathOperator{\GL}{GL}
\DeclareMathOperator{\cont}{cont}
\DeclareMathOperator{\Hs}{H}

\DeclareMathOperator{\D}{D}
\DeclareMathOperator{\lcm}{lcm}

\DeclareMathOperator{\ev}{ev}
\DeclareMathOperator{\sg}{sg}

\date{}
\author{}

\begin{document}
\section*{Cantidad de Bloques de Cierto Tamaño}
Sea $b_i$ la cantidad de bloques de tamaño mayor a $i$. Teníamos que:
\[
	b_i = \rg A^i - \rg A^{i+1}
\]
Luego, si decimos $c_i$ es la cantidad de bloques de tamaño $i$, tenemos:
\[
	c_i = b_{i-1} - b_i = \rg A^{i-1} - 2\rg A^i + \rg A^{i+1}
\]
\section*{Unicidad de la forma de Jordan}
Sale trivial por los tamaños de la cantidad de bloques del título anterior.

\section*{Jordanes con minimal $(x - \l)^r$}
Si tenemos $f : \V \to \V$ con $m_f = (x-\l)^r$, luego tenemos que $f - \l \id$ es una $\tl$ de índice de nilpotencia $r$

Luego por Jordan en Nilpotentes, tenemos que:
\[
	[f-\l \id]_\BB = J
\]
Para una matriz de Jordan nilpotente $J$.
\[
	[f-\l \id]_\BB = J
\]
\[
	[f]_\BB - \l [\id]_\BB = J
\]
\[
	[f]_\BB - \l I = J
\]
\[
	[f]_\BB = J + \l I = J(\l)
\]

\section*{Caso General}
Si tenemos $f : \V \to \V$ con $m_f = \prod (x - \l_i)^{r_i}$, con $(\l_i)$ autovalores distintos.

Por descomposición primaria, $\V = \bigoplus \V_i$, con $\V_i = \ker (f - \l_i)^{r_i}$.

Luego tomemos $\BB_i$ una base de jordán de $f|_{\V_i}$, ya que sabemos que $\V_i$ es $f$-invariante, y $m_{f|_{\V_i}} = (f - \l_i \id)^{r_i}$.

Luego sabemos $[f|_{\V_i}]_{\BB_i} = J(\l_i)$

Si tomamos entonces $\BB = \bigcup \BB_i$, queda:
\[
	[f]_\BB =
	\begin{bmatrix}
		J(\l_1) & 0 & 0 & 0 & 0 \\
		0 & J(\l_2) & 0 & 0 & 0 \\
		0 & 0 & J(\l_3) & 0 & 0 \\
		0 & 0 & 0 & J(\l_4) & 0 \\
		0 & 0 & 0 & 0 & J(\l_5) \\
	\end{bmatrix}
\]
\end{document}
