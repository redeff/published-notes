\documentclass{article}
\usepackage{amssymb}
\usepackage{mathtools}

\everymath{\displaystyle}
\setlength{\parskip}{3mm}
\setlength{\parindent}{0mm}

\def\R{\mathbb{R}}
\def\K{\mathbb{K}}
\def\V{\mathbb{V}}
\def\W{\mathbb{W}}
\def\WW{\mathcal{W}}
\def\PP{\mathcal{P}}
\def\FF{\mathcal{F}}
\def\U{\mathbb{U}}
\def\C{\mathbb{C}}
\def\N{\mathbb{N}}
\def\Q{\mathbb{Q}}
\def\Z{\mathbb{Z}}

\def\BB{\mathcal{B}}
\def\AA{\mathcal{A}}
\def\EE{\mathcal{E}}
\def\CC{\mathcal{C}}
\def\OO{\mathcal{O}}
\def\DD{\mathcal{D}}

\def\e{\varepsilon}
\def\inn{\subseteq}

\def\S{\mathbb{S}}
\def\T{\mathbb{T}}
\def\l{\lambda}

\def\To{\Rightarrow}
\def\from{\leftarrow}
\def\From{\Leftarrow}

\DeclareMathOperator{\cl}{cl}
\DeclareMathOperator{\li}{li}
\DeclareMathOperator{\tl}{tl}
\DeclareMathOperator{\Id}{Id}
\DeclareMathOperator{\tr}{tr}
\DeclareMathOperator{\oo}{o}
\DeclareMathOperator{\spec}{spec}
\DeclareMathOperator{\mult}{mult}

\DeclareMathOperator{\iso}{iso}
\DeclareMathOperator{\mono}{mono}
\DeclareMathOperator{\epi}{epi}
\DeclareMathOperator{\adj}{adj}

\DeclareMathOperator{\Nu}{Nu}
\DeclareMathOperator{\Ima}{Im}
\DeclareMathOperator{\id}{id}
\DeclareMathOperator{\ze}{ze}

\DeclareMathOperator{\rg}{rg}

\DeclareMathOperator{\Hom}{Hom}
\DeclareMathOperator{\GL}{GL}
\DeclareMathOperator{\cont}{cont}
\DeclareMathOperator{\Hs}{H}

\DeclareMathOperator{\D}{D}
\DeclareMathOperator{\lcm}{lcm}

\DeclareMathOperator{\ev}{ev}
\DeclareMathOperator{\sg}{sg}

\date{}
\author{}

\begin{document}
\section*{Plinomio minimal}
Dada $f : \V \to \V$, el \emph{polinomio minimal} $m_f$ es el polinomio mónico de menor grado, tal que $m_f(f) = 0$ (la tl nula).

Por ejemplo, si $\dim V > 0$, para $f = \id$, tenemos que $m_f(x) = x-1$
y para $f = 0$ tenemos $m_f(x) = 0$

\subsection*{Caso $\V = \{0\}$}
Es ese caso, la única función $f : \V \to \V$ es la tl nula, que también es la identidad, luego $m_f(x) = 1$, ya que $\id = 0$

\section*{Se rompe para $\V$ infinito}
Si tenemos $f : k[x] \to k[x]$, $f(p(x)) = xp(x)$, luego tenmos que:
\[
	Q(f)(p(x)) = \sum a_i x^i p(x) = Q(x)p(x)
\]
Luego ningún no nulo $Q$ multiplicado por $f$ da la tl nula.

\section*{Subespacios invariantes}
Dado $\dim \V < \infty$ y $f : \V \to \V$, luego $S$ sub de $\V$ es \emph{$f$-invariante} sii $f(S) \inn S$.

Se dice que $f_S$ es la restricción de $f$ al subespacio $S$, con $f_S:S\to S$

\subsection*{Subespacio $f$-cíclico de $v$}
Dado un $v$ no nulo en $\V$, tomemos $S = \langle f^iv \rangle_{i \geq 0}$, luego $S$ es claramente $f$-invariante. Es te subsespacio se nota $\langle v\rangle_f$

Decimos que $f_v = f_{\langle v \rangle_f}$

\section*{Polinomio minimales sobre restringidos}
Para cada polinomio $p(x) \in k[x]$, luego
\[p(f_v) = 0 \iff p(f_v)(v) = 0\]
Y además, si $\dim V < \infty$, luego existe un único polinomio mónico $m_{f_v}(x) \in k[x]$ tal que:
\begin{enumerate}
	\item $m_{f_v}(f)(v) = 0$
	\item $\forall p \in k[x] : p(f)(v) = 0 \To m_{f_v}(x) \mid p(x)$
	\item $\deg m_{f_v}(x) = \dim \langle v \rangle_f$.
\end{enumerate}

Demo:
Supongamos que $p(f_v)(v) = 0$, luego veamos que todo $w \in \langle v \rangle_f$ cumple $p(f_v)(w) = 0$.

Tenemos
\[w = \sum a_i f^iv\]
Tomemos \[q(x) = \sum a_i x^i\]
Luego
\[
	w = q(f)(v) = q(f_v)(v)
\]
Luego
\[
	p(f_v)(w) = p(f_v)(q(f_v)(w))
\]
\[
	p(f_v)(w) = (p(f_v) \cdot q(f_v))(w)
\]
\[
	p(f_v)(w) = q(f_v)(p(f_v)(w))
\]
\[
	p(f_v)(w) = q(f_v)(0) = 0
\]
Y cualquier tl en $0$ da $0$.

Demo:
Tomemos $m_{f_v}$ el polinomio minimal de $f_v$. Veamos que $m_{f_v}(f)(v) = 0$. Notemos que $m_{f_v}(f)(v) = m_{f_v}(f_v)(v)$, ya que solo lo evalúo en $v$, luego es trivial.

Para la segunda parte, tenemos que $0 = p(f)(v) = p(f_v)(v)$, pero como esto es $0$, tenemos por los que demostramos $p(f_v) = 0$, luego $p$ está en el ideal del minimal, luego $m_{f_v}(x) \mid p(x)$.

Para la tercera parte, tomemos $d = \deg m_{f_v}$:
\[
	m_{f_v}(f_v)(v) = 0 = \sum a_if^i(v)
\]
Luego
\[
	f^d(v) = -\sum_{i < d} a_if^i(v)
\]
Luego $\langle v \rangle_f = \langle v, fv, \dots, f^{d-1}v \rangle$, luego tenemos que $\dim \langle v \rangle_f \leq d$

Pero además estos son li, ya que si fueran ld podríamos usar la combinación lineal que da $0$ para construirnos un polinomio más chico que el minimal.

Luego $\dim \langle v \rangle_f = d$.

Tomemos entonces $\BB = \{v, fv, \dots, f^{d-1}v \}$ base del subespacio cíclico. Veamos cómo es $[f]_\BB$:
\[
	[f]_\BB = \begin{bmatrix}
		0 & 0 & 0&\dots & 0&-a_0 \\
		1 & 0 & 0&\dots & 0&-a_1 \\
		0 & 1 & 0&\dots & 0&-a_2 \\
		0 & 0 & 1&\dots & 0&-a_3 \\
		\vdots &
		\vdots &
		\vdots &
		\ddots &
		\vdots &
		\vdots \\
		0 & 0 & 0&\dots & 1&-a_{d-1} \\
	\end{bmatrix}
\]
Que es la matriz compañera del polinomio característico.

\section*{Mínimo común múltiplo}
Sea $f : \V \to \V$, con $\W\inn\V$ $f$-invariante. Luego $m_{f_\W} \mid m_f$.

También tenemos
\begin{enumerate}
	\item $m_f = \lcm \{m_{f_v} : v \in \V\}$
	\item Si $\V = \V_1 + \dots + \V_2$, luego $m_f = \lcm \{m_{f_{\V_i}}\}_i$
	\item Si $\V = \sum \langle v_i \rangle_f$, luego $m_f = \lcm \{m_{f_{v_i}}\}_i$
\end{enumerate}

Notemos que $0 = m_f(f)(v) = m_f(f_v)(v)$, luego $m_f(f_v) = 0$, entonces $m_{f_v} \mid m_f$. Escribamos entonces $m_f(x) = m_{f_v}(x) \cdot q_v(x)$.

Vamos a ver que si existe $p(x) : \forall w : m_{f_w} \mid p(x)$, luego demostremos que $m_f \mid p$.

Tenemos $p(f)(v) = (p'_v(f) \cdot m_{f_v}(f)) (v) = p'_v(f) (m_{f_v}(f)(v)) = p'_v(f)(0) = 0$, luego $p(f)(v) = 0$ para todo $v$, luego $m_f \mid p$.

Entonces $m_f = \lcm \{m_{f_v} : v \in \V\}$
\end{document}
