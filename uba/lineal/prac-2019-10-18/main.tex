\documentclass{article}
\usepackage{amssymb}
\usepackage{mathtools}

\everymath{\displaystyle}
\setlength{\parskip}{3mm}
\setlength{\parindent}{0mm}

\def\R{\mathbb{R}}
\def\K{\mathbb{K}}
\def\V{\mathbb{V}}
\def\W{\mathbb{W}}
\def\WW{\mathcal{W}}
\def\PP{\mathcal{P}}
\def\FF{\mathcal{F}}
\def\U{\mathbb{U}}
\def\C{\mathbb{C}}
\def\N{\mathbb{N}}
\def\Q{\mathbb{Q}}
\def\Z{\mathbb{Z}}

\def\BB{\mathcal{B}}
\def\AA{\mathcal{A}}
\def\EE{\mathcal{E}}
\def\CC{\mathcal{C}}
\def\OO{\mathcal{O}}
\def\DD{\mathcal{D}}

\def\e{\varepsilon}
\def\inn{\subseteq}

\def\S{\mathbb{S}}
\def\T{\mathbb{T}}
\def\l{\lambda}

\def\To{\Rightarrow}
\def\from{\leftarrow}
\def\From{\Leftarrow}

\DeclareMathOperator{\cl}{cl}
\DeclareMathOperator{\li}{li}
\DeclareMathOperator{\tl}{tl}
\DeclareMathOperator{\Id}{Id}
\DeclareMathOperator{\tr}{tr}
\DeclareMathOperator{\oo}{o}
\DeclareMathOperator{\spec}{spec}
\DeclareMathOperator{\mult}{mult}

\DeclareMathOperator{\iso}{iso}
\DeclareMathOperator{\mono}{mono}
\DeclareMathOperator{\epi}{epi}
\DeclareMathOperator{\adj}{adj}

\DeclareMathOperator{\Nu}{Nu}
\DeclareMathOperator{\Ima}{Im}
\DeclareMathOperator{\id}{id}
\DeclareMathOperator{\ze}{ze}

\DeclareMathOperator{\rg}{rg}

\DeclareMathOperator{\Hom}{Hom}
\DeclareMathOperator{\GL}{GL}
\DeclareMathOperator{\cont}{cont}
\DeclareMathOperator{\Hs}{H}

\DeclareMathOperator{\D}{D}
\DeclareMathOperator{\lcm}{lcm}

\DeclareMathOperator{\ev}{ev}
\DeclareMathOperator{\sg}{sg}

\date{}
\author{}

\begin{document}
\section*{Polinomio Característico}
Propuedades:
\begin{itemize}
	\item $S_\l = \{v : fv = \l v\}$
	\item $S_0 = \ker f$
	\item $A$ inversible sii $S_0 = \{0\}$
	\item $\spec A = \spec A^t$
	\item $\chi_A = \chi_{A^t}$
	\item $\l \in \spec A \To \l^k \in \spec A^k$
	\item $p (\spec A) = \spec (pA)$ para todo polinomio $p \in \K[x]$
	\item $S_\l$ y $S_\mu$ están en suma directa si $\l \neq \mu$.
\end{itemize}

\section*{Ejercicio 1}
Si $\{Ac, x\}$ es ld para todo $x$, luego demostrar que $A = cI$.

Notemos que implica que todo $x$ es autovector. Supongamos que hay $v_1$ y $v_2$ con autovalores $\l_1$, $\l_2$. Luego
\[
	\l_1 v_1 + \l_2 v_2 = A(v_1 + v_2) = \l(v_1 + v_2) = \l v_1 + \l v_2
\]
Luego si $\l_1 \neq \l_2$ son li, luego $\l_1 = \l = \l_2$. Contradicción.

Luego todos los vectores tienen el mismo autovalor y estamos.

\section*{Ejercicio 2}
Una matriz $A \in \C^{n \times n}$ es nilpotente sii su único autovalor es $0$.La ida es fácil, ya que si $\l$ es autovalor de $A$ luego $\l^k$ es de $A^k$.

Para la vuelta hacemos $A \sim T$ con $T$ triangular, y con los autovelores en la diagonal. Luego tenemos una triangular estricta, que sabemos que es nulpotente.

\section*{Ejercicio 3}
Dada $A \in \C^{n \times n}$, existe $T \sim A$, $T$ triangular superior. Usemos inducción.

Si tenemos la matriz $A$, expresémosla en una base $\BB$ donde el primer vector es un autovector de $A$. Luego la primer columna será todo $0$ menos en su primer entrada:

\[C_{E\BB}AC_{\BB E} = 
\begin{bmatrix}
	\l & u \\
	0 & A'
\end{bmatrix}\]

Por inducción, $A' = CT'C^{-1}$. Luego tomemos $D =
\begin{bmatrix}
	1 & 0 \\
	0 & C
\end{bmatrix}$
Es claro que $DC_{E\BB}AC_{\BB E}D^{-1}$ es triangular superior.

Notemos que en la matriz final $T$ tendremos que en la diagonal están los autovalores con multiplicidad algebraica.

Además, tendremos $\tr A = \sum \l_i$ y $\det A = \prod \l_i$

\section*{Ejercicio 4}
$M$ nilpotente sii $\tr M^i = 0$ para todo $i$.

\section*{Ejercicio 5}
Sean $f, g: \V \to \V$, con $f$ diagonalizable. $\V$ es $\C$-ev. finitamente generado. Luego $f \circ g = g \circ f \iff g(S_\l) \inn S_\l$ para todo $\l$.

Notemos que para la ida, tenemos que
\[v \in S_\l \To\]
\[fv = \l v\]
\[g(fv) = g(\l v)\]
\[f(gv) = \l gv\]
Luego $gv \in S_\l$

Para la vuelta, tomemos sobre una base de autovectores:
\[g(fv) = g(\l v) = \l (gv) = f(gv)\]
Luego estamos.
\end{document}
