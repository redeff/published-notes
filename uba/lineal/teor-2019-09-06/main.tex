\documentclass{article}
\usepackage{amssymb}
\usepackage{mathtools}

\everymath{\displaystyle}
\setlength{\parskip}{3mm}
\setlength{\parindent}{0mm}

\def\R{\mathbb{R}}
\def\K{\mathbb{K}}
\def\V{\mathbb{V}}
\def\W{\mathbb{W}}
\def\WW{\mathcal{W}}
\def\PP{\mathcal{P}}
\def\FF{\mathcal{F}}
\def\U{\mathbb{U}}
\def\C{\mathbb{C}}
\def\N{\mathbb{N}}
\def\Q{\mathbb{Q}}
\def\Z{\mathbb{Z}}

\def\BB{\mathcal{B}}
\def\AA{\mathcal{A}}
\def\EE{\mathcal{E}}
\def\CC{\mathcal{C}}
\def\OO{\mathcal{O}}
\def\DD{\mathcal{D}}

\def\e{\varepsilon}
\def\inn{\subseteq}

\def\S{\mathbb{S}}
\def\T{\mathbb{T}}
\def\l{\lambda}

\def\To{\Rightarrow}
\def\from{\leftarrow}
\def\From{\Leftarrow}

\DeclareMathOperator{\cl}{cl}
\DeclareMathOperator{\li}{li}
\DeclareMathOperator{\tl}{tl}
\DeclareMathOperator{\Id}{Id}
\DeclareMathOperator{\tr}{tr}
\DeclareMathOperator{\oo}{o}
\DeclareMathOperator{\spec}{spec}
\DeclareMathOperator{\mult}{mult}

\DeclareMathOperator{\iso}{iso}
\DeclareMathOperator{\mono}{mono}
\DeclareMathOperator{\epi}{epi}
\DeclareMathOperator{\adj}{adj}

\DeclareMathOperator{\Nu}{Nu}
\DeclareMathOperator{\Ima}{Im}
\DeclareMathOperator{\id}{id}
\DeclareMathOperator{\ze}{ze}

\DeclareMathOperator{\rg}{rg}

\DeclareMathOperator{\Hom}{Hom}
\DeclareMathOperator{\GL}{GL}
\DeclareMathOperator{\cont}{cont}
\DeclareMathOperator{\Hs}{H}

\DeclareMathOperator{\D}{D}
\DeclareMathOperator{\lcm}{lcm}

\DeclareMathOperator{\ev}{ev}
\DeclareMathOperator{\sg}{sg}

\date{}
\author{}

\begin{document}
\section{Proyectores}
Una transformación lineal $f: \V \to \V$ se duce \emph{proyector} sii $f^2 =
f$.

\subsection{$\oplus$ son proyectores}
Dados dos ev en suma directa $S\oplus T = \V$, 
luego para cada $v \in \V$ existe una única escritura $v = v_t + v_s$,
donde $v_s \in S, v_t \in T$.

Si tomamos \[f v =
s : s + t = v, s \in S, t \in T\].

Notemos que $f$ es lineal, es trivial.

Notemos además que:
\begin{itemize}
	\item $S \inn \Ima f$ ya que para $s \inn S$ tenemos $fs = s$
	\item $\Ima f \inn S$ ya que si $fv$ está, por definición en $S$.
	\item $\Nu f \inn T$ ya que si $t \in T$, $f t = 0$.
	\item $T \inn \Nu F$ por definición.
\end{itemize}

Es proyector ya que $f (s+t) = fs + ft = s$, y $f^2 (s+t)
= f(f(s+t)) = fs = s$.

Esta proyección se denomina \emph{proyección sobre el subespacio S},
y se denota $f_{S}$.

\subsection{Proyecciones son $\oplus$}
Dado un proyector $f : \V \to \V$, supongamos que $v \in \Nu f \cap \Ima f$.
Tenemos $f v = 0$, y a su vez $f x = v$ para algún $x$.

Notemos que $v = fx = f^2 x = f (fx) = f v = 0$, luego $v = 0$. Entonces la
intersección entre $\Nu f \cap \Ima f = \{0\}$.

Notemos además que $\Nu f + \Ima f = \V$, ya que podemos escribir $v = (v - fv)
+ fv$, y $f (v-fv) = fv - f(fv) = fv - fv = 0$, osea que $v - fv \in \Nu f$, y
además $fv$ claramente está en la imagen.

Luego tenemos que $\Nu f \oplus \Ima f = \V$.

\subsection{Identidad como suma}
Se tiene $\Id_{\V} = f_{S} + f_{T}$ para cualquiera $\V = S + T$.

Notemos además que si $f: \V \to \V$ proyector, se tiene $(\id - f) \circ (\id
- f) = \id^2 - \id \circ f - f \circ \id + f^2 = \id - f - f + f = \id - f$.

\subsection{Proyectores Triviales}
Notemos que la transformación nula $\ze = x\mapsto 0$ y $\id = x \mapsto x$ son
proyectores.

\section{Espacio Hom}
Dados $\V$, $\W$ $\K$-espacios vectoriales, se denomina:
\[\Hom_\K(\V, \W) = \{\tl f : \V \to \W\}\]
Al conjunto de las transformaciones lineales entre $\V$ y $\W$.

Se tiene que $\Hom_\K(\V, \W)$ es un $\K$-espacio vectorial.

\subsection{Dimensión del Espacio Hom}
Sea $\dim \V = n$ y $\dim \W = m$, luego vamos a ver que $\dim \Hom(\V, \W) = n
\cdot m$

\subsection{Isomorfismo $\Hom_\K(\V, \W) \sim \K^{\dim \W \times \dim \V}$}
Tomemos una base $B = v_{i < n}$ de $\V$ y sea $B' = w_{i < m}$ una base de $\W$.
Para cada $\tl f : \V \to \W$, asociémosle una matriz en $\K^{m \times n}$.

Para cada $i < n$, sean $\lambda_{ji}$ tal que $fv_i = \sum \lambda_ji w_j$.
Contruyámonos entonces la matriz que se nota \[ [f]_{BB'} = 
\begin{pmatrix}
	\lambda_{11} & \lambda_{12} & \dots & \lambda_{1n} \\
	\lambda_{21} & \lambda_{22} & \dots & \lambda_{2n} \\
	\vdots & \vdots & \ddots & \vdots \\
	\lambda_{m1} & \lambda_{m2} & \dots & \lambda_{mn} \\
\end{pmatrix}
\]

\subsection{Proyectores como Matrices}
Si tenemos $f = f_{S}$ con $\V = S + T$, y una base de $\V$ dada por $B_\V =
B_S \cap B_T$ con $n = |B_S|$, luego veamos que:
\[
	[f]_{BB} = 
	\begin{pmatrix}
		\Id_{n \times n} & 0 \\
		0 & 0 \\
	\end{pmatrix}
\]

\section{Derivadas}
Sea $V = \K[x]_{\leq n}$, y tomemos $fp = p'$, luego
\[
	[f]_{EE} = 
	\begin{pmatrix}
		0 & 1 & 0 & \dots & 0 \\
		0 & 0 & 2 & \dots & 0 \\
		0 & 0 & 0 & \dots & 0 \\
		\vdots & \vdots & \vdots & \ddots & \vdots \\
		0 & 0 & 0 & \dots & n-1 \\
		0 & 0 & 0 & \dots & 0 \\
	\end{pmatrix}
\]

\section{Prop}
Dados $\V$, $\W$ una $\tl f : \Hom(\V, \W)$ y bases $B$ de $\V$ y $B'$ de $\W$,
se tiene que \[[fv]_{B'} = [f]_{BB'} \times [v]_B\]

\section{Isomorfismo $\Hom_\K(\V, \W) \sim \K^{\dim \W \times \dim \V}$}
Definamos una $\phi : Hom(\V, \W) \to \K^{m \times n}, \phi = f \mapsto
[f]_{BB'}$.

Es fácil ver que dadas $\tl f, g : \V \to \W$, tenemos que $\phi (f + g) = \phi f +
\phi g$, es decir, $[f+g] = [f] + [g]$. Y que dado $\lambda$, se tiene
$\lambda \cdot [f] = [\lambda \cdot f]$

Para ver que es iso, definamos la inversa $\psi : \K^{m \times n} \to \Hom(\V,
\W), [\psi M v_i]_{B'} = M_{*i}$.

Notemos que $(\phi \circ \psi) M = [\psi M]_{BB'}= M$, y además $(\psi \circ
\phi) f = f$.

\subsection{Corolario}
$\dim \Hom(\V, \W) = \dim \K^{m \times n} = m \cdot n$

\section{Prop}
Sean $\V_{1,2,3}$ ev de dimensión $n_{1,2,3}$ y de bases $B_{1,2,3}$.
Sean $\tl f : \V_1 \to \V_2$ y $\tl g : \V_2 \to \V_3$.

Tenemos que $\tl f \circ g$ y que $[g \circ f]_{B_1B_3} = [g]_{B_2B_3} \times
[f]_{B_1B_2}$

\end{document}
