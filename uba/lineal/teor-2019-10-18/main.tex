\documentclass{article}
\usepackage{amssymb}
\usepackage{mathtools}

\everymath{\displaystyle}
\setlength{\parskip}{3mm}
\setlength{\parindent}{0mm}

\def\R{\mathbb{R}}
\def\K{\mathbb{K}}
\def\V{\mathbb{V}}
\def\W{\mathbb{W}}
\def\WW{\mathcal{W}}
\def\PP{\mathcal{P}}
\def\FF{\mathcal{F}}
\def\U{\mathbb{U}}
\def\C{\mathbb{C}}
\def\N{\mathbb{N}}
\def\Q{\mathbb{Q}}
\def\Z{\mathbb{Z}}

\def\BB{\mathcal{B}}
\def\AA{\mathcal{A}}
\def\EE{\mathcal{E}}
\def\CC{\mathcal{C}}
\def\OO{\mathcal{O}}
\def\DD{\mathcal{D}}

\def\e{\varepsilon}
\def\inn{\subseteq}

\def\S{\mathbb{S}}
\def\T{\mathbb{T}}
\def\l{\lambda}

\def\To{\Rightarrow}
\def\from{\leftarrow}
\def\From{\Leftarrow}

\DeclareMathOperator{\cl}{cl}
\DeclareMathOperator{\li}{li}
\DeclareMathOperator{\tl}{tl}
\DeclareMathOperator{\Id}{Id}
\DeclareMathOperator{\tr}{tr}
\DeclareMathOperator{\oo}{o}
\DeclareMathOperator{\spec}{spec}
\DeclareMathOperator{\mult}{mult}

\DeclareMathOperator{\iso}{iso}
\DeclareMathOperator{\mono}{mono}
\DeclareMathOperator{\epi}{epi}
\DeclareMathOperator{\adj}{adj}

\DeclareMathOperator{\Nu}{Nu}
\DeclareMathOperator{\Ima}{Im}
\DeclareMathOperator{\id}{id}
\DeclareMathOperator{\ze}{ze}

\DeclareMathOperator{\rg}{rg}

\DeclareMathOperator{\Hom}{Hom}
\DeclareMathOperator{\GL}{GL}
\DeclareMathOperator{\cont}{cont}
\DeclareMathOperator{\Hs}{H}

\DeclareMathOperator{\D}{D}
\DeclareMathOperator{\lcm}{lcm}

\DeclareMathOperator{\ev}{ev}
\DeclareMathOperator{\sg}{sg}

\date{}
\author{}

\begin{document}
\section*{Polinomio Característico}
Dada una matriz $A \in \K^{n \times n}$, se define $\chi_A(x) = \det (xI - A)$ el polinomio característico de la matriz $A$.

Se tiene que $\chi_A$ es un polinomio mónico de grado $n$, y con $\chi_A(0) = (-1)^n \det A$

\section*{Característico de Semejantes}
Si $A \sim B$, luego $\chi_A = \chi_B$
Esto es ya que:
\[
	\det (xI - B) =
\]
\[
	\det (xI - CAC^{-1}) =
\]
\[
	\det (C(xI)C^{-1} - CAC^{-1}) =
\]
\[
	\det C \cdot \det (xI - A) \cdot \det C^{-1}=
\]
\[
	\det (xI - A)
\]
Luego está.

\section*{Característico en $\V \to \V$}
Por el teorema anterior, está bien definido $\chi_f$ donde $f : \V \to \V$, ya que es independiente de la base.

\section*{Autovalores}
Podemos ahora decir que $\lambda \in \K$ es un autovalor de $f : \V \to \V$ sii $\chi_f(\lambda) = 0$, y se dice que $\spec f = \{\lambda : \chi_f \lambda = 0 \}$

Esto implica que toda $f : \V \to \V$ tiene $\dim \V$ autovalores contados con multiplicidad, con $\V$ un $\C$-ev.

\section*{Multiplicidad de Autovalores}
La dimensión de $S\l = \{v : fv = \l v\}$ es la \emph{multiplicidad geométrica} de $\l$, pero el máximo $k$ que $(x - \l)^k | \chi_f$ es la \emph{multiplicidad algebraica} de $\l$.

\subsection*{Lema $\dim S_\l \leq\mult_\l$}
Tenemos $m = \mult_\l \chi_f$. Luego $\chi_f = (x - \l)^mq(x)$, con $q(\l) \neq 0$.

Supongamos que $r = \dim S_\l \geq m+1$. Tomamos $\{v_i\}_{i \leq r}$ base de $S_\l$, y extendámosla a una base $\BB$ de $\V$.

Tenemos que:
\[
	[f]_\BB = 
	\begin{bmatrix}
		\l I_r & * \\
		0 & B
	\end{bmatrix}
\]

Tenemos que \[\chi_f =
\det \left(
	xI - \begin{bmatrix}
		\l I_r & * \\
		0 & B
\end{bmatrix} \right) = \]
\[
\det (xI - \l I) \cdot \det(xI - B)
\]
\[
	\det ((x - \l) I) \cdot \det(xI - B)
\]
\[
	(x - \l)^r \cdot \det(xI - B)
\]
Pero luego la multiplicidad algebraica es al menos $r$. Contradicción.

\section*{Diagonalizabilidad}
Tenemos que $f$ es \emph{diagonalizable} si $[f]_\BB$ es diagonal en alguna base

Alternativamente, $f$ es diagonalizable sii la multiplicidad geométrica y la algebraica coinciden para todo autovalor, ya que podemos elegirnos $\mult_l$ vectores an cada autoespacio que formarán una base.

La vuelta es contriuirse la matriz diagonal y hacer la determinante por Leibniz, que da la productoria.

\section*{Ideales en $\K[x]$}
Un \emph{ideal} de $\K[x]$ es un subespacio $I \inn \K[x]$ tal que $fg \in I$ para todo $f \in I$ y $g \in \K[x]$.

Notemos que $\{0\}$ es un ideal, y que además $\{f \cdot g : f \in \K[x]\}$ es un ideal para cualquier polinomio $g$. Este es el ideal generado por $g$, es decir $(g)$.

Se dice también que $(g_1, \dots, g_n) = \left\{\sum f_i \cdot g_i : f_i \in \K[x]\right\}$

\subsection*{Caracterización de los Ideales}
Tomemos un ideal no nulo $I \in \K[x]$, luego $I = (g)$ para algún $g$.

Demo: tomemos algún $g \in I$ mónico y de grado mínimo. Notemos que $(g) = I$, ya que para todo $f \in I$, tenemos que $f - gq = r$, con $r = 0$ o $\deg r < \deg g$. Como es ideal, sabemos que $r \in I$, luego $\deg r \geq \deg g$, entonces necesariamente $r = 0$. Pero entonces $f$ es múltiplo de $g$. Estamos.

\section*{Polinomios de matrices}
Dado un poly en $p \in \K[x]$, y una $A\in \K^{n \times n}$, luego podemos hacer $p(A) = \sum a_iA^i$, y cumple las propiedades:
\begin{itemize}
	\item $(p+q)A = pq + qA$
	\item $(pq)A = (pA)(qA)$
\end{itemize}

\subsection*{Polinomio que anula a la matriz}
Vamos a ver que existe un polinomio $p \in \K[x]$ tal que $pA = 0$. Esto es verdad ya que $I, A, A^2, \dots$ viven todos en $\K^{n \times n}$, luego existen algún subconjunto de las potencias de $A$ que sean ld, luego hay una combionación lineal no nula que da $0$, luego ya estamos.

\section*{Polinomio Minimal}
Dada una $A \in \K^{n \times n}$, tomemos $I = \{p \in \K[x] : p(A) = 0\}$. Es claramente un ideal, y es no nulo por el lema anterior. Luego $I = (m_A)$ para un único polinomio mónico $m_A$. $m_A$ se llama el \emph{polinomio munimal} de $A$.
\end{document}
