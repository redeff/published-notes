\documentclass{article}
\usepackage{amssymb}
\usepackage{mathtools}

\everymath{\displaystyle}
\setlength{\parskip}{3mm}
\setlength{\parindent}{0mm}

\def\R{\mathbb{R}}
\def\K{\mathbb{K}}
\def\V{\mathbb{V}}
\def\W{\mathbb{W}}
\def\WW{\mathcal{W}}
\def\PP{\mathcal{P}}
\def\FF{\mathcal{F}}
\def\U{\mathbb{U}}
\def\C{\mathbb{C}}
\def\N{\mathbb{N}}
\def\Q{\mathbb{Q}}
\def\Z{\mathbb{Z}}

\def\BB{\mathcal{B}}
\def\AA{\mathcal{A}}
\def\EE{\mathcal{E}}
\def\CC{\mathcal{C}}
\def\OO{\mathcal{O}}
\def\DD{\mathcal{D}}

\def\e{\varepsilon}
\def\inn{\subseteq}

\def\S{\mathbb{S}}
\def\T{\mathbb{T}}
\def\l{\lambda}

\def\To{\Rightarrow}
\def\from{\leftarrow}
\def\From{\Leftarrow}

\DeclareMathOperator{\cl}{cl}
\DeclareMathOperator{\li}{li}
\DeclareMathOperator{\tl}{tl}
\DeclareMathOperator{\Id}{Id}
\DeclareMathOperator{\tr}{tr}
\DeclareMathOperator{\oo}{o}
\DeclareMathOperator{\spec}{spec}
\DeclareMathOperator{\mult}{mult}

\DeclareMathOperator{\iso}{iso}
\DeclareMathOperator{\mono}{mono}
\DeclareMathOperator{\epi}{epi}
\DeclareMathOperator{\adj}{adj}

\DeclareMathOperator{\Nu}{Nu}
\DeclareMathOperator{\Ima}{Im}
\DeclareMathOperator{\id}{id}
\DeclareMathOperator{\ze}{ze}

\DeclareMathOperator{\rg}{rg}

\DeclareMathOperator{\Hom}{Hom}
\DeclareMathOperator{\GL}{GL}
\DeclareMathOperator{\cont}{cont}
\DeclareMathOperator{\Hs}{H}

\DeclareMathOperator{\D}{D}
\DeclareMathOperator{\lcm}{lcm}

\DeclareMathOperator{\ev}{ev}
\DeclareMathOperator{\sg}{sg}

\date{}
\author{}

\begin{document}
\section*{Prop}
si tenemos $f : \V \to \W$, luego:
\begin{itemize}
	\item $\ker f^t = (\Ima f)^0$
	\item $\Ima f^t \inn (\ker f)^0$, la igualdad vale con dimensión finita.
	\item $f \mono \iff f^t \epi$
	\item $\rg f = \rg f^t$
\end{itemize}

\section*{Matriz de la Transpuesta}
Si tenemos $f : \V \to \W$, y $v, w$ bases de $\V, \W$, entonces:
\[
	(f^t)_{w^*v^*} = (f_{vw})^t
\]

Demo: si tomamos $A = f_{vw}$, luego tenemos:
\[
	fv_i = \sum a_{ji}w_j
\]

Notemos además que
\[
	(f^tw^*_j)_{v^*i} = f^tw^*_jv_i = w^*_j(fv_i) = w^*_j \sum a_{ji}w_j = a_{ji}
\]
Entonces ya estamos.

\section*{Determinantes}
Se define como una función $\det : \K^{n \times n} \to \K$ que toma $n$ vectores en $\K^n$, y devuelve el volumen signado del hiperparalepípedo formado por esos vectores. Por propuedades geométricas llegamos a que:
\begin{itemize}
	\item $\det (A_1 \dots A_i + v \dots A_n) = \det (A_i\dots) + \det (A_1 \dots v \dots)$ (se dice $f$ es multilineal alternado por columna)
	\item $\det (A_1 \dots \lambda A_i\dots) = \lambda \det A$ ()
	\item itercambiar columnas cambia de signo
	\item Si son li, entonces $\det A = 0$
	\item $\det A \cdot B = \det A \cdot \det B$
\end{itemize}

\subsection*{Notación}
Si tenemos una sucesión $A_{i < n}$, luego $k^iA$ es la sucesión $A$ pero con la $i$-ésima posición cambiada por $k$

Sea $A(i|j)$ la matriz que resulta de sacarle la fila $i$ y la columna $j$ a $A$.

\subsection*{Función multilineal alternada}
Una función $f : \K^{n \times n} \to \K$ se dice \emph{multilineal alternada} si:
\begin{itemize}
	\item Si se finan todas las columnas de una matriz $M$ excepto la $i$-ésima, entonces la función es lineal en la $i$-ésima columna. Es decir, $v \mapsto f (v^iA )$ es lineal.
	\item Si hay columnas repetidas entonces $f$ vale $0$. Es decir, $f (v^iv^jA) = 0$ para $i \neq j$
\end{itemize}

\subsubsection*{Propiedades de las Multilineales Alternadas}
\begin{itemize}
	\item Si una fila es $0$, entonces vale $fA = 0$. Demo trivial por linearidad.
	\item Intercambiar filas cambia el signo del determinante. Demo: si tomamos la igualdad $f(A_1 \mid \dots \mid A_i + A_j \mid \dots \mid A_i + A_j \mid A_n)$ y expandimos en los cuatro sumandos por linearidad, nos quedan dos cosas que son $0$ y las otras dos tienen que ser opuestas y estamos.

		\[f ((v+w)^i(v+w)^jA) = 0\]
	\[f (v^iv^jA) + f (v^iw^jA) + f (w^iv^jA) + f(w^iw^jA) = 0\]
	\[f (v^iw^jA) + f (w^iv^jA) = 0\]

	Entonces estamos.

	\item $fA = f(A_1 \mid \dots \mid A_i + \alpha A_j \mid \dots \mid A_n)$
	\item Si alguna columna es combinación lineal de las otras, entonces $fA = 0$.
\end{itemize}

\subsection*{La única multilineal alternada es $\det$}
Para cualquier $\alpha \in \K$, y $n \in \N$, existe una única multilineal alternada $f : \K^{n \times n} \to K$ tal que $f I_n = \alpha$. Demo:

\subsubsection*{Existencia}
Para $n=1$, luego existe $f[a] = \alpha a$.

Ahora por inducción, asumimos que existe para $g : \K^{n \times n} \to \K$ multilineal.

Definamos $f A = \sum_{i = 0}^n (-1)^i \cdot a_{0i}\cdot g \; A(0|i)$. Vamos a ver que cumple las condiciones de multilinearidad alternada.

Para ver que es lineal en la $k$-ésima fila, alcanza con ver que cada término de la suma es lineal. El $k$-ésimo término es de la forma $a_{0k} \cdot c$, con $c$ constante, por lo que claramente es lineal en la columna $k$.

En un térmono que no es el $k$-ésimo, nos queda una constante por algo que depende de $g \; A(0|i)$, que por inducción es lineal en la $k$-ésima columna sin el primer elemento.

Además $f (v^iv^jA) = \sum g \; (v^iv^jA)(0|k) + g (v^iv^jA)(0|i) + g(v^iv^jA)(0|j)$, y ya estamos.
\end{document}
