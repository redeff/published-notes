\documentclass{article}
\usepackage{amssymb}
\usepackage{mathtools}

\everymath{\displaystyle}
\setlength{\parskip}{3mm}
\setlength{\parindent}{0mm}

\def\R{\mathbb{R}}
\def\K{\mathbb{K}}
\def\V{\mathbb{V}}
\def\W{\mathbb{W}}
\def\WW{\mathcal{W}}
\def\PP{\mathcal{P}}
\def\FF{\mathcal{F}}
\def\U{\mathbb{U}}
\def\C{\mathbb{C}}
\def\N{\mathbb{N}}
\def\Q{\mathbb{Q}}
\def\Z{\mathbb{Z}}

\def\BB{\mathcal{B}}
\def\AA{\mathcal{A}}
\def\EE{\mathcal{E}}
\def\CC{\mathcal{C}}
\def\OO{\mathcal{O}}
\def\DD{\mathcal{D}}

\def\e{\varepsilon}
\def\inn{\subseteq}

\def\S{\mathbb{S}}
\def\T{\mathbb{T}}
\def\l{\lambda}

\def\To{\Rightarrow}
\def\from{\leftarrow}
\def\From{\Leftarrow}

\DeclareMathOperator{\cl}{cl}
\DeclareMathOperator{\li}{li}
\DeclareMathOperator{\tl}{tl}
\DeclareMathOperator{\Id}{Id}
\DeclareMathOperator{\tr}{tr}
\DeclareMathOperator{\oo}{o}
\DeclareMathOperator{\spec}{spec}
\DeclareMathOperator{\mult}{mult}

\DeclareMathOperator{\iso}{iso}
\DeclareMathOperator{\mono}{mono}
\DeclareMathOperator{\epi}{epi}
\DeclareMathOperator{\adj}{adj}

\DeclareMathOperator{\Nu}{Nu}
\DeclareMathOperator{\Ima}{Im}
\DeclareMathOperator{\id}{id}
\DeclareMathOperator{\ze}{ze}

\DeclareMathOperator{\rg}{rg}

\DeclareMathOperator{\Hom}{Hom}
\DeclareMathOperator{\GL}{GL}
\DeclareMathOperator{\cont}{cont}
\DeclareMathOperator{\Hs}{H}

\DeclareMathOperator{\D}{D}
\DeclareMathOperator{\lcm}{lcm}

\DeclareMathOperator{\ev}{ev}
\DeclareMathOperator{\sg}{sg}

\date{}
\author{}

\title {Práctica de Lineal}
\begin{document}
	\maketitle
	\section{Ejercicio 1}
	$A - \overline{A^T}$.

	Tomamos $T = \{M \in \C^{2 \times 2} \mid M = - M^\dag\}$,
	y $S = \{M \in \C^{2 \times 2} \mid M = M^\dag\}$

	Notemos que si $(A + A^\dag)^\dag = A^\dag + A$, luego $A + A^\dag \in S$,
	similarmente $(A - A^\dag)^\dag = A^\dag - A$, luego $A - A^\dag \in T$.

	Entonces $A = \frac{A + A^\dag}{2} + \frac{A - A^\dag}{2}$ es una escritura.
	\section{Ejercicio 2}
	$S = \{A \in \R^{2 \times 2} : a_{11} + ka_{12} - ka_{21} = (2k+1)a_{11}
	+ 2ka_{12} - ka_{22} = 0\}$

	$T = \left\langle
		\begin{bmatrix}
			1 & k-2 \\
			0 & 1
		\end{bmatrix}
		\begin{bmatrix}
			0 & k \\
			1+k-k^2 & 2
		\end{bmatrix}
		\begin{bmatrix}
			1 & k-2 \\
			k^2 - k & k
		\end{bmatrix}
	\right\rangle$

	Tenemos $1 + k(k-2) - 0 = 0 = k^2 - 2k + 1 = (k-1)^2$, luego solo anda $k = -1$.
	Tenemos:
	$S = \{A \in \R^{2 \times 2} : a_{11} + a_{12} - a_{21} = 3a_{11}
	+ 2a_{12} - a_{22} = 0\}$

	$T = \left\langle
		\begin{bmatrix}
			1 & -1 \\
			0 & 1
		\end{bmatrix}
		\begin{bmatrix}
			0 & 1 \\
			1 & 2
		\end{bmatrix}
		\begin{bmatrix}
			1 & -1 \\
			0 & 1
		\end{bmatrix}
	\right\rangle$

	$T = \left\langle
		\begin{bmatrix}
			1 & -1 \\
			0 & 1
		\end{bmatrix}
		\begin{bmatrix}
			0 & 1 \\
			1 & 2
		\end{bmatrix}
	\right\rangle$

	Además las ecuaciones son li osea que las dimensiones dan. Y luego alcanza con ver que
	las matrices cumplen.

	\section{Ejercicio 3}
	Demostrar $min \{k : A^k = 0\} \leq n$.
	Notemos $\Im A^i$ para $i = 0\dots n$. Sabemos $\Im A^i \inn \Im A^{i-1}$, y además
	$\Im A^i = A \times \Im A^{i-1}$, luego las dimensiones tienen
	que decrecer estrictas o se estancan.

	\section{Matrices de Markov}
	Una cadena de Markov es una sucesión de eventos donde la probabilidad de uno depende
	sólo del anterior.

	Si hay $n$ estados posibles, las probabilidades se pueden codificar como una matriz
	dada por $M_{ij} = p_{i \to j}$.

	Tenemos $M_{ij} \in [0, 1]$, y además $\sum_j M_{ij} = 1$.

	Las matrices que cumplen eso se llaman matrices de Markov.

	\subsection{Prop}
	Si $M, N \in R^{n \times n}$ son Markov, luego $M \times N$ también es Markov.
	\[(M \times N)_{ij} = \sum_k M_{ik} \cdot N_{kj}\]
	\[\sum_j (M \times N)_{ij} = \sum_j \sum_k M_{ik} \cdot N_{kj}\]
	\[\sum_j (M \times N)_{ij} = \sum_k M_{ik} \sum_j \cdot N_{kj}\]
	\[\sum_j (M \times N)_{ij} = \sum_k M_{ik} \cdot 1\]
	\[\sum_j (M \times N)_{ij} = \sum_k M_{ik}\]
	\[\sum_j (M \times N)_{ij} = 1\]

	Demo alternativa:

	La condición de Makrov implica $M \times 
	\begin{bmatrix}
		1 \\
		\vdots \\
		1
	\end{bmatrix} =
	\begin{bmatrix}
		1 \\
		\vdots \\
		1
	\end{bmatrix}
	$. La composición trivialmente mantiene esta condición.

	\subsection{Ejemplo}
	\[
	\begin{pmatrix}
		\frac{1}{4} & \frac{1}{4} & \frac{1}{4} & \frac{1}{4} \\
		\frac{1}{2} & \frac{1}{2} & 0 & 0 \\
		\frac{1}{2} & 0 & \frac{1}{2} & 0 \\
		\frac{1}{2} & 0 & 0 & \frac{1}{2}
	\end{pmatrix}
	\]
\end{document}
