\documentclass{article}
\usepackage{amssymb}
\usepackage{mathtools}

\everymath{\displaystyle}
\setlength{\parskip}{3mm}
\setlength{\parindent}{0mm}

\def\R{\mathbb{R}}
\def\K{\mathbb{K}}
\def\V{\mathbb{V}}
\def\W{\mathbb{W}}
\def\WW{\mathcal{W}}
\def\PP{\mathcal{P}}
\def\FF{\mathcal{F}}
\def\U{\mathbb{U}}
\def\C{\mathbb{C}}
\def\N{\mathbb{N}}
\def\Q{\mathbb{Q}}
\def\Z{\mathbb{Z}}

\def\BB{\mathcal{B}}
\def\AA{\mathcal{A}}
\def\EE{\mathcal{E}}
\def\CC{\mathcal{C}}
\def\OO{\mathcal{O}}
\def\DD{\mathcal{D}}

\def\e{\varepsilon}
\def\inn{\subseteq}

\def\S{\mathbb{S}}
\def\T{\mathbb{T}}
\def\l{\lambda}

\def\To{\Rightarrow}
\def\from{\leftarrow}
\def\From{\Leftarrow}

\DeclareMathOperator{\cl}{cl}
\DeclareMathOperator{\li}{li}
\DeclareMathOperator{\tl}{tl}
\DeclareMathOperator{\Id}{Id}
\DeclareMathOperator{\tr}{tr}
\DeclareMathOperator{\oo}{o}
\DeclareMathOperator{\spec}{spec}
\DeclareMathOperator{\mult}{mult}

\DeclareMathOperator{\iso}{iso}
\DeclareMathOperator{\mono}{mono}
\DeclareMathOperator{\epi}{epi}
\DeclareMathOperator{\adj}{adj}

\DeclareMathOperator{\Nu}{Nu}
\DeclareMathOperator{\Ima}{Im}
\DeclareMathOperator{\id}{id}
\DeclareMathOperator{\ze}{ze}

\DeclareMathOperator{\rg}{rg}

\DeclareMathOperator{\Hom}{Hom}
\DeclareMathOperator{\GL}{GL}
\DeclareMathOperator{\cont}{cont}
\DeclareMathOperator{\Hs}{H}

\DeclareMathOperator{\D}{D}
\DeclareMathOperator{\lcm}{lcm}

\DeclareMathOperator{\ev}{ev}
\DeclareMathOperator{\sg}{sg}

\date{}
\author{}

\begin{document}
\section*{Minimal en autovalores}
Sea $p \in k[x]$, $p(A) = 0$. Luego $p(\l) = 0$ para todo autovalor.

Sea $v \in \V$ tal que $Av = \l v$, luego
\[0 = p(A) \cdot v = \sum a_iA^iv = \sum a_i\l^iv = v \cdot p(\l)\]
\[0 = v \cdot p(\l)\]
\[0 = p(\l)\]

\section*{Minimal de un vector}
Dado una función $f : \V \to \V$, y un $v \in \V$, se dice que $m_v$ es el menor polinomio mónico tal que $m_v(f)(v) = 0$

Tenemos que $m_f = \lcm \{m_{\BB_i}(x)\}$, con $\BB$ una base.

\section*{Ejemplo 1}
Si tenemos
\[
	A =
	\begin{bmatrix}
		1 & 0 & 0 \\
		1 & 1 & 0 \\
		0 & 0 & 2 \\
	\end{bmatrix}
\]
Calcular el minimal.

Calcularmos el caraterísticos de cada canónico.
\[\{e_1, Ae_1, A^2e_1\}=\]
\[\left\{
		\begin{bmatrix}
			1 \\ 0 \\ 0
		\end{bmatrix},
		\begin{bmatrix}
			1 \\ 1 \\ 0
		\end{bmatrix},
		\begin{bmatrix}
			1 \\ 2 \\ 0
		\end{bmatrix}
\right\}\]

\section*{Criterio de Diagonalización}
Una función $f : \V \to \V$ es diagonalizable sii $m_f(x)$ se factoriza linealmente como producto de factores distintos.

En particular, si $f$ tiene $n$ autovalores distintos, es diagonalizable.

\section*{Hamilton Cayley}
Si tenemos una matriz cuadrada $A \in \K^{n \times n}$, entonces $\chi_A(A) = 0$.

Demo:
Para diagonalizables es trivial.
Para cualquier otra por continuidad.

\section*{Potencias de una matriz}
Si tenemos una matriz inversible $A$, luego notemos que si sabemos $\chi_A$, entonces podemos despejar $A^{-1}$

Además, si escrivimos $x^n = q_n(x) \cdot \chi_A(x) + r_n(x)$. Luego, por cada autovalor $\l$ de $A$, evaluamos este polinomio en $\l$, y si es raíz múltiple, evaluamos $\l$ es la derivada primera, segunda, etc de $r(x)$, y así despejamos los coeficientes de $r_n(x)$.

Así obtenemos $r_n$, entonces tenemos que $A^n = r_n(A)$.
\end{document}
