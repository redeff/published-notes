\documentclass{article}
\usepackage{amssymb}
\usepackage{mathtools}

\everymath{\displaystyle}
\setlength{\parskip}{3mm}
\setlength{\parindent}{0mm}

\def\R{\mathbb{R}}
\def\K{\mathbb{K}}
\def\V{\mathbb{V}}
\def\W{\mathbb{W}}
\def\WW{\mathcal{W}}
\def\PP{\mathcal{P}}
\def\FF{\mathcal{F}}
\def\U{\mathbb{U}}
\def\C{\mathbb{C}}
\def\N{\mathbb{N}}
\def\Q{\mathbb{Q}}
\def\Z{\mathbb{Z}}

\def\BB{\mathcal{B}}
\def\AA{\mathcal{A}}
\def\EE{\mathcal{E}}
\def\CC{\mathcal{C}}
\def\OO{\mathcal{O}}
\def\DD{\mathcal{D}}

\def\e{\varepsilon}
\def\inn{\subseteq}

\def\S{\mathbb{S}}
\def\T{\mathbb{T}}
\def\l{\lambda}

\def\To{\Rightarrow}
\def\from{\leftarrow}
\def\From{\Leftarrow}

\DeclareMathOperator{\cl}{cl}
\DeclareMathOperator{\li}{li}
\DeclareMathOperator{\tl}{tl}
\DeclareMathOperator{\Id}{Id}
\DeclareMathOperator{\tr}{tr}
\DeclareMathOperator{\oo}{o}
\DeclareMathOperator{\spec}{spec}
\DeclareMathOperator{\mult}{mult}

\DeclareMathOperator{\iso}{iso}
\DeclareMathOperator{\mono}{mono}
\DeclareMathOperator{\epi}{epi}
\DeclareMathOperator{\adj}{adj}

\DeclareMathOperator{\Nu}{Nu}
\DeclareMathOperator{\Ima}{Im}
\DeclareMathOperator{\id}{id}
\DeclareMathOperator{\ze}{ze}

\DeclareMathOperator{\rg}{rg}

\DeclareMathOperator{\Hom}{Hom}
\DeclareMathOperator{\GL}{GL}
\DeclareMathOperator{\cont}{cont}
\DeclareMathOperator{\Hs}{H}

\DeclareMathOperator{\D}{D}
\DeclareMathOperator{\lcm}{lcm}

\DeclareMathOperator{\ev}{ev}
\DeclareMathOperator{\sg}{sg}

\date{}
\author{}

\begin{document}
\section*{Ejercicio 1}
Consideremos $D : \R_n[x] \to \R_n[x]$ la derivación. Hallar una base de $\ker D^t$.

Notemos que $\ker D^t = (\Ima D)^0$. Es decir, son las funciones que se anulan en la $\Ima D$, que son los polinomios con el coeficiente de $x^n$ nulo.

Una base, por lo tanto, está formada por la función
\[
	\phi x^i = 
	\begin{cases}
		1 & \text{Si $i = n$} \\
	0 & \text{Si no}
	\end{cases}
\]
Y esto genera ya que por teo de la dimensión $\dim \ker D^t = 1$ y $\phi \neq 0$. La función $\phi$ tablién se puede expresar como $\phi P = D^nP (0)$

\subsection*{Nota}
La base dual de los polinomios se puede escribir:
\[
	E^*_i = \frac{1}{i!} \cdot \ev_0 \circ D^i
\]

\section*{Determinantes}
\begin{itemize}
	\item $\det ((\lambda v)^iA) = \lambda \det v^iA$, que implica
	\item $\det (\lambda A) = \lambda^n \det A$
	\item $\det ((v+w)^iA) = \det v^iA + \det w^iA$
	\item $\det (v^iw^jA) = - \det (v^jw^iA)$
	\item $\det A = \det A^t$
\end{itemize}

\section*{Ejercicio 2}
Tenemos $A = (C_1 \mid C_2 \mid C_3)$. Tenemos que $\det (5C_1 \mid 2C_2+C_3 \mid C_1 - C_3) = 20$. Calcular $\det A$.

Notemos que:
\[\det (5C_1 \mid 2C_2+C_3 \mid C_1 - C_3) = 20\]
\[\det (5C_1 \mid 2C_2 \mid C_1 - C_3) +\det (5C_1 \mid C_3 \mid C_1 - C_3) = 20\]
\[\det (5C_1 \mid 2C_2 \mid C_1 - C_3) = 20\]
\[\det (5C_1 \mid 2C_2 \mid - C_3) = 20\]
\[-10\det (C_1 \mid C_2 \mid C_3) = 20\]
\[\det A = -2\]

\section*{Ejercicio 6 de la Práctica Det}
Dada la matriz:
\[
	V(\alpha_1, \dots, \alpha_n) = 
	\begin{bmatrix}
		1 & 1 & \dots & 1 \\
		\alpha_1 & \alpha_2 & \dots & \alpha_n \\
		\vdots & \vdots & \ddots & \vdots \\
		\alpha_1^n & \alpha_2^n & \dots & \alpha_n^n
	\end{bmatrix}
\]
Demostrar que $\det V(\alpha_i) = \prod_{i < j} (\alpha_j - \alpha_i)$.

Notemos que la función $P(x) = \det V (\alpha_i, \dots,  x)$ es un polinomio, con raíces en $\alpha_i$, ya que ahí tomamos la determinante de una matriz con funciones repetidas. Luego $P(x) = \lambda \cdot \prod_{i < n} (x - \alpha_i)$, donde $\lambda$ es el coeficiente principal, que si hacemos la expansion por columnas vemos que es $\lambda = \det V(\alpha_1, \dots, \alpha_{n-1})$, luego evaluándolo en $x = \alpha_n$, y con hpótesis inductiva, tenemos:

\[P(\alpha_n) = \prod_{i < j < n} (\alpha_j - \alpha_i) \cdot \prod (\alpha_n - \alpha_i) = \prod_{i < j} (\alpha_j - \alpha_i)\]
\end{document}
