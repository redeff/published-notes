\documentclass{article}
\usepackage{amssymb}
\usepackage{mathtools}

\everymath{\displaystyle}
\setlength{\parskip}{3mm}
\setlength{\parindent}{0mm}

\def\R{\mathbb{R}}
\def\K{\mathbb{K}}
\def\V{\mathbb{V}}
\def\W{\mathbb{W}}
\def\WW{\mathcal{W}}
\def\PP{\mathcal{P}}
\def\FF{\mathcal{F}}
\def\U{\mathbb{U}}
\def\C{\mathbb{C}}
\def\N{\mathbb{N}}
\def\Q{\mathbb{Q}}
\def\Z{\mathbb{Z}}

\def\BB{\mathcal{B}}
\def\AA{\mathcal{A}}
\def\EE{\mathcal{E}}
\def\CC{\mathcal{C}}
\def\OO{\mathcal{O}}
\def\DD{\mathcal{D}}

\def\e{\varepsilon}
\def\inn{\subseteq}

\def\S{\mathbb{S}}
\def\T{\mathbb{T}}
\def\l{\lambda}

\def\To{\Rightarrow}
\def\from{\leftarrow}
\def\From{\Leftarrow}

\DeclareMathOperator{\cl}{cl}
\DeclareMathOperator{\li}{li}
\DeclareMathOperator{\tl}{tl}
\DeclareMathOperator{\Id}{Id}
\DeclareMathOperator{\tr}{tr}
\DeclareMathOperator{\oo}{o}
\DeclareMathOperator{\spec}{spec}
\DeclareMathOperator{\mult}{mult}

\DeclareMathOperator{\iso}{iso}
\DeclareMathOperator{\mono}{mono}
\DeclareMathOperator{\epi}{epi}
\DeclareMathOperator{\adj}{adj}

\DeclareMathOperator{\Nu}{Nu}
\DeclareMathOperator{\Ima}{Im}
\DeclareMathOperator{\id}{id}
\DeclareMathOperator{\ze}{ze}

\DeclareMathOperator{\rg}{rg}

\DeclareMathOperator{\Hom}{Hom}
\DeclareMathOperator{\GL}{GL}
\DeclareMathOperator{\cont}{cont}
\DeclareMathOperator{\Hs}{H}

\DeclareMathOperator{\D}{D}
\DeclareMathOperator{\lcm}{lcm}

\DeclareMathOperator{\ev}{ev}
\DeclareMathOperator{\sg}{sg}

\date{}
\author{}

\begin{document}
\section*{Complemento Ortogonal}
Dado $S \inn \V$, decimos $S^\perp = \{ v \in \V : v \perp s \forall s \in \S \}$.

Cuando es finito $\V$, tenemos que $S \oplus S^\perp = \V$

Para calcular $\S^\perp$, tomamos una base de $S$ extendida a una de $\V$, y aplicamos el algoritmo para hacerla ortogonal.

\section*{Proyectores}
Se define $p_S : \V \to \V$ al proyector que cumple:
\begin{itemize}
    \item $\Ima p = S$
    \item $\ker p = S^\perp$
\end{itemize}

\section*{Descomposición en base Ortonormal}
Si tenemos una base ortonormal $\BB$ de $\V$, tenemos que:
\[
    v = \sum \langle v, \BB_i \rangle \cdot \BB_i
\]

\section*{Ejercicio 1}
Dado $\V = \R^{3 \times 3}$, $\langle A, B\rangle = \tr(AB^t)$:
\begin{enumerate}
    \item Calcular $S^\perp$, donde $S = \{A \in \R^{3 \times 3} : A_{ij} = 0 \forall i \neq 1\}$
    \item Dada $A = 
        \begin{bmatrix}
            1 & 2 & 3 \\
            3 & 1 & 2 \\
            2 & 3 & 1 \\
        \end{bmatrix}$, calcular el $B \in S^\perp$ más cercano a $A$.
\end{enumerate}

Notemos que, si $A \in S$, luego
\[\tr (AB^t) = 0\]
\[\sum_i (AB^t)_{ii} = 0\]
\[\sum_i \sum_j A_{ij} B^t_{ji} = 0\]
\[\sum_i \sum_j A_{ij} B_{ij} = 0\]
\[\sum_j A_{1j} B_{1j} = 0\]
Luego $S^\perp = \{B : \forall j : B_{1j} = 0\}$

Notemos que demás la canónica $E_{ij}$ es BON, entonces:
\[p_S(A) = \sum \langle A, E_{1j} \rangle E_{1j} =\]
\[p_S(A) = \sum A_{1j} E_{1j} =\]
\[
    \begin{bmatrix}
        1 & 2 & 3 \\
        0 & 0 & 0 \\
        0 & 0 & 0 \\
    \end{bmatrix}
\]

\section*{Ejercicio 2}
Probar que existe pi en $\R^{3 \times 3}$ tq:
\begin{itemize}
    \item $S^\perp$ = $\{(x \; y \; z) : x+y-z = 0\}^\perp = \langle(0 \; 0 \; 1)\rangle$
    \item $\langle (2 \; 2 \; 17), (1 \; 0 \; 1) \rangle = 0$
\end{itemize}
Calcular $\langle (1 \; 1 \; 2) \rangle ^ \perp$

\section*{Caracterización de los Productos Internos}
Si $\BB$ es una base, y $A$ una matriz simétrica y definida positiva, luego:
\[
    \Phi(x,y) = (x)_\BB A((y)_\BB)^t
\]

Vamos a querer construirnos una base ortonormal para trabajar, que queremos que tenga subbases de $S$ y de $S^\perp$. Pongamos entonces a $(0 \; 0 \; 1) \in S^\perp$ en la base, y además pongamos a $(1 \; 0 \; 1)$ en la base, y queremos unomás, ortogonal a los otros dos. Podemos elejirnos $(2 \; 2 \; 17) - 13 \cdot (0 \; 0 \; 1)$

Luego dada esta base ortogonal, el producto interno queda definido sólo por la matriz de producto interno (que no nos importa), y además, como $(1 \; 1 \; 2)$ estaba ya en la base, tenemos que el ortogonal al generado por él es $\langle (1 \; 0 \; 1) \quad (0 \; 0 \; 1) \rangle$

\section*{Ejercicio 3}
Si $f : \C^2 \to \C^2$, $f(x,y) = (x - iy, 3x - 4iy)$, con
\[\langle (x,y), (z,w) \rangle = x\overline z + y \overline w\]
Tenemos:
\[\langle(x-iy, 3x-4iy, (z,w)\rangle =\]
\[(x-iy)\overline z + (3x-4iy) \overline w =\]
\[x\overline z-iy\overline z + 3x\overline w -4iy \overline w =\]
\[x\overline {(z + 3w)} + y\overline {(-iz -4i w)} =\]
\[\langle (x, y), (z + 3w, -iz -4i w \rangle =\]

\section*{Ejercicio 4}
Dado $\V = \R[x]$, $\langle p, q \rangle = \int_0^1 p(x)q(x) dx$, y $m_f(p) = f \cdot p$, calcular $m_f^*$
\[\langle m_f p, q\rangle = \langle p, m_f^* q \rangle\]
\[\int_0^1 f(x)p(x)q(x) dx = \int_0^1 p(x) m_f^*(q)(x) dx\]
Si tomamos $m_f^* = m_f$, es claro que esto anda, luego ya estamos.

\section*{Ejercicio 5}
Dado $\V = \C^{n \times n}$, con $f(A) = P^{-1}AP$, hallar la adjunta y ver que es autoadjunta cuando $P$ es hermitiana
\[\langle f(A), B \rangle =\]
\[\tr P^{-1}APB^* =\]
\[\tr AP^{-1}B^*P =\]
\[\tr A(P^{-*}BP^*)^* =\]
\[\langle A, P^{-*}BP^* \rangle =\]
Luego $f^*(B) = P^{-*}BP^*$, que cumple $f = f^*$ cuando $P$ es hermitiana

\section*{Identidades}
\begin{itemize}
    \item $(f \circ g)^* = g^* \circ f^*$
    \item $(f^*)^* = f$
    \item $\ker (f^*) = (\Ima f)^\perp$
    \item $\Ima (f^*) = (\ker f)^\perp$ (Para finitos)
    \item $\ker f = \ker (f^* \circ f)$
    \item $\Ima f = \Ima (f \circ f^*)$
\end{itemize}
Notemos que 
\[w \in \ker (f^*) \iff\]
\[f^*(w) = 0 \iff\]
\[\forall v : \langle v, f^*(w) \rangle = 0 \iff\]
\[\forall v : \langle f(v), w \rangle = 0 \iff\]
\[w \perp \Ima f\]

Además, si $w \in \ker f$, queremos ver que $w \perp f^*(v)$, luego
\[\langle w, f^*(v) \rangle =\]
\[\langle fw, v \rangle =\]
\[0\]
Por lo que $\Ima f^* \inn (\ker f) ^ \perp$

Tomemos $v \in \ker (f^* \circ f)$, luego:
\[
    0 = \langle v, f^*(f(v)) \rangle =
\]
\[
    \langle f(v), f(v) \rangle
\]
Luego $f(v) = 0$

Notemos además que:
\[
\Ima (f \circ f^*) = \]
\[\ker (f \circ f^*)^\perp =\]
\[\ker (f^{**} \circ f^*)^\perp =\]
\[(\ker f^*)^\perp =\]
\[(\Ima f)^{\perp\perp} =\]
\[(\Ima f) =\]

\section*{Transformaciones Normales}
Una transformación $f : \V \to \V$ es normal sii $f \circ f^* = f^* \circ f$
Notemos que:
\begin{itemize}
    \item $||f(v)||^2 = ||f^*(v)||^2$
    \item $f$ normal, entonces $v$ autovector de autovalor $\l$ sii $v$ autovector de $f^*$ con autovalor $\overline \l$
\end{itemize}
\section*{Caracterización de las Normales}
$f$ normal sii $f$ admine una BON de autovectores.

Sea $\l \in \C$ un autovalor, de autovector $v$, luego tomamos $S = \langle v \rangle^\perp$. Tenemos que:
\[\langle fs, v \rangle =\]
\[\langle \l s, v \rangle =\]
\[\l\langle s, v \rangle = 0\]
Y así hacemos inducción.
\end{document}
