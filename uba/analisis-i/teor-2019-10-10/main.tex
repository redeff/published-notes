\documentclass{article}
\usepackage{amssymb}
\usepackage{mathtools}

\everymath{\displaystyle}
\setlength{\parskip}{3mm}
\setlength{\parindent}{0mm}

\def\R{\mathbb{R}}
\def\K{\mathbb{K}}
\def\V{\mathbb{V}}
\def\W{\mathbb{W}}
\def\WW{\mathcal{W}}
\def\PP{\mathcal{P}}
\def\FF{\mathcal{F}}
\def\U{\mathbb{U}}
\def\C{\mathbb{C}}
\def\N{\mathbb{N}}
\def\Q{\mathbb{Q}}
\def\Z{\mathbb{Z}}

\def\BB{\mathcal{B}}
\def\AA{\mathcal{A}}
\def\EE{\mathcal{E}}
\def\CC{\mathcal{C}}
\def\OO{\mathcal{O}}
\def\DD{\mathcal{D}}

\def\e{\varepsilon}
\def\inn{\subseteq}

\def\S{\mathbb{S}}
\def\T{\mathbb{T}}
\def\l{\lambda}

\def\To{\Rightarrow}
\def\from{\leftarrow}
\def\From{\Leftarrow}

\DeclareMathOperator{\cl}{cl}
\DeclareMathOperator{\li}{li}
\DeclareMathOperator{\tl}{tl}
\DeclareMathOperator{\Id}{Id}
\DeclareMathOperator{\tr}{tr}
\DeclareMathOperator{\oo}{o}
\DeclareMathOperator{\spec}{spec}
\DeclareMathOperator{\mult}{mult}

\DeclareMathOperator{\iso}{iso}
\DeclareMathOperator{\mono}{mono}
\DeclareMathOperator{\epi}{epi}
\DeclareMathOperator{\adj}{adj}

\DeclareMathOperator{\Nu}{Nu}
\DeclareMathOperator{\Ima}{Im}
\DeclareMathOperator{\id}{id}
\DeclareMathOperator{\ze}{ze}

\DeclareMathOperator{\rg}{rg}

\DeclareMathOperator{\Hom}{Hom}
\DeclareMathOperator{\GL}{GL}
\DeclareMathOperator{\cont}{cont}
\DeclareMathOperator{\Hs}{H}

\DeclareMathOperator{\D}{D}
\DeclareMathOperator{\lcm}{lcm}

\DeclareMathOperator{\ev}{ev}
\DeclareMathOperator{\sg}{sg}

\date{}
\author{}

\begin{document}
\section*{Formas cuadráticas}
Dada una matriz $A \in \K^{n \times n}$ simétrica, existe una base $\{v_i\}$ ortonormal con autovalores reales. Luego:
\[
	A = P \cdot
	\begin{bmatrix}
		\lambda_1 & 0 & \dots & 0 \\
		0 & \lambda_2 & \dots & 0 \\
		\vdots & \vdots & \ddots & \vdots \\
		0 & 0 & \dots & \lambda_n \\
	\end{bmatrix} \cdot
	P^{-1}
\]
Y como $P$ era ortonormal, tenemos $P^{-1} = P^t$, luego:
\[
	A = P \cdot
	\begin{bmatrix}
		\lambda_1 & 0 & \dots & 0 \\
		0 & \lambda_2 & \dots & 0 \\
		\vdots & \vdots & \ddots & \vdots \\
		0 & 0 & \dots & \lambda_n \\
	\end{bmatrix} \cdot
	P^t
\]
\section*{Formas cuadráticas}
Si tenemos $qv = vAv^t$, lo escribimos como $qv = vPDP^tv^t = (vP)D(vP)^t$, y como $D$ es diagonal, solo aparecen los térmonos al cuadrado, con los coeficientes correspondientes a los autovalores de $A$.

Luego si todos los autovalores tienen signo positivo o todos negativos, sabemos que es mínimo o máximo respectivamente.

Por el contrario, si hay un positivo y un negativo, luego necesariamente es un punto de silla.

Si no pasa ninguna de esas dos cosas, entonces se dice semidefinico (son todos o no negativos o no positivos, pero hay algún autovalor que es $0$), y no sabemos nada de la función.

\section*{Determinante}
Notemos que $\det A = \prod \lambda_i$ con $\lambda_i$ los autovalores, entonces calcular el determinante nos puede dar información acerca de la definitud de una forma cuadrática.

En particular, para que sea mínimo o máximo, debe ser invertible (pero no es sí y sólo si)

\section*{Derivadas Direccionales}
Tenemos $(t \mapsto f (v \cdot t))''o = v (\Hs f) v^t$

\section*{Sylvester}
Teníamos que existían $m, M \in \R$ tal que $m||v||^2 \leq qv \leq M||v||^2$, con $q$ una forma cuadrática.

Notemos que si $\{\lambda_i\}$ son los autovalores, entonces tenemos que $m = \min \{\lambda_i\}, M = \max \{\lambda_i\}$ son esos mínimos y máximos, ya que los cambios de base ortonormales rpeservan la normal.

\section*{Ejemplo 1}
Si tenemos $f \; xy = (y - 3x^3)(y-x^2)$, ver que:
\begin{enumerate}
	\item minimos locales en el origen por cada dirección.
	\item punto silla en $(0,0)$
\end{enumerate}

Es claro que $\det \Hs f = 0$, ya que si fuese $\det \Hs f > 0$ tenemos un mínimo o un máximo, y si $\det \Hs f > 0$, luego en alguna dirección tendríamos un máximo local, que no pasa.

Luego la forma cuadrática es semidefinida (positiva).
\section*{Ejemplo 2}
Tenemos $f \; xy = (x-y)^2 + x^2(x-1)^2$. Hallar y estudiar sus extremos locales.

Tenemos $\nabla f (x,y) = (2(x-y) + 2x(x-2)^2 + 2x^2(x-1), -2(x-y))$, luego igualamos esto a $0$.
Tenemos $x = y$, y los casos posibles son $x = 0,1,\frac{1}{2}$.

El hessiano estádado por:
\[
	\Hs f =
	\begin{bmatrix}
		2 + 2(x-1)^2 + 8x(x-1) + 2x^2 & -2\\
		-2 & 2
	\end{bmatrix}
\]

con $x=y=0$, tenemos $\Hs f = 
\begin{bmatrix}
	4 & -2 \\
	-2 & 2
\end{bmatrix}$ como es positivo el determinante, tenemos que es necesariamente un extremo local.

Pero notemos que la forma cuadrática dada por esta matriz en $v = (1,0)$ vale $4$, luego necesariamente estoy en un mínimo local.

\section*{Criterio de Hessiano}
Si tenemos $
H =
\begin{bmatrix}
	a & b \\
	b & d
\end{bmatrix}
$
Luego la forma cuadrática dada por $qv = vHv^t$ cumple:
\begin{itemize}
	\item $q$ definida positiva $\iff$ $ac-b^2 > 0$ y $a > 0$.
	\item $q$ definida negativa $\iff$ $ac-b^2 > 0$ y $a < 0$.
	\item $q$ indefinida $\iff$ $ac-b^2 < 0$.
\end{itemize}

Si tenemos $
H =
\begin{bmatrix}
	A & b^t \\
	b & c
\end{bmatrix}
$, con $A \in \R{^2 \times 2}$

\section*{Criterio de Sylvester}
Si tenemos $H = 
\begin{bmatrix}
	A & b^t \\
	b & c
\end{bmatrix}$
con $H \in \R^{n \times n}$, $A \in \R^{n-1 \times n-1}$, luego son equivalentes:
\begin{itemize}
	\item $x \mapsto xHx^t$ es definida positiva
	\item $A$ es definida positiva y $\det H > 0$
\end{itemize}

Esto se ve ya que si $H$ no es def+, luego como $\det H > 0$, tiene almenos dos autovalores negativos, entonces el subespacio generado por esos dos itersecta al generado por los primeros $n-1$ canónicos, luego $A$ no es def+.
\end{document}
