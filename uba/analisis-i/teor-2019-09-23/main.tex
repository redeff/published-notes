\documentclass{article}
\usepackage{amssymb}
\usepackage{mathtools}

\everymath{\displaystyle}
\setlength{\parskip}{3mm}
\setlength{\parindent}{0mm}

\def\R{\mathbb{R}}
\def\K{\mathbb{K}}
\def\V{\mathbb{V}}
\def\W{\mathbb{W}}
\def\WW{\mathcal{W}}
\def\PP{\mathcal{P}}
\def\FF{\mathcal{F}}
\def\U{\mathbb{U}}
\def\C{\mathbb{C}}
\def\N{\mathbb{N}}
\def\Q{\mathbb{Q}}
\def\Z{\mathbb{Z}}

\def\BB{\mathcal{B}}
\def\AA{\mathcal{A}}
\def\EE{\mathcal{E}}
\def\CC{\mathcal{C}}
\def\OO{\mathcal{O}}
\def\DD{\mathcal{D}}

\def\e{\varepsilon}
\def\inn{\subseteq}

\def\S{\mathbb{S}}
\def\T{\mathbb{T}}
\def\l{\lambda}

\def\To{\Rightarrow}
\def\from{\leftarrow}
\def\From{\Leftarrow}

\DeclareMathOperator{\cl}{cl}
\DeclareMathOperator{\li}{li}
\DeclareMathOperator{\tl}{tl}
\DeclareMathOperator{\Id}{Id}
\DeclareMathOperator{\tr}{tr}
\DeclareMathOperator{\oo}{o}
\DeclareMathOperator{\spec}{spec}
\DeclareMathOperator{\mult}{mult}

\DeclareMathOperator{\iso}{iso}
\DeclareMathOperator{\mono}{mono}
\DeclareMathOperator{\epi}{epi}
\DeclareMathOperator{\adj}{adj}

\DeclareMathOperator{\Nu}{Nu}
\DeclareMathOperator{\Ima}{Im}
\DeclareMathOperator{\id}{id}
\DeclareMathOperator{\ze}{ze}

\DeclareMathOperator{\rg}{rg}

\DeclareMathOperator{\Hom}{Hom}
\DeclareMathOperator{\GL}{GL}
\DeclareMathOperator{\cont}{cont}
\DeclareMathOperator{\Hs}{H}

\DeclareMathOperator{\D}{D}
\DeclareMathOperator{\lcm}{lcm}

\DeclareMathOperator{\ev}{ev}
\DeclareMathOperator{\sg}{sg}

\date{}
\author{}

\title{Derivadas de Mayor Orden y Aproximaciones}
\begin{document}
\maketitle
\section*{Notación}
$f^{(n)}$ es derivar $f$ $n$ veces.
\section*{Lagrange Multivariable}
Sea $A \in \R^n$ un abierto convexo
con $f : A \to \R$ diferenciable.

Sean $X, Y \in A$, luego existe $Z$ en el segmento $\overline{XY} \inn A$ tal que:
\[
	fX - fY = \langle\nabla f Z, Y-Z \rangle
\]

Demo: Sea $\alpha : [0,1] \to \R^n, \alpha = t \mapsto (Y-X)t + X$ la función que parametriza el intervalo $\overline{XY}$.

Como $\Ima \alpha \inn A$, luego está bien definida la composición $f \circ \alpha : [0,1] \to \R$, que es diferenciable por regla de la cadena.

Tenemos $\alpha' t = Y-X$, luego $(f \circ \alpha)'t = \nabla f(\alpha t) \cdot \alpha' t$. Notemos que $f \circ \alpha$ cumple las hipótesis de lagrange, entonces tenemos que existe un $t \in (0, 1)$ tal que:
\[
	(f \circ \alpha)1 - (f \circ \alpha)0 = (f \circ \alpha)' t
\]
\[
	fY - fX = (f \circ \alpha)' t
\]
\[
	fY - fX = \nabla f(\alpha t) \times \alpha' t
\]
\[
	fY - fX = \nabla fZ \times (Y-Z) 
\]
Entonces está.

\subsection*{Corolario}
Entonces si tenemos $P, X \in \R^n$, luego $fX = fP + \nabla fZ \cdot (X-P)$ para algún $Z \in \overline{XY}$. Notemos que esto se parece al poly de taylor, ya que $\nabla fZ$ es el error de la aproximación trivial por un plano paralelo al dominio.

\section*{Opraciones en Funciones $\CC^k$}
Si tengo $f, g \in \CC^k$, luego:
\begin{itemize}
	\item $f + g$
	\item $f \cdot g$
	\item $\frac{f}{g}$, con $gx \neq 0$
	\item $g \circ f$
\end{itemize}
son $\CC^k$

\section*{Derivadas de Orden Superior}
Tomemos una función $f : \R^n \to \R$, en principio $f \in \CC^\infty$
Notemos que una derivada parcial en el canónico $x$ es $\frac{\partial f}{\partial x }  = f_x: \R^n \to \R$, entonces la puedo derivar de nuevo.

Las derivadas parciales dobles se notan
\[
	\frac{\partial}{\partial y} \left(\frac{\partial f}{\partial x} \right) = \frac{\partial^2f}{\partial y \partial x}  = f_{yx}
\]

\subsection*{Derivadas Cruzadas (Claraut-Schwarz)}
Sea $f : (A \inn \R^n) \to \R, f \in \CC^2$.

Luego $f_{xy} = f_{yx}$, para cualesquiera canónicos $x$, $y$.

Como corolario, si tenemos una función en $\CC^n$, luego las derivadas $n$-ésimas son todas permutables.

\section*{Taylor Monovariable}
Si teníamos $g : (a, b) \to \R, g \in \CC^3$. Sea $p \in (a, b)$, luego para cualquier otro punto $x$, existe un punto $c$ entre $p$ y $x$ tal que:
\[
	gx = \sum_{n=0}^{d-1} \frac{g^{(n)}p}{n!} \cdot (x-p)^n + \frac{g^{(d)}c}{d!} \cdot (x-p)^d
\]

\section*{Taylor Multivariable}
Dado $P = (a,b)$, $X = (x,y) \in \R^2$, si $f \in \CC^3$, luego existen $C=(c,d)\in\overline{PX}$
tal que:
\[f (x,y) = P(x,y) + R(x,y)\]
Con el polinomio dado por:
\[P(x,y) = fP\]
\[ + (x-a)f_xP + (y-b)f_yP\]
\[ + \frac{1}{2}(x-a)^2f_{xx} + (x-a)(y-b)f_{xy} + \frac{1}{2}(y-b)^2f_{yy}\]
Y el resto dado por:
\[R(x,y) = \frac{1}{6}(x-a)^3f_{xxx} C\]
\[ + \frac{1}{2}(x-a)^2(y-b)f_{xxy}\]
\[ + \frac{1}{2}(x-a)(y-b)^2f_{xyy}\]
\[ + \frac{1}{6}(y-b)^3f_{yyy}\]
\end{document}
