\documentclass{article}
\usepackage{amssymb}
\usepackage{mathtools}

\everymath{\displaystyle}
\setlength{\parskip}{3mm}
\setlength{\parindent}{0mm}

\def\R{\mathbb{R}}
\def\K{\mathbb{K}}
\def\V{\mathbb{V}}
\def\W{\mathbb{W}}
\def\WW{\mathcal{W}}
\def\PP{\mathcal{P}}
\def\FF{\mathcal{F}}
\def\U{\mathbb{U}}
\def\C{\mathbb{C}}
\def\N{\mathbb{N}}
\def\Q{\mathbb{Q}}
\def\Z{\mathbb{Z}}

\def\BB{\mathcal{B}}
\def\AA{\mathcal{A}}
\def\EE{\mathcal{E}}
\def\CC{\mathcal{C}}
\def\OO{\mathcal{O}}
\def\DD{\mathcal{D}}

\def\e{\varepsilon}
\def\inn{\subseteq}

\def\S{\mathbb{S}}
\def\T{\mathbb{T}}
\def\l{\lambda}

\def\To{\Rightarrow}
\def\from{\leftarrow}
\def\From{\Leftarrow}

\DeclareMathOperator{\cl}{cl}
\DeclareMathOperator{\li}{li}
\DeclareMathOperator{\tl}{tl}
\DeclareMathOperator{\Id}{Id}
\DeclareMathOperator{\tr}{tr}
\DeclareMathOperator{\oo}{o}
\DeclareMathOperator{\spec}{spec}
\DeclareMathOperator{\mult}{mult}

\DeclareMathOperator{\iso}{iso}
\DeclareMathOperator{\mono}{mono}
\DeclareMathOperator{\epi}{epi}
\DeclareMathOperator{\adj}{adj}

\DeclareMathOperator{\Nu}{Nu}
\DeclareMathOperator{\Ima}{Im}
\DeclareMathOperator{\id}{id}
\DeclareMathOperator{\ze}{ze}

\DeclareMathOperator{\rg}{rg}

\DeclareMathOperator{\Hom}{Hom}
\DeclareMathOperator{\GL}{GL}
\DeclareMathOperator{\cont}{cont}
\DeclareMathOperator{\Hs}{H}

\DeclareMathOperator{\D}{D}
\DeclareMathOperator{\lcm}{lcm}

\DeclareMathOperator{\ev}{ev}
\DeclareMathOperator{\sg}{sg}

\date{}
\author{}

\begin{document}
\section*{Producto interno}
El producto interno se define $\langle , \rangle : \R^n \to \R^n \to \R$, por la ecuación $\langle x, y \rangle = \sum x_iy_i$.

\subsection*{Cauchy-Schwarz}
Se tiene que $|\langle v, w \rangle| \leq ||v|| \cdot ||w||$

Demo:

tomemos $\lambda = \frac{\langle v, w \rangle}{\langle v, v \rangle}$, entonces tenemos
\[0 \leq \langle w - \lambda v, w - \lambda v \rangle =\]
\[||w||^2 - 2\lambda\langle v, w\rangle + \lambda^2 ||v||^2 =\]
\[||w||^2 - 2\frac{\langle v, w\rangle^2}{\langle v, v \rangle} + \lambda^2 ||v||^2 =\]
\[||w||^2 - 2\frac{\langle v, w\rangle^2}{\langle v, v \rangle} + \frac{\langle v, w\rangle^2}{\langle v, v \rangle} =\]
\[||w||^2 - 2\frac{\langle v, w\rangle^2}{||v||^2} + \frac{\langle v, w\rangle^2}{||v||^2} =\]
\[||w||^2 - \frac{\langle v, w\rangle^2}{||v||^2} \To\]
\[||v||^2 \cdot||w||^2 \geq \langle v, w\rangle^2\]

\section*{Máximo Crecimiento}
Como corolario de lo anterior, la dirección de máximo creciemeinto de una función en un punto no crítico es $\frac{\nabla f p}{||\nabla f p||} $. Esto se usa para el método de \emph{gradient descent}, para encontrar mínimos de funciones feas.

\section*{Diferenciación de funciones en $\R^n \to \R^m$}
\subsection*{Transformación Lineal}
Una $T : \R^n \to \R^m$ se dice \emph{transformación lineal} si $T (v+w) = Tv + Tw$, y $T (\lambda v) = \lambda Tv$

\subsection*{Diferencial}
Dada una función $f : \R^n \to \R^m$ y un punto $p \in \R^n$, decimos que una transformación lineal $T : \R^n \to \R^m$ \emph{aproxima bien la función} si se cumple que:
\[
	\lim_{x \to p}\frac{fx - fp - T(x-p)}{||x-p||} = 0_{\R^m}
\]
notemos que si descomponemos $f$ en sus componentes $f_iv = (fv)_i$, luego tenemos que es necesario y suficiente que cada una de las $f_i$ sea diferenciabla para que $f$ sea diferenciable. Y aún más tenemos:
\[[T] =
\begin{bmatrix}
	\vdots \\
	\nabla f_i p \\
	\vdots 
\end{bmatrix}
\]

Decimos entonces que la función $f:\R^n \to \R^m$ es \emph{diferenciable} si existe tal $T$, y la \emph{diferencial} de $f$ en $p$ es $[T] \in \R^{m \times n}$
, y se nota $Df(p) = [T]$

\subsection*{Ejemplo}
Si tenemos $fxy = (x+y, xy, x^2 + y^3)$, luego tenemos
\[Df(x, y) = 
\begin{bmatrix}
	1 & 1 \\
	y & x \\
	2x & 3y^2 \\
\end{bmatrix}\]
Y es claramente diferenciable porque cada componente es $\CC_1$.

Luego tenemos que
\[Df(1, 2) = 
\begin{bmatrix}
	1 & 1 \\
	2 & 1 \\
	2 & 12 \\
\end{bmatrix}\]
Entonces podemos aproximar la función cerca de $(1, 2)$:
\[
	f(x,y) \approx f(1, 2) + 
\begin{bmatrix}
	1 & 1 \\
	2 & 1 \\
	2 & 12 \\
\end{bmatrix} \times
\begin{bmatrix}
	x - 1 \\
	y - 2 \\
\end{bmatrix}
\]

\section*{Acotar con transformaciones lineales}
Para cualquier norma, para toda matriz $T \in \R^m \times \R^n$, tenemos que existe una constante $M$ tal que $||Tv|| \leq M ||v||$.

Demo:
Si tomamos $A = \partial B(0, 1) \in \R^n$ la $n$-esfera de radio $1$, luego como $A$ es compacto y $x \mapsto ||Tx||$ es contínua, tenemos que la función alcanza un máximo $M$ en $A$. Además, como $T$ es lineal, este máximo será el mismo para toda esfera. Entonces ya está. En otras palabras:
\[||Tx|| = ||x|| \cdot \left|\left|T\frac{x}{||x||}\right|\right| \leq ||x|| \cdot |M|\]
Entonces estamos. A este valor lo llamamaos $M = ||T||$.

\section*{Propiedad}
Si $f : (A \inn \R^n) \to \R^m$ es deferenciable en $p \inn A^\circ$, luego existe una bola $p \in B \inn A$ y una constante $c > 0$ tal que:
\[
	\forall x \in B : \frac{||fx - fp||}{||x-p||} < c
\]

Demo: Tenemos $fx = fp + Dfp \cdot (x-p) + o(x-p)$. Luego tomemos
\[||fx - fp - Dfp \cdot (x-p)|| \leq ||x-p||\]
\[\frac{||fx - fp - Dfp \cdot (x-p)||}{||x-p||} \leq 1\]
\[\frac{||fx - fp||}{||x-p||} - \frac{||Dfp \cdot (x-p)||}{||x-p||} \leq 1\]
\[\frac{||fx - fp||}{||x-p||} \leq 1+ \frac{||Dfp \cdot (x-p)||}{||x-p||}\]
\[\frac{||fx - fp||}{||x-p||} \leq 1+ ||Dfp||\]
Entonces estamos.

\section*{Regla de la Cadena}
Tenemos $A \inn R^n$, $B \inn \R^m$, $C \inn \R^t$, y funciones $f : A \to B$, $g : B \to C$. Y sea $p \inn A$.

Supongamos que $f$ diferenciable en $p$, y $g$ diferenciable en $fp$, luego tenemos que $g \circ f$ es diferenciable en $p$ y se tiene:
\[D(g \circ f)p = Dg(fp) \times Dfp\]

Demo:
Tenemos
\[fx = fp + Dfp \cdot (x-p) + r_fx\]
\[gx = g(fp) + Dg(fp) \cdot (x-fp) + r_gx\]

Reemplazamos $fx$ en $g$ y tenemos:
\[g(fx) = g(fp) + Dg(fp) \cdot (fx-fp) + r_g(fx)\]
\[g(fx) = g(fp) + Dg(fp) \cdot (Dfp \cdot (x-p) + r_fx) + r_g(fx)\]
\[g(fx) = g(fp) + Dg(fp) \cdot Dfp \cdot (x-p) + Dg(fp) \cdot r_fx + r_g(fx)\]

Entonces nos falta chequear que
\[\lim_{x \to p} \frac{||Dg(fp) \cdot r_fx + r_g(fx)||}{||x-p||} = 0\]

Notemos que
\[\lim_{x \to p} \frac{||Dg(fp) \cdot r_fx||}{||x-p||} = 0 \From\]
\[\lim_{x \to p} \frac{||Dg(fp)|| \cdot ||r_fx||}{||x-p||} = 0 \From\]
\[||Dg(fp)|| \cdot\lim_{x \to p} \frac{ ||r_fx||}{||x-p||} = 0\]
Que es verdad por definición de $r_f$

Para el segundo cacho, tenemos
\[\lim_{x \to p} \frac{||r_g(fx)||}{||x-p||} = 0 \From\]
Si $fx = fp$ entonces ya estamos. Si no:
\[\lim_{x \to p} \frac{||r_g(fx)||}{||fx-fp||} \cdot \frac{||fx-fp||}{||x-p||} = 0\]

Como vimos que $\frac{||fx-fp||}{||x-p||}$ está acotado, ya estamos.

\section*{Ejemplo}
Si tenemos $hx = x^x$, entonce podemos excribirla como $h = (x,y \mapsto x^y) \circ (x \mapsto (x,x))$. Y diferenciamos las dos y multiplicamos.
\end{document}
