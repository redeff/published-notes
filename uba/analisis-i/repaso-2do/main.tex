\documentclass{article}
\usepackage{amssymb}
\usepackage{mathtools}

\everymath{\displaystyle}
\setlength{\parskip}{3mm}
\setlength{\parindent}{0mm}

\def\R{\mathbb{R}}
\def\K{\mathbb{K}}
\def\V{\mathbb{V}}
\def\W{\mathbb{W}}
\def\WW{\mathcal{W}}
\def\PP{\mathcal{P}}
\def\FF{\mathcal{F}}
\def\U{\mathbb{U}}
\def\C{\mathbb{C}}
\def\N{\mathbb{N}}
\def\Q{\mathbb{Q}}
\def\Z{\mathbb{Z}}

\def\BB{\mathcal{B}}
\def\AA{\mathcal{A}}
\def\EE{\mathcal{E}}
\def\CC{\mathcal{C}}
\def\OO{\mathcal{O}}
\def\DD{\mathcal{D}}

\def\e{\varepsilon}
\def\inn{\subseteq}

\def\S{\mathbb{S}}
\def\T{\mathbb{T}}
\def\l{\lambda}

\def\To{\Rightarrow}
\def\from{\leftarrow}
\def\From{\Leftarrow}

\DeclareMathOperator{\cl}{cl}
\DeclareMathOperator{\li}{li}
\DeclareMathOperator{\tl}{tl}
\DeclareMathOperator{\Id}{Id}
\DeclareMathOperator{\tr}{tr}
\DeclareMathOperator{\oo}{o}
\DeclareMathOperator{\spec}{spec}
\DeclareMathOperator{\mult}{mult}

\DeclareMathOperator{\iso}{iso}
\DeclareMathOperator{\mono}{mono}
\DeclareMathOperator{\epi}{epi}
\DeclareMathOperator{\adj}{adj}

\DeclareMathOperator{\Nu}{Nu}
\DeclareMathOperator{\Ima}{Im}
\DeclareMathOperator{\id}{id}
\DeclareMathOperator{\ze}{ze}

\DeclareMathOperator{\rg}{rg}

\DeclareMathOperator{\Hom}{Hom}
\DeclareMathOperator{\GL}{GL}
\DeclareMathOperator{\cont}{cont}
\DeclareMathOperator{\Hs}{H}

\DeclareMathOperator{\D}{D}
\DeclareMathOperator{\lcm}{lcm}

\DeclareMathOperator{\ev}{ev}
\DeclareMathOperator{\sg}{sg}

\date{}
\author{}

\title{Repaso de Análisis I, Segundo Parcial}
\begin{document}
    \maketitle
    \section*{Derivadas Unidimensionales}
    \begin{defi}
    Dada $f : (I \inn \R) \to \R$, se define:
    \[
        f'(x) = \lim_{h \to 0} \frac{f(x) + f(x+h)}{h}
    \]
    Como la derivada de $f$ en $x$, para cualquier $x \in I^\circ$. En general, cuando hablamos de derivada, siempre tomamos el punto en el interior del dominio
    \end{defi}
    \section*{Derivadas con Imagen Multidimensional}
    Si tenemos una función $f : \R \to \R^n$, definida por $f(x_i)_i = (f_i(x_i))_i$, definimos $f'(x_i)_i = (f_i'(x_i))_i$; es decir, derivando cada componente.

    \section*{Derivadas con Dominio Multidimensional}
    Tomemos una $f : \R^n \to \R$
    \subsection*{Derivadas Direccionales}
    Si $v \in \R^n$, $||v|| = 1$, decimos que:
    \[
        \frac{\partial f}{\partial v}(p) = f_v(p) = \lim_{t \to 0} \frac{f(p + vt) - f(p)}{t}
    \]
    Notemos que la recta que pasa por $p$ y tiene director $(v, f_v(p))$ derá la recta ``tangente'' en esa dirección que pasa por $p$.
    \subsubsection*{Derivadas Parciales}
    No son más que las derivadas direccionales pero con vector director un canónico, y se notan:
    \begin{nota}
    \[
        \frac{\partial f}{\partial x_i}(p) = f_{x_i}(p) = f_{e_i}(p)
    \]
    \end{nota}
    \subsection*{Diferenciabilidad}
    Queremos aproximar a la función $f : \R^n \to \R$ alrededor de un punto $p \in \R^n$ con un plano tangente. La normal al plano debe ser perpendiular a los $(e_i, f_{x_i}(p))$, y la que cumple es $(f_{x_i}(p), \dots_i, -1)$ Luego tenemos que la ecuación del plano está dada por:
    \[\Pi : \R^n \to \R, \Pi(v_i)_i = fp + \sum (v_i - p_i)f_{x_i}(p)\]
    Para que este plano sea tangente, y la dunción sea diferenciable, necesitamos que:
    \[
        \lim_{v \to p} \frac{|f(v) - \Pi(v)|}{||v-p||} = 0
    \]
    Entonces:
    \begin{defi}
        Una función se llama diferenciable sii:
    \[
        \lim_{v \to p} \frac{|f(v) - \Pi(v)|}{||v-p||} = 0
    \]
    \end{defi}
    Al vector $(f_{x_i})_i$ se lo llama:
    \[
        (f_{x_i})_i = \nabla f(p)
    \]
    Y tenemos
    \[\Pi(v) = f(p) + \langle v - p, \nabla f(p) \rangle\]

    Se define además $\D f(p) : \R^n \to \R$ como la transformación lineal dada por:
    \[
        \D f(p)(v) = \langle v, \nabla f(p) \rangle
    \]
    \subsubsection*{Propiedades}
    Si $f$ es diferenciable, luego
    \[
        f_v(p) = \D f(p)(v)
    \]
    \subsection*{Criterio de Diferenciabilidad}
    \begin{teo}
    Supongamos que $f : \R^n \to \R$, y tenemos las $n$ derivadas parciales $(f_i)_i$ definidas en el origen $p$, y además las primeras $n-1$ están definidas en un entorno y con contínuas en $p$. Luego $f$ es diferenciable en $p$.
    \end{teo}

    \begin{demo}
    WLOG $f_i(p) = f_i(p) = 0$ y $p$ es el origen.
    Tomamos:
    \[
        f(0, 0, \dots, x_i, x_{i+1}, \dots, x_n) =
    \]
    \[
        f(0, 0, \dots, 0, x_{i+1}, \dots, x_n) +
        f_i(0, 0, \dots, c_i, x_{i+1}, \dots, x_n)x_i =
    \]
    Por Lagrange, para algún $c_i \in (0, x_i)$, luego:
    \[
        f(x_i) = f(0, \dots, 0, x_n) + \sum_{i=1}^{n-1} f_i(0, 0, \dots, c_i, x_{i+1}, \dots, x_n) x_i
    \]

    Tenemos que probar que:
    \[
    \frac{f(p)}{||p||} \to 0
    \]
    Cada uno de los sumandos es:
    \[
        \frac{f_i(0, 0, \dots, c_i, x_{i+1}, \dots, x_n) x_i}{||(x_j)_j||} \leq
    \]
    \[
        \frac{f_i(0, 0, \dots, c_i, x_{i+1}, \dots, x_n) x_i}{|x_i|} \leq
    \]
    \[
        f_i(0, 0, \dots, c_i, x_{i+1}, \dots, x_n)
    \]
    Que por continuidad (la hipótesis), tiende a $0$.
    Además el término suelto es de la pinta:
    \[
        \lim_{x_n \to 0} \frac{f(0, 0, \dots, 0, x_n)}{|x_n|}
    \]
    Que es justamente $f_n(0)_i$, que asusmimos que es $0$. Luego por álgebra de límites ya estamos.
    \end{demo}
    \subsection*{Diferenciabilidad Multivariable}
    \begin{defi}
        Dada una función $f : \R^n \to \R^m$, se dice que $Df(p) : \R^n \to \R^m$ es la única transformación lineal que cumple
        \[
            \lim_{x \to p} \frac{f(x) - f(p) - Df(p)(x-p)}{||x-p||} = 0
        \]
    \end{defi}
    \section*{Regla de La Cadena}
    \begin{teo}
        Dadas $f : \R^n \to \R^m$, y $g : \R^m \to \R^k$, con $f$ diferenciable en $p$ y $g$ diferenciable en $f(p)$, luego
        \[\D (f \circ g)(p) = (\D g)(f(p)) \cdot \D f(p)\]
    \end{teo}
    \section*{Fermat}
    \begin{teo}
        Si $p$ es un mínimo local de $f : \R^n \to \R$, luego tenemos que $Df(p) = 0$.
    \end{teo}
    \section*{Función Inversa}
    \begin{teo}
        Dada una función $f : \R^n \to \R^n$, y $p \in \R^n$ con $f$ de clase $\CC^1$ en un entorno de $p$ y $\D f(p)$ inversible, luego existen $V, W$ entornos de $p$ y $f(p)$ resp. tal que $f : V \to W$ biyectiva y:
        \[
            \forall q \in V : (\D f^{-1})(f(q)) = (\D f(q))^{-1}
        \]
    \end{teo}
    \begin{demo}
        WLOG $\D f(p) = I$. Por continuidad tomamos entorno $V$ tal que
        \[\forall q \in V : ||\D f(q) - I|| \leq \frac{1}{2}\]
        Luego por teorema de valor medio en $f - \id$ queda que:
        \[\forall x, y \in V : ||(f - \id)(x) - (f - \id)(y)|| \leq \frac{1}{2}||x-y||\iff\]
        \[\forall x, y \in V : ||(x - y) - (f(x) - f(y))|| \leq \frac{1}{2}||x-y||\iff\]
        \[\forall x, y \in V : ||x - y|| - ||f(x) - f(y)|| \leq \frac{1}{2}||x-y||\iff\]
        \[\forall x, y \in V : ||x - y|| \leq 2||f(x) - f(y)||\]
        Luego es inyectiva y la inversa es contínua.
        Ahora queremos ver que hay un entorno de $f(p)$ que está en al imagen de $f(V)$.

        Para esto tomemos una bola $B$ en $V$ y tomemos $C$ su borde (una circunferencia). Tenemos que $||\cdot|| \circ f : C \to \R$ tiene dominio compacto, luego alcanza su mínimo y éste no es $0$ por inyectividad

        Sea $2d > 0$ este mínimo. Vamos a trabajar sobre la bola $K = B_d(f(p))$. Vamos a pobar que todo $y \in K$, hay una preimagen $f^{-1}(y) \in B$.

        Tomemos la función $h : B \to \R$, definida por $h(x) = ||f(x) - y||^2$. Por compactidad alcanza su mínimo, además por como definimos $K$ tenemos que para $x \in C$, $h(x) > d^2$, y además para $x = p$ se tiene $f(x) < d^2$, luego el mínimo se alcanza en el interior.

        Sea $x$ su mínimo. Por Fermat, tenemos entonces:
        \[
            \D h(x) = 0
        \]
        \[
            \D (x \mapsto \langle f(x) - y, f(x) - y \rangle)(x) = 0
        \]
        \[
            (x \mapsto 2 \cdot \langle \D f(x), f(x) - y \rangle)(x) = 0
        \]
        \[
            \langle \D f(x), f(x) - y \rangle = 0
        \]
        \[
            \D f(x) \times (f(x) - y)^t = 0
        \]
        Y como $\D f(x)$ es inversible por estar en $B$, tenemos que esto implica que $f(x) - y = 0$, luego ya estamos.
    \end{demo}
    \section*{Función Implícita}
    \begin{teo}
    Dada una función $f : \R^{n+1} \to \R$, $f \in \CC^1$ con $p \in \R^{n+1}$ si $f_1(p) \neq 0$, entonces existe $\phi : (B \inn \R^n) \to \R$ y un abierto $p \in A$ tal que:
    \[
        \{x : f(x) = 0\} \cap A = \{(\phi(x), x) : x \in B\}
    \]
    Donde $\{x : f(x) = 0\}$ es una superficie de nivel y $\{(\phi(x), x) : x \in B\}$ es el gráfico de una función.

    Se tiene además:
    \[
        \forall y \in B : \phi_i(y) = -\frac{f_i(y)}{f_1(y)}
    \]
    Y que el plano tangente a todo punto $q \in A$ es:
    \[
        x \mapsto (\langle \nabla f(q), x - q \rangle = 0)
    \]
    \end{teo}
    \begin{demo}
        Sale fácil con inversa. Tomemos la auxiliar
        \[t : \R^{n+1} \to \R^{n+1}, \; t(x_i)_i = (f(x_i)_i, x_2, x_3, \dots)\]
        Si $f_1(p) \neq 0$, claramente $\D t(p)$ es inversible, luego existe su unversa $h$. Tendremos que si nos movemos por el hiperplano $(0, x_2, x_3, \dots)$ en el dominio de $h$, se cumplirá que las imágenes (en el dominio de $t$), estarán en la superficie de nivel. Nos tomamos entonces
        \[\phi(x_2,x_3,\dots) = h(0,x_2,x_3,\dots)\]
        Además, por inyectividad, son los únicos en un entorno y ya estamos.

        Además, aplicando regla de la cadena a $f(\phi(x), x)$, tenemos:
        \[\nabla (x \mapsto f(\phi(x), x))(x) = 0\]
        \[\nabla f(\phi(x), x) \cdot (\nabla \phi(x), 1) = 0\]
        Pero $\phi(x)$ era la primer coordenada si estábamos en la superficie de nivel, osea que tenemos:
        \[\nabla f(x_1,x_2,\dots) \cdot (\nabla \phi(x_2,x_3,\dots), I) = 0\]
        \[\nabla f(x_1,x_2,\dots) \cdot (\nabla \phi(x_2,x_3,\dots), I) = 0\]
        Dónde $(\nabla \phi(x_2,x_3,\dots), I)$ es una matriz. Tomando la $i$-ésima coordenada del vector resultante tenemos:
        \[f_i(x) + f_1(x)\phi_i(x_{\geq 2}) = 0\]
        \[\phi_i(x_{\geq 2}) = -\frac{f_i(x)}{f_1(x)}\]
        Entonces ya estamos.
    \end{demo}
    \section*{Taylor Una Variable}
    \begin{defi}
        Dada una función $f : \R \to \R$, con $f \in \CC^n$ y un $p \in \R$, el polinomio de taylor $P(x)$ es el único polinomio de grado menor o igual a $n$ que verifica:
        \[
            \forall k \leq n : P^{(k)}(p) = f^{(k)}(p)
        \]
        Y su expresión es:
        \[
            P(x) = \sum_{i=0}^n \frac{1}{i!} \cdot f^{(i)}(p) \cdot (x-p)^i
        \]
        Y se dice que $R(x) = f(x) - P(x)$ es el resto de Taylor
    \end{defi}
    \begin{teo}
        Aproximación del resto de taylor.
        Dada $f : \R \to \R$ $n+1$ veces derivable, se tiene que:
        \[
            R(x) = (x-p)^{n+1} \cdot \frac{f^{(n+1)}(\xi)}{(n+1)!}
        \]
    \end{teo}
    \begin{demo}
        Tomemos la función auxiliar
        \[g(t) = f(x) - \frac{\omega}{(n+1)!}(x-t)^{n+1} -\sum \frac{1}{i!} \cdot f^{(i)}(t) \cdot (x-t)^i\]
        Con $\omega$ tal que $g(p) = 0$. Como $g(p) = g(x) = 0$, tenemos que:
        \[
            g'(\xi) = 0
        \]
        Para algún $\xi$ entre $p$ y $x$. Luego, por regla del producto:
        \[
            g'(\xi) = 0 \iff
        \]
        \[
            0 = - \frac{\omega}{n!}(\xi-t)^{n} - \sum \frac{1}{i!} \cdot f^{(i+1)}(t) \cdot (\xi-t)^i - \frac{1}{(i-1)!}f^{(i)}(t) \cdot (\xi - t)^{i-1} \iff
        \]
        Por telescópica
        \[
            0 = - \frac{\omega}{n!}(\xi-t)^{n} - \frac{1}{n!} \cdot f^{(n+1)}(t) \cdot (\xi-t)^n\iff
        \]
        Luego $\omega = f^{(n+1)}(t)$, por lo que, reemplazando $\omega$ en $g(p)$ tenemos:
        \[0 = g(p) = f(x) - \frac{f^{(n+1)}(t)}{(n+1)!}(x-p)^{n+1} -\sum \frac{1}{i!} \cdot f^{(i)}(t) \cdot (x-p)^i\]
        \[
            \frac{f^{(n+1)}(t)}{(n+1)!}(x-p)^{n+1} =
            f(x) - P(x)
        \]
        Entonces estamos.
    \end{demo}
    \section*{Formas Cuadráticas}
    \begin{defi}
        Dada una matriz $T \in \R^{n \times n}$, decimo que su forma cuadrática asociada esa dada por:
        \[
            Q_T : \R^n \to \R^n, \quad Q_T(x) = x^t \cdot T \cdot x
        \]
    \end{defi}
    \begin{teo}
        El compotamiento de las formas cuadráticas está dada por sus autovalores.
    \end{teo}
    \begin{demo}
        Si tenemos $T$ simétrica, luego tomemos una base $\{v_i\}$ ortonormal de autovectores, con autovalores $\l_i$, tenemos:
        \[
            x = \sum \alpha_i v_i
        \]
        Luego:
        \[
            Q_T(x) = \sum_{i,j} (\alpha_i v_i)^t \cdot T \cdot \alpha_j v_j
        \]
        \[
            Q_T(x) = \sum_{i,j} \alpha_i \alpha_j (v_i^t \cdot T \cdot v_j)
        \]
        \[
            Q_T(x) = \sum_{i,j} \alpha_i \alpha_j (v_i^t \cdot \l_j v_j)
        \]
        \[
            Q_T(x) = \sum_{i,j} \alpha_i \alpha_j \l_j (v_i^t \cdot v_j)
        \]
        Pero $v_i^t \cdot v_j = 0$ para $i \neq j$, y es $1$ cuando son iguales, luego:
        \[
            Q_T(x) = \sum_i \alpha_i^2 \l_i
        \]
    \end{demo}
    \begin{defi}
        Se dice que
        \begin{itemize}
            \item $Q_T$ es \emph{degenerada} cuando hay autovalores nulos.
            \item $Q_T$ es \emph{indefinida} cuando hay autovalores positivos y negativos.
            \item $Q_T$ es \emph{definida positiva} cuando $\l_i > 0$.
            \item $Q_T$ es \emph{definida negativa} cuando $\l_i < 0$.
            \item $Q_T$ es \emph{semidefinida positiva} cuando $\l_i \geq 0$.
            \item $Q_T$ es \emph{semidefinida negativa} cuando $\l_i \leq 0$.
        \end{itemize}
    \end{defi}
    \section*{Extremos}
    \begin{defi}
        Dada una función $f : \R^n \to \R$, decimos que $p \in \R^n$ es un \emph{extremo local} sii $f(p) \leq f(a)$ para todo $a$ en un abierto que contenga a $p$.

        Decimos que es un \emph{extremo local estricto} sii pasa lo mismo pero estricto (con $p \neq a$).
    \end{defi}
    \begin{teo}
        Ya vimos que para que $p$ sea extremo local su diferencial tiene que ser $0$. Pero eso no alcanza. Si $\D f(p) = 0$, Tenemos que:
        \begin{itemize}
            \item Si $\Hs f(p)$ es \emph{definido positivo}, entonces $p$ es mínimo local estricto.
            \item Si $\Hs f(p)$ es \emph{definido negativo}, entonces $p$ es máximo local estricto.
            \item Si $\Hs f(p)$ es \emph{indefinido}, entonces $p$ es un punto silla.
            \item Si no se cumple nada de esto, no sabemos nada.
        \end{itemize}
    \end{teo}
    \begin{teo}
        Una matriz $T \in \R^{n \times n}$ es definida prositiva sii todos los determinantes menores principales (los de las matrices cuadradas que están en la parte suerior izquierda de la matriz) son estrictamente positivos.

        Asimismo, una matriz es definida negativa si todos esos determinantes son de signos alternados (el de la esquina superior iquierda tiene que comenzar siendo negativo).
    \end{teo}
    \section*{Mínimos en Conjuntos Arbitrarios}
    Los multiplicadores de lagrange sirven para hallar mínimos locales y absolutos de funciones $f : (A \in \R^n) \to \R$. Para esto, se separa en dos casos:
    \begin{itemize}
        \item El interior $A^\circ$, en el cual se pueden hallar puntos críticos uzando fermat (donde el diferencial se anula)
    \item El borde $\partial A$, el cual se puede estudiar parmetrizándolo por una (o varias) funciones $\phi : \R^{n-1} \to \R^n$, y luego estudiando la composición $f \circ \phi$ recursivamente. Si no es posible parmetrizarlo, hay que usar multiplicadores de lagrange.
    \end{itemize}
    \section*{Multiplicadores de Lagrange}
    \begin{teo}
        Dada una función $f : \R^n \to \R$ que queremos minimizar sobre una superficie de nivel $S = \{x \in \R^n : g(x) = 0\}$ de una función $g : \R^n \to \R$, los candidatos a extremos serán los $p \in \R^n$ donde:
        \[
            \nabla f(p) = \l \cdot \nabla g(p) \quad \text{Para algún $\l \in \R$}
        \]
        O los $p$ que anulan $\nabla g(p)$.

        En general, si tenemos $k$ restricciones $g_{1, \dots, k}$, luego los puntos críticos son los $p \in \R^n$ tal que:
        \[
            \{\nabla f(p), \nabla g_1(p), \dots, \nabla g_k(p) \} \; \text{Son li.}
        \]
    \end{teo}
    \section*{Integrales de Riemann}
    Se definen de manera usual.
    \section*{Teorema Fundamental de Cálculo}
    \begin{teo}
        Si $f : [a, b] \to \R$ es contínua, sea $F : [a, b] \to \R$ dada por:
        \[F(x) = \int_a^x f\]
        Luego $F$ es contínua en $[a,b]$, derivable en $(a,b)$, y además:
        \[\forall x \in (a,b) : F'(x) = f(x)\]
    \end{teo}
\end{document}
