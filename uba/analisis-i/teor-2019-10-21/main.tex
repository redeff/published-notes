\documentclass{article}
\usepackage{amssymb}
\usepackage{mathtools}

\everymath{\displaystyle}
\setlength{\parskip}{3mm}
\setlength{\parindent}{0mm}

\def\R{\mathbb{R}}
\def\K{\mathbb{K}}
\def\V{\mathbb{V}}
\def\W{\mathbb{W}}
\def\WW{\mathcal{W}}
\def\PP{\mathcal{P}}
\def\FF{\mathcal{F}}
\def\U{\mathbb{U}}
\def\C{\mathbb{C}}
\def\N{\mathbb{N}}
\def\Q{\mathbb{Q}}
\def\Z{\mathbb{Z}}

\def\BB{\mathcal{B}}
\def\AA{\mathcal{A}}
\def\EE{\mathcal{E}}
\def\CC{\mathcal{C}}
\def\OO{\mathcal{O}}
\def\DD{\mathcal{D}}

\def\e{\varepsilon}
\def\inn{\subseteq}

\def\S{\mathbb{S}}
\def\T{\mathbb{T}}
\def\l{\lambda}

\def\To{\Rightarrow}
\def\from{\leftarrow}
\def\From{\Leftarrow}

\DeclareMathOperator{\cl}{cl}
\DeclareMathOperator{\li}{li}
\DeclareMathOperator{\tl}{tl}
\DeclareMathOperator{\Id}{Id}
\DeclareMathOperator{\tr}{tr}
\DeclareMathOperator{\oo}{o}
\DeclareMathOperator{\spec}{spec}
\DeclareMathOperator{\mult}{mult}

\DeclareMathOperator{\iso}{iso}
\DeclareMathOperator{\mono}{mono}
\DeclareMathOperator{\epi}{epi}
\DeclareMathOperator{\adj}{adj}

\DeclareMathOperator{\Nu}{Nu}
\DeclareMathOperator{\Ima}{Im}
\DeclareMathOperator{\id}{id}
\DeclareMathOperator{\ze}{ze}

\DeclareMathOperator{\rg}{rg}

\DeclareMathOperator{\Hom}{Hom}
\DeclareMathOperator{\GL}{GL}
\DeclareMathOperator{\cont}{cont}
\DeclareMathOperator{\Hs}{H}

\DeclareMathOperator{\D}{D}
\DeclareMathOperator{\lcm}{lcm}

\DeclareMathOperator{\ev}{ev}
\DeclareMathOperator{\sg}{sg}

\date{}
\author{}

\begin{document}
\section*{Forma Implícita y Paramétrica}
Forma implícita de un plano:
\[
	P = \{(x,y,z) : ax + by + cz = d\}
\]

Forma paramétrica:
\[
	P = \{(x, y, f(x,y)) : x,y \in \R\}
\]
La forma paramétrica se rompe cuando el coeficiente de $z$ es $0$, pero en esos casos siempre podemos despejar otra variable.

La forma implícita también se puede escribir como:
\[\Pi = \{x : \langle v, x \rangle = \langle v, p \rangle = 0\}\]
donde $p$ es un punto en el plano y $v$ es el fector normal.
Notemos que si $fx = \langle v, x \rangle$, luego $\nabla f = v^t$.

Las rectas de la forma $y = ax + b$ son también la curva de nivel dada por $F(x, y) = -b$, con $F(x) = ax - y$, es decir, $\nabla F = (a, -1)$

\section*{Conjuntos de Nivel}
si tenemos una función $f(x,y)$ y tomamos una curva de nivel $f(x,y) = k$, luego por ejemplo si $x^2+y^2 = 1$, luego en todo punto hay un entorno donde podemos despejar $y$ en función de $x$ o viceversa.

Entonces supongamos que queremos $\max \{g(x,y) : f(x,y) = k\}$ para una función $g : \R^2 \to \R$. Si podemos partir el gráfico de la curva de nivel $f(x,y) = k$ en gráficos de a pedazos, luego podemos aplicar regla de la cadena e igualar a $0$.

\section*{Función Implícita}
Si tenemos $p \in \{(x,y) : f(x,y) = r\} = S_r(f), f \in \CC_1$ una curva de nivel y una parametrización local $\sigma : (0,1) \to B, \sigma \in \CC_1$ tal que $\Ima \sigma = S_r(f) \cap B$, con $B$ un entorno de $p$.

Lugo notemos que por definición $f \circ \sigma$ es constantemente $r$.
Luego por regla de la cadena:
\[
	\nabla f (\sigma t) \cdot \nabla \sigma t = (f \circ \sigma)' = 0
\]
\[
	\nabla f p \cdot \nabla \sigma t_0 = 0
\]
Luego el gradiente de $\sigma$ es ortogonal al gradiente de $f$.

\section*{Recta tangente}
Si tenemos una función $f : \R^n \to \R, f \in \CC^1$, luego la recta tangente a $S_{fp}(f)$ por $p$ está dada por $\ell = \{v \in \R^n : \langle\nabla fp, v-p \rangle = 0\}$

El razonamiento es el mismo que para $f : \R^2 \to \R$

\section*{Ejercicio}
Hallar la ecuación de la recta tangente en $(1,2)$ a la curva $\{e^{2x-y} -xy + 1 = 0\}$.

Sea $F(x,y) = e^{2x-y} -xy + 1$. Luego queremos calcular:
\[\nabla F(x,y) = (2e^{2x-y}-y, -e^{2x-y}-x)\]
\[\nabla F(1,2) = (0, -2)\]
Luego la tangente está dada por
\[\ell = \{v : \langle (0,-2), v \rangle = \langle (0,-2), (1,2) \rangle \}\]
\[\ell = \{v : -2v_y = -4 \}\]
\[\ell = \{v : v_y = 2 \}\]

\section*{Teorema de la Función Implícita}
Si tenemos $f : \R^2 \to \R, F \in \CC^1$, con $p \in F$, y sea $r = Fp$, con $\frac{\partial F}{\partial y}p \neq 0$, luego existe $f (B \inn \R) \to \R, f\in \CC^1$, tal que $p_x \in B^o$ y $F(x, fx) = r, \forall x \in B$
\end{document}
