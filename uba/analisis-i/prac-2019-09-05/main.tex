\documentclass{article}
\usepackage{amssymb}
\usepackage{mathtools}

\everymath{\displaystyle}
\setlength{\parskip}{3mm}
\setlength{\parindent}{0mm}

\def\R{\mathbb{R}}
\def\K{\mathbb{K}}
\def\V{\mathbb{V}}
\def\W{\mathbb{W}}
\def\WW{\mathcal{W}}
\def\PP{\mathcal{P}}
\def\FF{\mathcal{F}}
\def\U{\mathbb{U}}
\def\C{\mathbb{C}}
\def\N{\mathbb{N}}
\def\Q{\mathbb{Q}}
\def\Z{\mathbb{Z}}

\def\BB{\mathcal{B}}
\def\AA{\mathcal{A}}
\def\EE{\mathcal{E}}
\def\CC{\mathcal{C}}
\def\OO{\mathcal{O}}
\def\DD{\mathcal{D}}

\def\e{\varepsilon}
\def\inn{\subseteq}

\def\S{\mathbb{S}}
\def\T{\mathbb{T}}
\def\l{\lambda}

\def\To{\Rightarrow}
\def\from{\leftarrow}
\def\From{\Leftarrow}

\DeclareMathOperator{\cl}{cl}
\DeclareMathOperator{\li}{li}
\DeclareMathOperator{\tl}{tl}
\DeclareMathOperator{\Id}{Id}
\DeclareMathOperator{\tr}{tr}
\DeclareMathOperator{\oo}{o}
\DeclareMathOperator{\spec}{spec}
\DeclareMathOperator{\mult}{mult}

\DeclareMathOperator{\iso}{iso}
\DeclareMathOperator{\mono}{mono}
\DeclareMathOperator{\epi}{epi}
\DeclareMathOperator{\adj}{adj}

\DeclareMathOperator{\Nu}{Nu}
\DeclareMathOperator{\Ima}{Im}
\DeclareMathOperator{\id}{id}
\DeclareMathOperator{\ze}{ze}

\DeclareMathOperator{\rg}{rg}

\DeclareMathOperator{\Hom}{Hom}
\DeclareMathOperator{\GL}{GL}
\DeclareMathOperator{\cont}{cont}
\DeclareMathOperator{\Hs}{H}

\DeclareMathOperator{\D}{D}
\DeclareMathOperator{\lcm}{lcm}

\DeclareMathOperator{\ev}{ev}
\DeclareMathOperator{\sg}{sg}

\date{}
\author{}

\begin{document}
\section{Ejercicio 1}
Sea $f(x, y) = \frac{x \cdot y}{x^2 + y^2} $.
Calcular $\lim_{(x, y) \to (0, 0)} fxy$.

Necesitamos hacer un límite por curva, es decir, encontrar
$\alpha : (0,1) \to \R^2$ tal que $\lim_{t \to 0} \alpha t = (0, 0)$, y evaluar
$\lim_{t \to 0} f(\alpha t)$ para encontrar un candidato.

Tomemos $\alpha_1 t = (t, 0)$, luego $f(\alpha_1 t) = \frac{0}{t^2} = 0$, luego tenemos
que $\lim_{t \to 0} f (\alpha_1 t) = 0$.

Si tomamos $\alpha_2 t = (t, t)$, luego $f(\alpha_2 t) = \frac{t^2}{2 \cdot t^2}
= \frac{1}{2}$,
luego el límite es $\lim_{t \to 0} f(\alpha_2 t) = \frac{1}{2} $.
Entonces el límite no existe.

Podemos probar todas las rectas juntas:
$\alpha_{ab}t = (at, bt)$, luego la composición va a ser $f \circ \alpha_{ab} =
t \mapsto \frac{a \cdot t \cdot b \cdot t}{(a \cdot t)^2 + (b \cdot t)^2} =
\frac{a \cdot b}{a^2 + b^2}$, cuyo límite no es constante en $a, b$.

\section{Ejercicio 2}
Sea $f = x,y \mapsto \frac{x \cdot y^2}{x^2 + y^4} $. Calculemos el
$\lim_{\to(0, 0)} f$.

Sea $\alpha_{ab} = t \mapsto (a \cdot t, b \cdot t)$. Luego
$f \circ \alpha = t \mapsto
\frac{t^3 \cdot a \cdot b^2}{a^2 \cdot t^2 + b^4 \cdot t^4}
= \frac{a \cdot b^2 \cdot t}{a^2 + b^4 \cdot t^2} $.
Si $a \neq 0$, luego el límite da $0$, porque lo de arriba tiende a $0$ y lo de abajo no.
Si $a = 0$ lo de arriba es $0$, y lo de abajo no, luego el límite también es $0$.

Entonces el límite es $0$ para todas las rectas.

tomemos $\alpha t = (t^2, t)$, luego $f \circ \alpha = t \mapsto \frac{t^4}{2t^4}
= \frac{1}{2} $, luego el límite es $\frac{1}{2} $, entonces no existe el límite.

\section{Ejercicio 3}
Dada $f = x,y \mapsto \frac{x^2 \cdot y}{x^2 + x - y}$, calcular el límite
$\lim_{\to(0,0)} f$.

Tomemos $\alpha t = (t, t + t^2 + t^3)$, tenemos

$f \circ \alpha = t\mapsto \frac{t^2 \cdot (t + t^2 + t^3)}{-t^3} = -(1 + t + t^2) = -1$,
luego el límite por esta curva es $-1$, entonces el límite total no existe.

\section{Composición de Límites}
Dadas $f : \R \to \R$ y $g : \R^2 \to \R$, si $\lim_{x \to x_0} gx = L$, y si
$\lim_{x \to L} fx = K$, luego $\lim_{x \to x_0} f (gx) = K$.

\section{Ejercicio 5}
Calcular limite de $f = x,y \mapsto \frac{1 - \cos (x^2 \cdot y)}{x^2 + y^2} $.

Notamos que $\frac{1 - \cos(x^2 \cdot y)}{x^2 \cdot y} \cdot \frac{x^2 \cdot y}{x^2 + y^2}$.

Queremos acotar por la norma de $||x,y||$ la expresión
$\left|\frac{x^2 \cdot y}{x^2 + y^2}\right| =
\frac{x^2 \cdot |y|}{||x,y||^2} \leq
\frac{||x,y||^2 \cdot ||x,y||}{||x,y||^2} \leq
||x,y||$
\end{document}
