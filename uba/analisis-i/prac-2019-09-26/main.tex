\documentclass{article}
\usepackage{amssymb}
\usepackage{mathtools}

\everymath{\displaystyle}
\setlength{\parskip}{3mm}
\setlength{\parindent}{0mm}

\def\R{\mathbb{R}}
\def\K{\mathbb{K}}
\def\V{\mathbb{V}}
\def\W{\mathbb{W}}
\def\WW{\mathcal{W}}
\def\PP{\mathcal{P}}
\def\FF{\mathcal{F}}
\def\U{\mathbb{U}}
\def\C{\mathbb{C}}
\def\N{\mathbb{N}}
\def\Q{\mathbb{Q}}
\def\Z{\mathbb{Z}}

\def\BB{\mathcal{B}}
\def\AA{\mathcal{A}}
\def\EE{\mathcal{E}}
\def\CC{\mathcal{C}}
\def\OO{\mathcal{O}}
\def\DD{\mathcal{D}}

\def\e{\varepsilon}
\def\inn{\subseteq}

\def\S{\mathbb{S}}
\def\T{\mathbb{T}}
\def\l{\lambda}

\def\To{\Rightarrow}
\def\from{\leftarrow}
\def\From{\Leftarrow}

\DeclareMathOperator{\cl}{cl}
\DeclareMathOperator{\li}{li}
\DeclareMathOperator{\tl}{tl}
\DeclareMathOperator{\Id}{Id}
\DeclareMathOperator{\tr}{tr}
\DeclareMathOperator{\oo}{o}
\DeclareMathOperator{\spec}{spec}
\DeclareMathOperator{\mult}{mult}

\DeclareMathOperator{\iso}{iso}
\DeclareMathOperator{\mono}{mono}
\DeclareMathOperator{\epi}{epi}
\DeclareMathOperator{\adj}{adj}

\DeclareMathOperator{\Nu}{Nu}
\DeclareMathOperator{\Ima}{Im}
\DeclareMathOperator{\id}{id}
\DeclareMathOperator{\ze}{ze}

\DeclareMathOperator{\rg}{rg}

\DeclareMathOperator{\Hom}{Hom}
\DeclareMathOperator{\GL}{GL}
\DeclareMathOperator{\cont}{cont}
\DeclareMathOperator{\Hs}{H}

\DeclareMathOperator{\D}{D}
\DeclareMathOperator{\lcm}{lcm}

\DeclareMathOperator{\ev}{ev}
\DeclareMathOperator{\sg}{sg}

\date{}
\author{}

\begin{document}
\section*{Polinomio de Taylor}
Tenemos por ejemplo $fx = \sqrt{x}$. Las derivadas son:
\[f x = x^{1/2}\]
\[f' x = \frac{1}{2} \cdot x^{-1/2}\]
\[f'' x = -\frac{1}{4} \cdot x^{-3/2}\]
\[f''' x = \frac{3}{8} \cdot x^{-5/2}\]

Luego el polinomio de Taylor en $1$ es:
\[
	P_2x = 1 + \frac{1}{2}(x-1) - \frac{1}{8}(x-1)^2
\]

Y el resto está dado por \[
	R_2x = \frac{1}{6} f'''c (x-1)^3
\]
\[
	R_2x = \frac{1}{16} c ^{-5/2} (x-1)^3
\]
Además, tenemos $c^{-5/2} < 1$, luego podemos acotar el polinomio por $\frac{1}{16} \cdot 0.2^3 < \frac{1}{100}$, luego tenemos que aproxima en $1.2$ con una precisión menor a $\frac{1}{100}$

\section*{Taylor Multivarible}
El polinomio de Taylor de orden dos está dado por:
\[
	P_2X = fP + \langle \nabla fP, X-P \rangle + \frac{1}{2} (X-P) (\Hs f P) (X-P)^t
\]

\section*{Ejercicio 1}
Dado \[
	f (x,y) = e^{3x+\sin y}
\]
Hallar el polinomio de orden $2$ en el origen.

Las derivadas parciales son:
\[
	f_{x} (x,y) = 3e^{3x + \sin y}
\]
\[
	f_{y} (x,y) = \cos y \cdot e^{3x + \sin y}
\]
\[
	f_{xx} (x,y) = 9e^{3x + \sin y}
\]
\[
	f_{yx} (x,y) = 3\cos y \cdot e^{3x + \sin y}
\]
\[
f_{xy} (x,y) = 3 \cos y \cdot e^{3x + \sin y}
\]
\[
	f_{yy} (x,y) = \cos^2 y \cdot e^{3x + \sin y} - \sin y \cdot e^{3x + \sin y}
\]

Luego el Hessiano en el origen es:
\[
	\Hs f (0,0) = 
	\begin{bmatrix}
		9 & 3 \\
		3 & 1
	\end{bmatrix}
\]

Y el gradiente vale:
\[
	\nabla f (0,0) =
	\begin{bmatrix}
		3 & 1
	\end{bmatrix}
\]

Y $f (0,0) = 1$

Luego el Taylor de orden 1 está dado por:
\[
	P_2 x = 1 + 
	\begin{bmatrix}
		3 & 1 \\
	\end{bmatrix} x^t + 
	\frac{1}{2} \cdot x \begin{bmatrix}
		9 & 3 \\
		3 & 1
	\end{bmatrix} x^t
\]
\[
	P_2 (x,y) = 1 + 3x + y + \frac{1}{2}(9x^2 + y^2 + 6xy)
\]
\[
	P_2 (x,y) = 1 + 3x + y + \frac{1}{2}(9x^2 + y^2 + 6xy)
\]

\section*{Ejercicio 2}
Tenemos $g (x,y) = (y \sin x + y - 1, x)$, y sea $f$ tal que el polinomio de Taylor de orden 2 de $f \circ g$ en $(0, 0)$ es
\[P_2(x,y) = 3 + 2x + 3x^2 + 2xy\]
Hallar el taylor 1 de $f$ en $(1,0)$

Notemos que $f (g (0, 0)) = f (1, 0)$, y como tenemo el polinomio de taylor, tenemos $f (1, 0) = 3$.

Además, veamos que
\[\D (f \circ g) (0, 0) = \D f (g (0, 0)) \cdot \D g (0, 0)\]
\[
\begin{bmatrix}
	2 & 0 \\
\end{bmatrix}= \D f (g (0, 0)) \cdot
\begin{bmatrix}
	0 & 1 \\
	1 & 0
\end{bmatrix}
\]

Luego queda un sistema lineal, y ya estamos.

\section*{Ejercicio 3}
Sea $f : \R^2 \to \R$, $f \in \CC^3$, calcular el taylor 2 de $g(x,y) = f(x, y^2)$ en el punto $(0, -1)$, sabiendo que el Taylor de $f(x,y)$ en $(0,1)$ es
\[P_2(x,y) = 1 + x - y + x^2 - 2xy\]

Tenemos por regla de la cadena:
\[
	\D g (0, 1) = \D f(0, 1) \cdot \D (x,y \mapsto x, y^2) (0, -1)
\]
Y tenemos todo.
\end{document}
