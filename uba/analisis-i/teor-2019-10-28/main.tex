\documentclass{article}
\usepackage{amssymb}
\usepackage{mathtools}

\everymath{\displaystyle}
\setlength{\parskip}{3mm}
\setlength{\parindent}{0mm}

\def\R{\mathbb{R}}
\def\K{\mathbb{K}}
\def\V{\mathbb{V}}
\def\W{\mathbb{W}}
\def\WW{\mathcal{W}}
\def\PP{\mathcal{P}}
\def\FF{\mathcal{F}}
\def\U{\mathbb{U}}
\def\C{\mathbb{C}}
\def\N{\mathbb{N}}
\def\Q{\mathbb{Q}}
\def\Z{\mathbb{Z}}

\def\BB{\mathcal{B}}
\def\AA{\mathcal{A}}
\def\EE{\mathcal{E}}
\def\CC{\mathcal{C}}
\def\OO{\mathcal{O}}
\def\DD{\mathcal{D}}

\def\e{\varepsilon}
\def\inn{\subseteq}

\def\S{\mathbb{S}}
\def\T{\mathbb{T}}
\def\l{\lambda}

\def\To{\Rightarrow}
\def\from{\leftarrow}
\def\From{\Leftarrow}

\DeclareMathOperator{\cl}{cl}
\DeclareMathOperator{\li}{li}
\DeclareMathOperator{\tl}{tl}
\DeclareMathOperator{\Id}{Id}
\DeclareMathOperator{\tr}{tr}
\DeclareMathOperator{\oo}{o}
\DeclareMathOperator{\spec}{spec}
\DeclareMathOperator{\mult}{mult}

\DeclareMathOperator{\iso}{iso}
\DeclareMathOperator{\mono}{mono}
\DeclareMathOperator{\epi}{epi}
\DeclareMathOperator{\adj}{adj}

\DeclareMathOperator{\Nu}{Nu}
\DeclareMathOperator{\Ima}{Im}
\DeclareMathOperator{\id}{id}
\DeclareMathOperator{\ze}{ze}

\DeclareMathOperator{\rg}{rg}

\DeclareMathOperator{\Hom}{Hom}
\DeclareMathOperator{\GL}{GL}
\DeclareMathOperator{\cont}{cont}
\DeclareMathOperator{\Hs}{H}

\DeclareMathOperator{\D}{D}
\DeclareMathOperator{\lcm}{lcm}

\DeclareMathOperator{\ev}{ev}
\DeclareMathOperator{\sg}{sg}

\date{}
\author{}

\begin{document}
\section*{Multiplicadores de Lagrange}
Si tenemos $f,g : (A \inn \R^n) \to_{\CC^1} \R$, luego tomemos el conjunto de nuvel $S = S_r(g)$. Luego tomemos un $p \in S$ que sea mínimo local restringido a $S$.

Luego el teorema nos dice que $\nabla f(p) \parallel \nabla g(p)$. Es decir, que $p$ es de la forma:
\[
	\nabla f(p) = \l \nabla g(p)
\]
o
\[
	\nabla g(p) = 0
\]
\section*{Ejemplo 1}
Tengo $S = \{y^2 = 8 - x^3\}$. Queremos el máximo de $f(x,y) = xy$ en $S \cap \{x,y \geq 0\}$.

Tomemos $g(x,y) = x^3 + y^2$, luego queremos el mínimo de $f$ en una curva de nivel de $g$, luego calculemos:
\[
	\nabla g(x,y) = (3x^2, 2y)
\]
\[
	\nabla f(x,y) = (y,x)
\]
Notemos que $\nabla g(x,y) = 0$ sólo para $(x,y) = (0,0)$, en cuyo caso $g(x,y) \neq 8$.

Busquemos entonces
\[
	\nabla f(x,y) = \l \nabla g(x,y)
\]
\[
	(y, x) = \l (3x^2, 2y)
\]
Con $g(x,y) = 8$. Queda el sistema de ecuaciones:
\[
	\begin{cases}
		y = 3\l x^2 \\
		x = 2\l y \\
		x^3 + y^2 = 8 \\
	\end{cases}
\]
Luego $x = 2\l y$, luego $y = 3\l (2\l y)^2 = 12\l^3 y^2$, y ahí resolvemos.

Alternativamente, tomemos el determinante:
\[
	\det \begin{bmatrix}
		\nabla f(x) \\
		\nabla g(x)
	\end{bmatrix} =
\]
\[
	2y^2 - 3x^3 = 0
\]
\[
	y^2 + x^8 = 0
\]
Que es un sistema lineal, y estamos.
\section*{Demo de Lagrange}
Tomemos $f, g : \R^2 \to_{\CC^1} \R$, con un mínimo local restringido a $g$ en $p$.

Si $\nabla gp = 0$, ya estamos.

Si no, entonces wlog $\frac{\partial g}{\partial y} \neq 0$. Como $g$ es $\CC^1$, por el teorema de función implícita, existe un $\phi : \R \to \R$ tal que $g(x, \phi (x)) = g(p_x,p_y)$ en un entorno.

Tomemos entonces la composición $h(x) = f(x, \phi x)$. Queremos que $p_x$ sea un mínimo local en $h$, luego $h'(p_x) = f_x(p_x, \phi p_x) + \phi'(p_x) f_y(p_x, \phi p_x) = 0$

Además, sabemos que $\phi'(p_x) = -\frac{g_x(p)}{g_y(p)} $. Reemplazando:
\[
	f_x(p_x, \phi p_x) - \frac{g_x(p)}{g_y(p)}  f_y(p_x, \phi p_x) = 0
\]
\[
	f_x(p) - \frac{g_x(p)}{g_y(p)}  f_y(p) = 0
\]
\[
	f_x(p)g_y(p) - g_x(p) f_y(p) = 0
\]
Que es la determinante

\section*{Observación}
Si tenemos que $\{ \nabla f(p), \nabla g(p) \}$ son li, luego forman una base. Luego la función $F(x,y) = (f(x,y), g(x,y))$ tiene un diferencial $DF(p)$ inversible, luego por función inversa es inversible en un entorno.

Luego puedo encontrar un $p'$ tal que $F(p') = (f(p) - \e, g(p))$, luego $p$ no es mínimo.

Esto nos dá otra demsotración.

\section*{Lagrange Multidimensional}
Si tenemos varias restricciones, entonces vale que el conjunto de todos los gradientes en el punto (tanto de las restricciones como de la función a minimizar) forman un conjunto linealmente dependiente.

\section*{Ejemplo 2}
Hallar los puntos más cercanos y lejanos del origen en el conjunto
\[S = \{(x,y,z) : z^2 = x^2 + y^2, x-2z = 3\}\]
Que es la intersección entre un cono y un plano.
Tomemos
\[f(x,y,z) = x^2 + y^2 + z^2\]
\[g(x,y,z) = -z^2 + x^2 + y^2\]
\[h(x,y,z) = x - 2z\]
Si las derivamos tenemos:
\[\nabla f(x,y,z) = (2x, 2y, 2z)\]
\[\nabla g(x,y,z) = (2x, 2y, -2z)\]
\[\nabla h(x,y,z) = (1, 0, -2)\]
Queremos entonces la determinante
\[
	\det \begin{bmatrix}
		2x & 2y & 2z \\
		2x & 2y & -2z \\
		1 & 0 & -2 \\
	\end{bmatrix} = -8yz - 2 \cdot 0 = -8yz = 0
\]
Luego o bien $y = 0$ o bien $z = 0$.

Pero notemos que si $z = 0$, luego ese plano solo interseca al cono en el orígen, que no pasa por el otro plano. Luego necesariamente $y = 0$.

Luego tenemos el sistema de ecuaciones:
\[
	\begin{cases}
		y = 0 \\
		z^2 = x^2 \\
		x - 2z = 3
	\end{cases}
\]
Despejo y listo. Nos dá los casos:
\[
	(x,y,z) = \begin{cases}
	(-3, 0, -3) \\
	(1, 0, -1)
	\end{cases}
\]
Que son el máximo y el mínimo respectivamente.
\end{document}
