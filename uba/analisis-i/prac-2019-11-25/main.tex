\documentclass{article}
\usepackage{amssymb}
\usepackage{mathtools}

\everymath{\displaystyle}
\setlength{\parskip}{3mm}
\setlength{\parindent}{0mm}

\def\R{\mathbb{R}}
\def\K{\mathbb{K}}
\def\V{\mathbb{V}}
\def\W{\mathbb{W}}
\def\WW{\mathcal{W}}
\def\PP{\mathcal{P}}
\def\FF{\mathcal{F}}
\def\U{\mathbb{U}}
\def\C{\mathbb{C}}
\def\N{\mathbb{N}}
\def\Q{\mathbb{Q}}
\def\Z{\mathbb{Z}}

\def\BB{\mathcal{B}}
\def\AA{\mathcal{A}}
\def\EE{\mathcal{E}}
\def\CC{\mathcal{C}}
\def\OO{\mathcal{O}}
\def\DD{\mathcal{D}}

\def\e{\varepsilon}
\def\inn{\subseteq}

\def\S{\mathbb{S}}
\def\T{\mathbb{T}}
\def\l{\lambda}

\def\To{\Rightarrow}
\def\from{\leftarrow}
\def\From{\Leftarrow}

\DeclareMathOperator{\cl}{cl}
\DeclareMathOperator{\li}{li}
\DeclareMathOperator{\tl}{tl}
\DeclareMathOperator{\Id}{Id}
\DeclareMathOperator{\tr}{tr}
\DeclareMathOperator{\oo}{o}
\DeclareMathOperator{\spec}{spec}
\DeclareMathOperator{\mult}{mult}

\DeclareMathOperator{\iso}{iso}
\DeclareMathOperator{\mono}{mono}
\DeclareMathOperator{\epi}{epi}
\DeclareMathOperator{\adj}{adj}

\DeclareMathOperator{\Nu}{Nu}
\DeclareMathOperator{\Ima}{Im}
\DeclareMathOperator{\id}{id}
\DeclareMathOperator{\ze}{ze}

\DeclareMathOperator{\rg}{rg}

\DeclareMathOperator{\Hom}{Hom}
\DeclareMathOperator{\GL}{GL}
\DeclareMathOperator{\cont}{cont}
\DeclareMathOperator{\Hs}{H}

\DeclareMathOperator{\D}{D}
\DeclareMathOperator{\lcm}{lcm}

\DeclareMathOperator{\ev}{ev}
\DeclareMathOperator{\sg}{sg}

\date{}
\author{}

\begin{document}
\section*{Ejercicio 1}
Dado el elipsoide
\[E = \left\{(x,y,z) : \frac{x^2}{a^2} + \frac{y^2}{b^2} + \frac{z^2}{c^2} \leq 1\right\}\]
Calcular su volumen.
Tomemeos la esfera unitaria $B$, y la función inyectiva $\CC^1$ $T(x,y,z) = (ax, by, cz)$, luego:
\[
    \iiint\limits_{T(B)} 1 \; dB= \iiint\limits_B JT \; dB
\]
Pero $T(B) = E$, luego:
\[
    \iiint\limits_{E} 1 \; dE = \iiint\limits_{T(B)} 1 \; dB= \iiint\limits_B JT \; dB = JT \cdot \iiint\limits_B 1 \; dB
\]
Y el volumen de la esfera $B$ lo sé (es $\frac{4}{3}\pi$), y el Jacobiano de mi transformación es claramente $abc$, luego ya estamos. El resultado final es:
\[
    \frac{4}{3} \pi \cdot abc
\]
\section*{Ejercicio 2}
Calcular:
\[V = \iiint\limits_W z \; dV\]
Donde:
\[W = \left\{x^2 + y^2 + z^2 \leq 1, z \geq \frac{1}{2}\right\}\]
Con coordenadas esféricas queda:
\begin{itemize}
    \item $0 \leq \theta \leq 2\pi$
    \item $\frac{1}{2 \cos \phi} \leq r \leq 1$
    \item $0 \leq \phi \leq \cos^{-1} \frac{1}{2} = \frac{\pi}{3}$
\end{itemize}
Luego tenemos:
\[
    V = \int_0^{2\pi} \int_0^{\frac{\pi}{3}} \int_0^{\frac{1}{2\cos\phi}} z \cdot r^2 \cdot \sin \phi \; drd\phi d\theta
\]
Pero $z = r \cos \phi$, luego:
\[
    V = \int_0^{2\pi} \int_0^{\frac{\pi}{3}} \int_{\frac{1}{2\cos\phi}}^1 r^3 \cdot \sin \phi \cdot \cos \phi \; drd\phi d\theta
\]
\[
    \frac{V}{2\pi} = \int_0^{\frac{\pi}{3}} \int_{\frac{1}{2\cos\phi}}^1 r^3 \cdot \sin \phi \cdot \cos \phi \; drd\phi
\]
\[
    \frac{V}{2\pi} = \int_0^{\frac{\pi}{3}} \sin \phi \cdot \cos \phi \cdot \int_{\frac{1}{2\cos\phi}}^1 r^3\; drd\phi
\]
\[
    \frac{V}{2\pi} = \int_0^{\frac{\pi}{3}} \sin \phi \cdot \cos \phi \cdot \left(\frac{r^4}{4}\right|_{\frac{1}{2\cos\phi}}^1\; d\phi
\]
\[
    \frac{2V}{\pi} = \int_0^{\frac{\pi}{3}} \sin \phi \cdot \cos \phi \cdot \left(r^4\right|_{\frac{1}{2\cos\phi}}^1\; d\phi
\]
\[
    \frac{2V}{\pi} = \int_0^{\frac{\pi}{3}} \sin \phi \cdot \cos \phi \cdot \left(1 - \frac{1}{16\cos^4 \phi}\right)\; d\phi
\]
\[
    \frac{2V}{\pi} = \int_0^{\frac{\pi}{3}} \sin \phi \cdot \cos \phi \cdot d\phi
    -\int_0^{\frac{\pi}{3}} \sin \phi \cdot \cos \phi \cdot \frac{1}{16\cos^4 \phi}\; d\phi
\]
\[
    \frac{2V}{\pi} = \int_0^{\frac{\pi}{3}} \sin \phi \cdot \cos \phi \cdot d\phi
    -\int_0^{\frac{\pi}{3}} \sin \phi \cdot \frac{1}{16\cos^3 \phi}\; d\phi
\]
Que sale haciendo sustitución $u = \cos \phi$, y $du = -\sin \phi$, queda:
\[
    \frac{2V}{\pi} = \int_0^{\frac{1}{2}} - u \cdot du
    -\int_0^{\frac{1}{2}} - \frac{1}{16u^3}\; du
\]
\[
    \frac{2V}{\pi} = -\int_0^{\frac{1}{2}} u \cdot du
    +\frac{1}{16}\int_0^{\frac{1}{2}} u^{-3}\; du
\]
Que sale.
\section*{Ejercicio 3}
Dado:
\[W = \left\{
        1 \leq x^2 + y^2 + z^2 \leq 4,
        \frac{1}{3}(x^2 + y^2) \leq z^2 \leq 3(x^2 + y^2),
        z \geq 0
\right\}\]
Calcular:
\[
    V = \iiint\limits_W \frac{1}{(x^2 + y^2 + z^2)^2}
\]
Notemos que esto es una corona esférica intersecada con el espacio entre dos conos.
Los ángulos de los conos son, $\phi_2 = \tan^{-1}\sqrt 3 = \frac{\pi}{3}$, y $\phi_1 = \tan^{-1}\frac{1}{\sqrt 3} = \frac{\pi}{6}$
Luego tenemos:
\[
    V = \int_0^{2\pi} \int_{\frac{\pi}{6}}^{\frac{\pi}{3}} \int_1^2 \frac{1}{(x^2 + y^2 + z^2)^2} r^2 \sin \phi \; dr d\phi d\theta
\]
\section*{Parcial}
\subsection*{Uno}
Sea $f : \R^3 \to \R$ la función $f(x,y,z) = z^3 - 2yz + x$
\begin{enumerate}
    \item Probar que la ecuación $f(x,y,z) = 0$ define una función $z = \phi(x,y)$ de clase $\CC^1$ en un entorno del punto $(1,0)$ tal que $f(x, y, \phi(x,y)) = 0$ para todo $(x,y)$ en un entorno.
    \item Suponiendo que $\phi$ es $\CC^2$, determinar su polinomio de taylor de orden 2 alrededos del punto $(1,0)$
\end{enumerate}
\subsection*{Dos}
Sea $f : \R^2 \to \R$ dada por $f(x,y) = x^2 + y^2 - 2x$
\begin{enumerate}
    Encontrar todos los extremos locales de $f$ en:
    \[D = \{(x,y) \in \R^2 : x^2 + y^2 < 2, x > 0\}\]
    Y analizar la existencia de extremos absolutos en $D$.
\end{enumerate}
\subsection*{Tres}
Probar que si $\alpha > 1$, la integral:
\[\int_3^{+\infty}\;^3\sqrt{\frac{|\sin (x - 3)|}{x^\alpha(x^2 - 6x + 9)}}\]
\end{document}
