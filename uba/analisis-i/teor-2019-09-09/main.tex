\documentclass{article}
\usepackage{amssymb}
\usepackage{mathtools}

\everymath{\displaystyle}
\setlength{\parskip}{3mm}
\setlength{\parindent}{0mm}

\def\R{\mathbb{R}}
\def\K{\mathbb{K}}
\def\V{\mathbb{V}}
\def\W{\mathbb{W}}
\def\WW{\mathcal{W}}
\def\PP{\mathcal{P}}
\def\FF{\mathcal{F}}
\def\U{\mathbb{U}}
\def\C{\mathbb{C}}
\def\N{\mathbb{N}}
\def\Q{\mathbb{Q}}
\def\Z{\mathbb{Z}}

\def\BB{\mathcal{B}}
\def\AA{\mathcal{A}}
\def\EE{\mathcal{E}}
\def\CC{\mathcal{C}}
\def\OO{\mathcal{O}}
\def\DD{\mathcal{D}}

\def\e{\varepsilon}
\def\inn{\subseteq}

\def\S{\mathbb{S}}
\def\T{\mathbb{T}}
\def\l{\lambda}

\def\To{\Rightarrow}
\def\from{\leftarrow}
\def\From{\Leftarrow}

\DeclareMathOperator{\cl}{cl}
\DeclareMathOperator{\li}{li}
\DeclareMathOperator{\tl}{tl}
\DeclareMathOperator{\Id}{Id}
\DeclareMathOperator{\tr}{tr}
\DeclareMathOperator{\oo}{o}
\DeclareMathOperator{\spec}{spec}
\DeclareMathOperator{\mult}{mult}

\DeclareMathOperator{\iso}{iso}
\DeclareMathOperator{\mono}{mono}
\DeclareMathOperator{\epi}{epi}
\DeclareMathOperator{\adj}{adj}

\DeclareMathOperator{\Nu}{Nu}
\DeclareMathOperator{\Ima}{Im}
\DeclareMathOperator{\id}{id}
\DeclareMathOperator{\ze}{ze}

\DeclareMathOperator{\rg}{rg}

\DeclareMathOperator{\Hom}{Hom}
\DeclareMathOperator{\GL}{GL}
\DeclareMathOperator{\cont}{cont}
\DeclareMathOperator{\Hs}{H}

\DeclareMathOperator{\D}{D}
\DeclareMathOperator{\lcm}{lcm}

\DeclareMathOperator{\ev}{ev}
\DeclareMathOperator{\sg}{sg}

\date{}
\author{}

\begin{document}
	\section{Bolzano}
	Si se tiene una función contínua $f : [a, b] \to \R$, y se cumple $fa \cdot
	fb \leq 0$, luego existe $c \in [a, b] : fc = 0$

	\subsection{Equivalente Teorema de los Valores Intermedios}
	Dada una función contínua $f : [a, b] \to \R$, con $fa < fb$, luego tenemos
	$[fa, fb] \inn f [a, b]$

	\subsection{Lema}
	Sea $f : \R \to \R$ contínua, luego si $fx > 0$ para algún $x$, luego existe
	un entorno alrededor de $x$ donde la función es positiva.

	Por continuidad, notemos que la preimagen de $\left(\frac{fx}{2}, \frac{3 \cdot
	fx}{2}\right)$ es un abierto, contiene a $x$, y además las imágenes son
	todas positivas, luego ya estamos.

	\subsection{Demo con Binary Search}
	WLOG $fa < 0$ y $fb > 9$
	Si $fa \cdot fb = 0$, luego tomo $x = a$ o $x = b$ y estoy.
	Definamos $a_0 = a$, $b_0 = b$, y $c_i = \frac{a_i + b_i}{2}$.
	Podemos asumir que en ningún momento $fa_i$, $fb_i$, o $fc_i$ es
	cero osinó ya estamos. Definimos:
	\[b_{i+1} = 
		\begin{cases}
			b_i & \text{Si $c_i < 0$} \\
			c_i & \text{Osinó}
		\end{cases}
	\]\[
	a_{i+1} = 
		\begin{cases}
			a_i & \text{Si $c_i > 0$} \\
			c_i & \text{Osinó}
		\end{cases}
	\]

	Por inducción, $fa_i < 0$ y $fb_i > 0$; y también
	notemos que $a_i < b_i$, y además que $b_i - a_i = 2^{-i}
	\cdot (b - a)$, ya que en cada paso se divide en dos su distancia.

	Además notemos que $a_i$ es creciente y $b_i$ es decreciente.

	Entonces sea $e = \lim a$. Pero notemos que \[|b_i - e| = |(b_i - a_i) +
	(a_i - e)| \leq |b_i - a_i| + |a_i - e|\] Y notemos que ambos sumandos
	tienden a $0$, luego $\lim |b_i - e| = 0$, por lo que $\lim b = e$

	Notemos que si $fe \neq 0$, hay un entorno de $e$ con imágenes de un mismo
	signo que $fe$. Pero como tanto $a_i$ como $b_i$ se meten en ese entorno,
	y los $a_i$ son todos negativos y los $b_i$ son todos positivos,
	contradicción. Luego $fe = 0$

	\section{Arcoconexo}
	Un conjunto $C \inn \R^n$ es arcoconexo sii para cualquiera $p, q \in C$, 
	los puedo unir por una curva contínua dentro de $C$. Es decir: 
	\[\forall p, q \in C : \exists \alpha \cont
	: [0; 1] \to C : \alpha 0 = p, \alpha 1 = q\]
	\section{Bolzano Multidimensional}
		Dado $C \inn \R^n$ arcoconexo, y una función contínua $f : C \to \R$, y
		dos puntos $p, q \in C$ con $fp < 0$ y $fq > 0$, luego existe algún $x
		\in C$ tal que $fx = 0$

		Demo: por arcoconectividad, tenemos alguna $\alpha \cont : [0, 1] \to C$, con
		$\alpha 0 = p$, $\alpha 1 = q$, entonces por bolzano en una varibale en
		$f \circ \alpha$, ya estamos.

	\section{Derivadas}
	Dada una función $f : \R \to \R$, decimos que \emph{La derivaba de f en x},
	notada como $f'x$, se define por:
	\[f'x = \lim_{h \to 0} \frac{f(x+h) - fx}{h} \]

	\section{Principio de Fermat}
	Dada una función $f : [a, b] \to \R$, y algún $c \in (a, b)$, con $f$
	derivable en $c$, si $f$ tiene
	un máximo o un mínimo local en $c$ entonces tenemos
	$f'c = 0$.

	Demo: Supongamos que hay un mínimo local en $c$, luego se tiene que en un
	entorno de $c$, se tiene $fc < f(c+h)$, pera todo $h < \delta$ para algún
	$\delta$.

	Luego tenemos:

	$\lim_{h \to 0^+} \frac{fc - f(c+h)}{h} \geq 0$, y
	$\lim_{h \to 0^-} \frac{fc - f(c+h)}{h} \leq 0$, luego si el límite existe,
	debe ser $0$.

	\subsection{Punto de Ensilladura}
	Es un punto en una función donde la derivada es $0$, pero no es un extremo.
	
\end{document}
