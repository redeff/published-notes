\documentclass{article}
\usepackage{amssymb}
\usepackage{mathtools}

\everymath{\displaystyle}
\setlength{\parskip}{3mm}
\setlength{\parindent}{0mm}

\def\R{\mathbb{R}}
\def\K{\mathbb{K}}
\def\V{\mathbb{V}}
\def\W{\mathbb{W}}
\def\WW{\mathcal{W}}
\def\PP{\mathcal{P}}
\def\FF{\mathcal{F}}
\def\U{\mathbb{U}}
\def\C{\mathbb{C}}
\def\N{\mathbb{N}}
\def\Q{\mathbb{Q}}
\def\Z{\mathbb{Z}}

\def\BB{\mathcal{B}}
\def\AA{\mathcal{A}}
\def\EE{\mathcal{E}}
\def\CC{\mathcal{C}}
\def\OO{\mathcal{O}}
\def\DD{\mathcal{D}}

\def\e{\varepsilon}
\def\inn{\subseteq}

\def\S{\mathbb{S}}
\def\T{\mathbb{T}}
\def\l{\lambda}

\def\To{\Rightarrow}
\def\from{\leftarrow}
\def\From{\Leftarrow}

\DeclareMathOperator{\cl}{cl}
\DeclareMathOperator{\li}{li}
\DeclareMathOperator{\tl}{tl}
\DeclareMathOperator{\Id}{Id}
\DeclareMathOperator{\tr}{tr}
\DeclareMathOperator{\oo}{o}
\DeclareMathOperator{\spec}{spec}
\DeclareMathOperator{\mult}{mult}

\DeclareMathOperator{\iso}{iso}
\DeclareMathOperator{\mono}{mono}
\DeclareMathOperator{\epi}{epi}
\DeclareMathOperator{\adj}{adj}

\DeclareMathOperator{\Nu}{Nu}
\DeclareMathOperator{\Ima}{Im}
\DeclareMathOperator{\id}{id}
\DeclareMathOperator{\ze}{ze}

\DeclareMathOperator{\rg}{rg}

\DeclareMathOperator{\Hom}{Hom}
\DeclareMathOperator{\GL}{GL}
\DeclareMathOperator{\cont}{cont}
\DeclareMathOperator{\Hs}{H}

\DeclareMathOperator{\D}{D}
\DeclareMathOperator{\lcm}{lcm}

\DeclareMathOperator{\ev}{ev}
\DeclareMathOperator{\sg}{sg}

\date{}
\author{}

\begin{document}
	\section{Cuerpo Completo}
	Un cuerpo completo es aquel en el cual se cumple el axioma del supremo
	\subsection{$\Q$ no es completo}
	Sea $A = \{x \in \Q^+ \mid x^2 < 2\}$.

	Sabemos que $A$ es no vacío ($1 \in A$), y $2$ es una cota,
	luego para que se cumpla el axioma del supremo, debería tener supremo.

	Supongamos que $s \in \Q$ es supremo de $A$.
	Luego tenemos o bien $s \leq \sqrt{2}$ o $s \geq \sqrt{2}$.

	Si $s \leq \sqrt{2}$, luego existe por densidad $q \in \Q$ tal que $s < q < \sqrt{2}$,
	entonces $q \in A$, luego $s$ no es cota.

	Si $s \geq \sqrt{2}$, luego existe por densidad $q \in \Q$ tal que $\sqrt{2} < q < s$,
	entonces $q$ es cota de $A$, lo que contradice minimalidad de cota.

	\section{Sucesiones}
	\subsection{Ínfimo de Monótonas}
	Si tenemos una monotona decreciente $(a_i)_i$, luego tenemos que
	$\inf \{a_i\} = \lim_{n \to \infty} a_n$.

	Se demuestra trivialmente ($\forall \varepsilon \dots$).

	Para todo $\varepsilon > 0$, necesitamos ver que todos los elementos a partir de un
	punto están a distancia de $\in A$ menor a $\varepsilon$.

	Notemos que una vez que pasa, siempre pasa porque es decreciente, luego para
	que no pase, ninguno de los elementos está a dist menor a epsilon de $\inf A$.

	Pero luego $\inf A - \varepsilon$ es cota inferior. Contradicción-
	\subsection{Ejercicio 1}
	Hallar, si existe, sup, inf, max, y min de
	$A = \left\{\frac{4n-2}{2n-9} \;\middle| \; n \in \N\right\}$.

	Vamos a ver si $a_n > a_{n+1}$.
\begin{align*}
	\frac{4n-2}{2n-9} &> \frac{4(n+1)-2}{2(n+1)-9} \\
	\iff (4n-2)(2(n+1)-9) &> (4(n+1)-2)(2n-9) \quad \text{Para $n \geq 5$}\\
	\iff (4n-2)(2n-7) &> (4n+2)(2n-9) \\
	\iff 8n^2-4n-28n+14 &> 8n^2+4n-36n-18 \\
	\iff -4n-28n+14 &> 4n-36n-18 \\
	\iff 14 &> -18 \\
\end{align*}
Vamos a partir a $A$ en $A = A_1 \cap A_2$, con $A_1 = \{a_{i < 5}\}$, y $A_2 = \{a_{i\geq 5}\}$

Luego $\sup A = \max \{\sup A_1, \sup A_2\}$, donde $\sup A_2 = a_5$, y $\sup A_1$ se
puede sacar porque hay finitos

Además $\inf A_2 = 2$ por límite, y $\inf A_1$ es fácil

\subsection{Ejercicio 2}
$A = \left\{\frac{(-1)^n + 7}{n+4} \; \middle| \; n \in \N\right\}$

Tomamos las dos sumsicesiones $A_1 = \{a_{2k+1}\}$, $A_2 = \{a_{2k}\}$.

Una de estas es decreciente, y la otra es creciente, entonces los ínfimos y supremos se
calculan con el límite y promer elemento, y con esos se calculan los de $A$

\subsection{Ejercicio 3}
Sucesión definida por recurrencia
$b_1 = \sqrt{2}$, $b_{n+1} = \sqrt{2 + b_n}$.

Vamos a probar $b_i < 2$. Por inducción,
$b_1 < 2$ y

\[\sqrt{2 + b_n} < 2 \iff 2 + b_n < 4\iff b_n < 2\]

Además notemos $(b_n)^2 < 2 + b_n$, ya que $b_n < 2$

Sea $f(x) = \sqrt{2 + x} - x$. La derivada es $\frac{1}{2\sqrt{2 + x}} - 1$, y tenemos
\[\frac{1}{2\sqrt{2 + x}} - 1 < 0\]
\[\frac{1}{2\sqrt{2 + x}} < 1\]
\[1 < 2\sqrt{2 + x}\]
\[1 < 4(2 + x)\]
\[\frac{1}{4} < 2 + x\]
\[\frac{1}{4} -2 < x\]
Osea que es decreceiente $f$ para los positivos.

Además notemos que $f 2 = 2$, entonces $f x > 0 \;\forall x < 2$

Supongamos que $\lim_{n \to \infty} a_n = L$ con $L < 2$, luego tomemos $a_n$ tal
que $L - a_n < f(L)$, entonces notemos que también $L - a_n < f a_n$, ya que $f$ es decreciente,
pero luego $a_{n+1} = a_n + f a_n > L$. Contradicción.

Luego $\lim_{n \to \infty} a_n = 2$

\section{Topología}
\subsection{Distancia}

A todos los puntos a distancia menor a $\varepsilon$ con centro $C$ se los llama
una bola (abierta) de centro $C$ con radio $\varepsilon$. Se nota $B_\varepsilon(C)$

\subsubsection{En $\R$}
La distancia se mide con módulos.

\subsubsection{En $\R^n$}
La norma es $||(x_i, \dots)|| = \sqrt{\sum x_i^2}$, y la distancia es $d(P, Q) = ||P-Q||$

\subsection{Apertura}
Un subconjunto $S \in \R^n$ es abierto $\iff \;\forall s \in S \;\exists \varepsilon \mid
B_{\varepsilon}(s) \subseteq S$

$S$ es cerrado $\iff$ $\R^n - S$ es abierto.
\subsection{Propiedad}
Se tiene que $|P_x - Q_x| \leq d(P, Q)$, para todos puntos $P$, $Q$.
\end{document}
