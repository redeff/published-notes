\documentclass{article}
\usepackage{amssymb}
\usepackage{mathtools}

\everymath{\displaystyle}
\setlength{\parskip}{3mm}
\setlength{\parindent}{0mm}

\def\R{\mathbb{R}}
\def\K{\mathbb{K}}
\def\V{\mathbb{V}}
\def\W{\mathbb{W}}
\def\WW{\mathcal{W}}
\def\PP{\mathcal{P}}
\def\FF{\mathcal{F}}
\def\U{\mathbb{U}}
\def\C{\mathbb{C}}
\def\N{\mathbb{N}}
\def\Q{\mathbb{Q}}
\def\Z{\mathbb{Z}}

\def\BB{\mathcal{B}}
\def\AA{\mathcal{A}}
\def\EE{\mathcal{E}}
\def\CC{\mathcal{C}}
\def\OO{\mathcal{O}}
\def\DD{\mathcal{D}}

\def\e{\varepsilon}
\def\inn{\subseteq}

\def\S{\mathbb{S}}
\def\T{\mathbb{T}}
\def\l{\lambda}

\def\To{\Rightarrow}
\def\from{\leftarrow}
\def\From{\Leftarrow}

\DeclareMathOperator{\cl}{cl}
\DeclareMathOperator{\li}{li}
\DeclareMathOperator{\tl}{tl}
\DeclareMathOperator{\Id}{Id}
\DeclareMathOperator{\tr}{tr}
\DeclareMathOperator{\oo}{o}
\DeclareMathOperator{\spec}{spec}
\DeclareMathOperator{\mult}{mult}

\DeclareMathOperator{\iso}{iso}
\DeclareMathOperator{\mono}{mono}
\DeclareMathOperator{\epi}{epi}
\DeclareMathOperator{\adj}{adj}

\DeclareMathOperator{\Nu}{Nu}
\DeclareMathOperator{\Ima}{Im}
\DeclareMathOperator{\id}{id}
\DeclareMathOperator{\ze}{ze}

\DeclareMathOperator{\rg}{rg}

\DeclareMathOperator{\Hom}{Hom}
\DeclareMathOperator{\GL}{GL}
\DeclareMathOperator{\cont}{cont}
\DeclareMathOperator{\Hs}{H}

\DeclareMathOperator{\D}{D}
\DeclareMathOperator{\lcm}{lcm}

\DeclareMathOperator{\ev}{ev}
\DeclareMathOperator{\sg}{sg}

\date{}
\author{}

\begin{document}
\section*{Cauchy}
Dadas $f, g$ continuas y derivables en $[a, b]$, existe un $c \in (a, b)$ tal
que:
\[f'c \cdot (gb - ga) = g'c \cdot (fb - fa)\]

\section*{Ejercicio 1}
Calcular $\sup_{(x,y) \in [0,3]^2} fxy$, donde:

\[
	fxy =
	\begin{cases}
		\frac{(x^5 + 5x^3) - (y^5 + 5y^3)}{x^3 - y^3} & \text{Si $x \neq y$} \\
		\\
		0 & \text{cc}
	\end{cases}
\]

Si $x = y$, el supremo es $0$. Si no, tomemos $\phi t = t^5+5t^3$, y $\psi t =
t^3$. WLOG, $x < y$, y podemos hacer Cauchy en el intervalo $(x, y)$ con las
funciones $\phi, \psi$. Luego tenemos:

\[
	\frac{(x^5 + 5x^3) - (y^5 + 5y^3)}{x^3 - y^3} =
\]
\[
	\frac{\phi x - \phi y}{\psi x - \psi y} =
\]
\[
	\frac{\phi' c}{\psi' c} =
\]
\[
	\frac{5c^4+15c^2}{3c^2}
\]
Para algún $c \in (x, y) \inn (0, 3)$. Entonces ahora queremos el supremo:
\[
	\sup_{c \in (0,3)}\frac{5c^4+15c^2}{3c^2} =
\]
\[
	\sup_{c \in (0,3)}\frac{5c^2+15}{3} =
\]
\[
	\sup_{c \in (0,3)}\frac{5}{3} \cdot c^2 + 5 =
\]
\[
	5 + \frac{5}{3} \cdot \sup_{c \in (0,3)} c^2 =
\]
\[
	5 + \frac{5}{3} \cdot 9 = 5 + 15 = 20
\]
Como la función es contínua, entonces el supremo es 20.

\section*{Derivadas Multivariables}
Dada una $f : \R^2 \to \R$, queremos las condiciones necesarias para la
existencia de un plano tangente al gráfico de $f$ en un punto $p \in \R^2$.

Es decir, queremos un plano definido por $\pi q = p + (p-q) \cdot m$, con $m, q
\in \R^2$.

Si queremos calcular la derivada en la recta $\phi_v = \lambda \mapsto p +
\lambda v$, para alguna dirección $v$, es lo mismo que calcular
$\phi_v' \; 0$, con $\phi : \R \to \R$.

Entonces, dado un vector $v \in \R^n$ con $||v|| = 1$, se dice que:

\[
	\frac{\partial f}{\partial v} \; p = (\lambda \mapsto f\;(p + \lambda v))' \; 0
	= \lim_{\lambda \to 0} \frac{f \; (p + \lambda v) - f \; p}{\lambda} 
\]

\section*{Ejercicio 2}
Calcular todas las derivadas direccionales en $(0, 0)$ de:

\[
	f \; x \; y =
	\begin{cases}
		\frac{x^4 + x^2y}{2y^2} & y \neq 0 \\
		\\
		0 & y = 0
	\end{cases}
\]

Si $v_y = 0$, luego claramente $\frac{\partial f}{\partial v} \; 0 = 0$, luego
tomemos $v_y \neq 0$, entonces la función siempre toma la primer forma, y
tenemos:

$\frac{\partial f}{\partial v} \; (0, 0) = (\lambda \mapsto f \; ((0, 0) +
\lambda v))' = 
(\lambda \to f \; \lambda v)' \; 0 = $
\[\left(\lambda \mapsto
\frac{\lambda^4v_x^4 + \lambda^3v_x^2v_y}{2\lambda^2v_y^2} \right)' \; 0= \]
\[\left(\lambda \mapsto
\lambda \cdot \frac{\lambda v_x^4 + v_x^2v_y}{2v_y^2} \right)' \; 0= \]
\[\text{Como en la derivada del producto, la parte con $\lambda$ se va,}
	\left(\lambda \mapsto
\frac{\lambda v_x^4 + v_x^2v_y}{2v_y^2} \right) \; 0= \]
\[\left(\lambda \mapsto
\frac{v_x^2v_y}{2v_y^2} \right) \; 0= \frac{v_x^2}{2v_y}\]

\section*{Derivadas Parciales}
Es una derivada direccional donde $v$ es un canónico. Esta se puede calcular
fácilmente tomando, por ejemplo,
\[
	\frac{\partial f}{\partial x} \; (x, y) = (\lambda \mapsto f \; (\lambda, y))' \; x
\]
Es decir, derivamos en una variable como si la otra fuese constante.

Por ejemplo, si tenemos:
\[f \; (x, y, z) = e^{zy} \cdot \ln (x^2 + y^2)\]
Entonces las derivadas son
\[f_x \; (x, y, z) = e^{zy} \cdot \frac{2x}{x^2+y^2} \]
\[f_y \; (x, y, z) = z e^{zy} \ln (x^2 + y^2) +
\frac{e^{zy}}{x^2+y^2} \]
\[f_z \; (x, y, z) = ye^{zy} \cdot \ln (x^2 + y^2)\]

\section*{Diferenciabilidad}
Dada $f : \R^n \to \R$ es diferenciable en $p$ si existen las derivadas parciales
$f_{x_i} \; p$ y
\[\lim_{q \to p} \frac{f \; p - \sum \; (q_i - p_i) \cdot f_{x_i} \;
p}{||q - p||} \]
\end{document}
