\documentclass{article}
\usepackage{amssymb}
\usepackage{mathtools}

\everymath{\displaystyle}
\setlength{\parskip}{3mm}
\setlength{\parindent}{0mm}

\def\R{\mathbb{R}}
\def\K{\mathbb{K}}
\def\V{\mathbb{V}}
\def\W{\mathbb{W}}
\def\WW{\mathcal{W}}
\def\PP{\mathcal{P}}
\def\FF{\mathcal{F}}
\def\U{\mathbb{U}}
\def\C{\mathbb{C}}
\def\N{\mathbb{N}}
\def\Q{\mathbb{Q}}
\def\Z{\mathbb{Z}}

\def\BB{\mathcal{B}}
\def\AA{\mathcal{A}}
\def\EE{\mathcal{E}}
\def\CC{\mathcal{C}}
\def\OO{\mathcal{O}}
\def\DD{\mathcal{D}}

\def\e{\varepsilon}
\def\inn{\subseteq}

\def\S{\mathbb{S}}
\def\T{\mathbb{T}}
\def\l{\lambda}

\def\To{\Rightarrow}
\def\from{\leftarrow}
\def\From{\Leftarrow}

\DeclareMathOperator{\cl}{cl}
\DeclareMathOperator{\li}{li}
\DeclareMathOperator{\tl}{tl}
\DeclareMathOperator{\Id}{Id}
\DeclareMathOperator{\tr}{tr}
\DeclareMathOperator{\oo}{o}
\DeclareMathOperator{\spec}{spec}
\DeclareMathOperator{\mult}{mult}

\DeclareMathOperator{\iso}{iso}
\DeclareMathOperator{\mono}{mono}
\DeclareMathOperator{\epi}{epi}
\DeclareMathOperator{\adj}{adj}

\DeclareMathOperator{\Nu}{Nu}
\DeclareMathOperator{\Ima}{Im}
\DeclareMathOperator{\id}{id}
\DeclareMathOperator{\ze}{ze}

\DeclareMathOperator{\rg}{rg}

\DeclareMathOperator{\Hom}{Hom}
\DeclareMathOperator{\GL}{GL}
\DeclareMathOperator{\cont}{cont}
\DeclareMathOperator{\Hs}{H}

\DeclareMathOperator{\D}{D}
\DeclareMathOperator{\lcm}{lcm}

\DeclareMathOperator{\ev}{ev}
\DeclareMathOperator{\sg}{sg}

\date{}
\author{}

\title{Teórica de Análisis 2019-09-05}
\begin{document}
	\maketitle
	\section{Funciones Contínuas}
	\subsection{Por Sucesiones}
	Una función $f : A \to B$ es contíua iff para toda sucesión $x \in \N \to A :
	\lim_{\infty} x = p$,
	se fiene que $\lim_{\infty} f \circ x = fp$.

	\subsection{Por coordenada}
	Supongamos que tenemos
	\[f : \R^d \to \R^n \sim (d \to \R) \to (n \to \R) \sim
	n \to (d \to \R) \to \R\]

	Tomando la última definición,
	la continuidad de $f$, es equivalente a que sean contínuas $fi \forall i \in n$

	\subsection{Propiedades}
	Suma, difrencia, producto, y división de contínuas es contínua. (división
	si la de abajo no es cero en el punto)

	\subsection{Composición de Contínuas}
	Sean $f, g: \R^d \to \R^n$, con $f$ contínua en $p$ y $g$ contínua en $fp$,
	tenemos que $g \circ f$ es contínua en $p$.

	Supongamos $\lim_{i \to \infty} xi = p$, luego tenemos por continuidad
	$\lim f \circ x = fp$, luego, por continuidad, $\lim g \circ f \circ x = gfp$,
	luego $g \circ f$ es contínua.

	\subsection{Suma de Contínuas}
	Tomémonos $f : \R^d \to \R^n$ y $g : \R^d \to \R^n$ contínuas en $p$,
	luego tomémonos $t : \R^d \to \R^{2n}, tp = (fp, gp)$, y la función $s(x, y) = x+y$,
	luego tenemos que $f + g = s \circ t$, y por composición de contínuas, la suma es
	contínua en $p$.

	\section{Conjuntos Cerrados}
	Son aquellos conjuntos $A$ tales que $\partial A \inn A$. Equivalentemente,
	$A$ es cerrado si para toda sucesión $x \in \N \to A$ que converge a
	algún $L = \lim x$, se tiene $L \in A$.

	\section{Weierstrass}
	Dado $A \in \R^d$ un compacto (osea cerrado y acotado) y una función contínua
	$f : A \to \R$, entonces:
	\begin{itemize}
		\item $f A \inn \R$ es un conjunto acotado.
		\item Existe un $x \in A$ tal que $fx = \sup f A$ (osea $f$
		alcanza su máximo en $A$).
		\item Equivalntemente, existen $x_1, x_2 \in
		A : \forall x \in A : fx_1 \leq fx \leq fx_2$
	\end{itemize}

	Demo:

	Veamos que $f$ está acotada superiormente. Supongamos que no, luego tomemos
	$x : \N \to A$, donde $\forall n: fx_n > n$. Luego existe una subsucesión
	$y : \N \to A, y = x \circ \sigma$, $\sigma$ creciente.
	que converge $\lim y = p$, que por clausura de $A$ sabemos
	que $p \in A$.

	Además, por continuidad $\lim f \circ y = f \lim y = fp$,
	pero también $fp = \lim f \circ x \circ \sigma > \lim \sigma >
	\lim_{n \to \infty} n = +\infty$, absurdo.

	Ahora veamos que se alcanza el máximo. Sea $M = \sup f A$. Si el
	máximo no se alcanza,
	considerémonos la función $f' : A \to \R, f' a = \frac{1}{M - fa}$, que
	va a estar bien definida ya que $fa \neq M$.

	Claramente $f'$ es contínua, entonces tiene una cota $N$, entonces se
	tiene que $\frac{1}{M - fa} < N$ para algún $N$. pero entonces
	$M - fa > \frac{1}{N}$, para todo $a$, pero luego $M$ no era supremo ya
	que está acotada la distancia a $M$.
\end{document}
