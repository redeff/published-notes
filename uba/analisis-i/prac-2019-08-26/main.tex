\documentclass{article}
\usepackage{amssymb}
\usepackage{mathtools}

\everymath{\displaystyle}
\setlength{\parskip}{3mm}
\setlength{\parindent}{0mm}

\def\R{\mathbb{R}}
\def\K{\mathbb{K}}
\def\V{\mathbb{V}}
\def\W{\mathbb{W}}
\def\WW{\mathcal{W}}
\def\PP{\mathcal{P}}
\def\FF{\mathcal{F}}
\def\U{\mathbb{U}}
\def\C{\mathbb{C}}
\def\N{\mathbb{N}}
\def\Q{\mathbb{Q}}
\def\Z{\mathbb{Z}}

\def\BB{\mathcal{B}}
\def\AA{\mathcal{A}}
\def\EE{\mathcal{E}}
\def\CC{\mathcal{C}}
\def\OO{\mathcal{O}}
\def\DD{\mathcal{D}}

\def\e{\varepsilon}
\def\inn{\subseteq}

\def\S{\mathbb{S}}
\def\T{\mathbb{T}}
\def\l{\lambda}

\def\To{\Rightarrow}
\def\from{\leftarrow}
\def\From{\Leftarrow}

\DeclareMathOperator{\cl}{cl}
\DeclareMathOperator{\li}{li}
\DeclareMathOperator{\tl}{tl}
\DeclareMathOperator{\Id}{Id}
\DeclareMathOperator{\tr}{tr}
\DeclareMathOperator{\oo}{o}
\DeclareMathOperator{\spec}{spec}
\DeclareMathOperator{\mult}{mult}

\DeclareMathOperator{\iso}{iso}
\DeclareMathOperator{\mono}{mono}
\DeclareMathOperator{\epi}{epi}
\DeclareMathOperator{\adj}{adj}

\DeclareMathOperator{\Nu}{Nu}
\DeclareMathOperator{\Ima}{Im}
\DeclareMathOperator{\id}{id}
\DeclareMathOperator{\ze}{ze}

\DeclareMathOperator{\rg}{rg}

\DeclareMathOperator{\Hom}{Hom}
\DeclareMathOperator{\GL}{GL}
\DeclareMathOperator{\cont}{cont}
\DeclareMathOperator{\Hs}{H}

\DeclareMathOperator{\D}{D}
\DeclareMathOperator{\lcm}{lcm}

\DeclareMathOperator{\ev}{ev}
\DeclareMathOperator{\sg}{sg}

\date{}
\author{}

\begin{document}
\section{Ejercicio 1}
Dado el conjunto $A = \{(x, y) \in \R^2 : -x < y, y + x^2 - 2 \leq 0\}$
No es abierto ya que el punto $(0, 2)$ está en $A$ pero $(0, 2 + \e) \notin A$
para todo $\e > 0$.

\includegraphics{para.eps}

Notemos que tampoco es cerrado, ya que $(0, 0) \notin A$, pero $(0, \e) \in A$ para
cualquier $\e$ suficientemente chico.

\section{Conjuntos Acotados}
Un conjunto $A \inn \R^n$ es acotado si existe alguna bola $A \inn B_M(0)$

\section{Propiedades}
\begin{itemize}
	\item $A^\circ \inn A$
	\item $A^\circ$ es abierto
	\item si $A$ abierto luego $A^\circ = A$
\end{itemize}

\section{Punto de Acumulación (Clausura)}
Un punto $x \in \R^n$ es un punto de acumulación
si existe una sucesión $a : \N \to A$ tal que $\lim_{i \to \infty} ai = x$.

Se define al conjunto de puntos de acumulación como $\cl A = \overline A$.
Es claro que $A \inn \overline A$, y además $\overline A = \cl A$ es cerrado.

\section{Borde}
el borde de $A$ se define como $\partial A = \cl A - A^\circ$.

Alternativamente, $\partial A = \{ x : \forall \e > 0 : B_r(x) \cap A \neq \emptyset,
B_r(x) \cap A^C \neq \emptyset
\}$

\section{Ejercicio 2}
para los siguientes conjuntos, decidir si son acotados, y hallar el interior,
la clausura y el borde.
\begin{itemize}
	\item $A = [-1, 2)$
	\item $B = \{(x, y) \in \R^2 : 1 < (x-1)^2 + y^2 < 4\}$
	\item $C = \{(x, y) : |x| \leq 3\}$
\end{itemize}

\section{Cónicas}
Una cónica es una es una curva en $\R^2$ que surge de cortar un cono con un plano.
Se dividen en:
\begin{itemize}
	\item Circunferencia: $x^2 + y^2 = r^2$.
	\item Elipse: $\frac{x^2}{a^2} + \frac{y^2}{b^2} = 1$
	\item Parábola: $y = x^2$
	\item Hipérbola: $\frac{x^2}{a^2} - \frac{x^2}{b^2} = 1$
\end{itemize}

\end{document}
