\documentclass{article}
\usepackage{amssymb}
\usepackage{mathtools}

\everymath{\displaystyle}
\setlength{\parskip}{3mm}
\setlength{\parindent}{0mm}

\def\R{\mathbb{R}}
\def\K{\mathbb{K}}
\def\V{\mathbb{V}}
\def\W{\mathbb{W}}
\def\WW{\mathcal{W}}
\def\PP{\mathcal{P}}
\def\FF{\mathcal{F}}
\def\U{\mathbb{U}}
\def\C{\mathbb{C}}
\def\N{\mathbb{N}}
\def\Q{\mathbb{Q}}
\def\Z{\mathbb{Z}}

\def\BB{\mathcal{B}}
\def\AA{\mathcal{A}}
\def\EE{\mathcal{E}}
\def\CC{\mathcal{C}}
\def\OO{\mathcal{O}}
\def\DD{\mathcal{D}}

\def\e{\varepsilon}
\def\inn{\subseteq}

\def\S{\mathbb{S}}
\def\T{\mathbb{T}}
\def\l{\lambda}

\def\To{\Rightarrow}
\def\from{\leftarrow}
\def\From{\Leftarrow}

\DeclareMathOperator{\cl}{cl}
\DeclareMathOperator{\li}{li}
\DeclareMathOperator{\tl}{tl}
\DeclareMathOperator{\Id}{Id}
\DeclareMathOperator{\tr}{tr}
\DeclareMathOperator{\oo}{o}
\DeclareMathOperator{\spec}{spec}
\DeclareMathOperator{\mult}{mult}

\DeclareMathOperator{\iso}{iso}
\DeclareMathOperator{\mono}{mono}
\DeclareMathOperator{\epi}{epi}
\DeclareMathOperator{\adj}{adj}

\DeclareMathOperator{\Nu}{Nu}
\DeclareMathOperator{\Ima}{Im}
\DeclareMathOperator{\id}{id}
\DeclareMathOperator{\ze}{ze}

\DeclareMathOperator{\rg}{rg}

\DeclareMathOperator{\Hom}{Hom}
\DeclareMathOperator{\GL}{GL}
\DeclareMathOperator{\cont}{cont}
\DeclareMathOperator{\Hs}{H}

\DeclareMathOperator{\D}{D}
\DeclareMathOperator{\lcm}{lcm}

\DeclareMathOperator{\ev}{ev}
\DeclareMathOperator{\sg}{sg}

\date{}
\author{}

\begin{document}
\section*{Integración en Varias variables}
Dado un rectángulo $D = [a,b] \times [c,d]$, y una función $f : D \to \R$, vamos a definir:
\[I = \iint\limits_D f(x,y) \; dx \; dy\]
Geométricamente, $I$ representa el volumen de los puntos en $\R^3$ debajo del gráfico de la función.

Formalmente, se define parecido a la integral de riemann.
Dada dos funciones crecientes $\pi : \{n\} \to [a,b]$, $\tau : \{m\} \to [c,d]$ particiones, decimos que:
\[s(f,\pi,\tau) = \sum_i \sum_j |\pi_{i+1} - \pi_i| \cdot |\tau_{i+1} - \tau_i| \cdot k_{ij}\]
Con
\[k_{ij} = \inf \{f(x,y) : \pi_i \leq x \leq \pi_{i+1}, \tau_j \leq y \leq \tau_{j+1}\}\]
Y además,
\[S(f,\pi,\tau) = \sum_i \sum_j |\pi_{i+1} - \pi_i| \cdot |\tau_{i+1} - \tau_i| \cdot K_{ij}\]
Con
\[K_{ij} = \sup \{f(x,y) : \pi_i \leq x \leq \pi_{i+1}, \tau_j \leq y \leq \tau_{j+1}\}\]

Luego tomamos
\[\iint\limits_D f(x,y) \; dx \; dy = \sup_{\pi, \tau} s(f, \pi, \tau) = \inf_{\pi, \tau} S(f, \pi, \tau)\]
Cuando estos valores coinciden.

\subsection*{Integrabilidad de Contínuas}
Toda función contínua es integrable.

\section*{Teorema de Fubini}
Para toda función contínua $f : [a,b] \times [c, d] \to \R$, se cumple que:
\[
	\iint\limits_{[a,b] \times [c,d]} f(x,y) \; dx \; dy = \int_c^d \left[\int_a^b f(x,y) \; dx\right] \; dy
\]
Y
\[
	\iint\limits_{[a,b] \times [c,d]} f(x,y) \; dx \; dy = \int_a^b \left[\int_c^d f(x,y) \; dy\right] \; dx
\]

\section*{Integral de Variables Separadas}
Por fubini, trivialmente:
\[
	\iint\limits_D f(x) \cdot g(y) \; dx \; dy = \left(\int f(x) \; dx\right) \left(\int g(y) \; dy\right)
\]

\section*{Integral sobre Regiones no Rectangulares}
Supongamos que tenemos una función $f : C \to \R$, con $C \inn \R^2$ acotada, luego va a existir un rectángulo $D : C \inn D$. Luego extendemos la función de tal forma que:

\[
	f'(x) =
	\begin{cases}
		f(x) & \text{Si $x \in C$} \\
		0 & \text{Si $x \notin C$} \\
	\end{cases}
\]

Y definimos:
\[
	\iint\limits_C f(x,y) \; dxdy = \iint\limits_D f'(x,y) \; dxdy
\]

\section*{Dominios Simples}
Un dominio simple es un subconjunto de $\R^2$ de la forma:
\[D = \{(x,y) : a \leq x \leq b, \phi_1(x) \leq y \leq \phi_2(x)\}\]

Por fubini, tenemos que que:
\[
	\iint\limits_D f(x,y) \; dxdy = \int_a^b \left[\int_{\phi_1(x)}^{\phi_2(x)} f(x,y) \; dy\right] dx
\]
\section*{Ejemplo}
Ponele que queremos calcular el volumen de un cono de altura $h$, y base de radio $r$, luego tenemos que su base es:
\[D = \{(x,y) : x^2 + y^2 \leq r^2\}\]
Y la integral que queremos es:
\[\iint\limits_D h \left(1 - \frac{\sqrt{x^2 + y^2}}{r}\right) =\]
\[h \pi r^2 - h\iint\limits_D \frac{\sqrt{x^2 + y^2}}{r} =\]
\[h \pi r^2 - \frac{h}{r}\iint\limits_D \sqrt{x^2 + y^2}\]
\[\iint\limits_D \sqrt{x^2 + y^2} =\]
\[\int_{-r}^r \int_{-\sqrt{r^2-x^2}}^{\sqrt{r^2-x^2}} \sqrt{x^2 + y^2} \; dydx\]

\section*{Regiones de Área $0$}
Un subconjunto $A \in \R^2$ se dice de \emph{Contenido, o área $0$}, si para todo $\e > 0$, existe un conjunto \emph{finito} $\DD$ de rectángulos, tal que $A \inn \bigcup \DD$, y $\sum_{d \in \D} \text{area}(d) \leq \e$.

Alternativamente, decimos que tiene área 0 sii:
\[0 = \inf_{\pi, \tau} \sum_{(i,j) \in K} |\pi_{i+1} - \pi_i| \cdot |\tau_{i+1} - \tau_i|\]
Dónde:
\[K = \{(i,j) : [\pi_i, \pi_{i+1}] \times [\tau_i, \tau_{i+1}] \cap A \neq \emptyset\}\]

\section*{Funciones que difieren en Conjuntos de Área 0}
Si tenemos un conjunto $A$ de área $0$, luego si:
\[\forall (x,y) \notin A : f(x,y) = g(x,y)\]
Entonces:
\[\iint\limits_D f = \iint\limits_D g\]
Para todo $D$.
\end{document}
