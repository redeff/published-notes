\documentclass{article}
\usepackage{amssymb}
\usepackage{mathtools}

\everymath{\displaystyle}
\setlength{\parskip}{3mm}
\setlength{\parindent}{0mm}

\def\R{\mathbb{R}}
\def\K{\mathbb{K}}
\def\V{\mathbb{V}}
\def\W{\mathbb{W}}
\def\WW{\mathcal{W}}
\def\PP{\mathcal{P}}
\def\FF{\mathcal{F}}
\def\U{\mathbb{U}}
\def\C{\mathbb{C}}
\def\N{\mathbb{N}}
\def\Q{\mathbb{Q}}
\def\Z{\mathbb{Z}}

\def\BB{\mathcal{B}}
\def\AA{\mathcal{A}}
\def\EE{\mathcal{E}}
\def\CC{\mathcal{C}}
\def\OO{\mathcal{O}}
\def\DD{\mathcal{D}}

\def\e{\varepsilon}
\def\inn{\subseteq}

\def\S{\mathbb{S}}
\def\T{\mathbb{T}}
\def\l{\lambda}

\def\To{\Rightarrow}
\def\from{\leftarrow}
\def\From{\Leftarrow}

\DeclareMathOperator{\cl}{cl}
\DeclareMathOperator{\li}{li}
\DeclareMathOperator{\tl}{tl}
\DeclareMathOperator{\Id}{Id}
\DeclareMathOperator{\tr}{tr}
\DeclareMathOperator{\oo}{o}
\DeclareMathOperator{\spec}{spec}
\DeclareMathOperator{\mult}{mult}

\DeclareMathOperator{\iso}{iso}
\DeclareMathOperator{\mono}{mono}
\DeclareMathOperator{\epi}{epi}
\DeclareMathOperator{\adj}{adj}

\DeclareMathOperator{\Nu}{Nu}
\DeclareMathOperator{\Ima}{Im}
\DeclareMathOperator{\id}{id}
\DeclareMathOperator{\ze}{ze}

\DeclareMathOperator{\rg}{rg}

\DeclareMathOperator{\Hom}{Hom}
\DeclareMathOperator{\GL}{GL}
\DeclareMathOperator{\cont}{cont}
\DeclareMathOperator{\Hs}{H}

\DeclareMathOperator{\D}{D}
\DeclareMathOperator{\lcm}{lcm}

\DeclareMathOperator{\ev}{ev}
\DeclareMathOperator{\sg}{sg}

\date{}
\author{}

\begin{document}
\section*{$\CC^1$}
	Una función $f : \R^n \to \R$ se dice que $f : \in \CC^1$ si existen todas las derivadas parciales y son contínuas.

	Tenemos las siguientes implicaciones:
	\begin{itemize}
		\item $\CC^1 \To $ diferenciable
		\item diferenciable $\To$ contínua
		\item diferenciable $\To$ existen las derivadas direccionales
		\item existen las derivadas direccionales $\To$ existen las parciales
	\end{itemize}

\section*{Ejemplo}
\[f(x, y) = 
\begin{cases}
	\frac{x^4+x^2y}{x^2 + y^2} & \text{Si $(x,y) \neq (0,0)$}\\
	0 & \text{Si $(x,y) = (0,0)$}
\end{cases}\]

Las derivadas parciales existen y dan $0$ (son fáciles)

Para continuidad tenemos
\[\frac{x^4+x^2y}{x^2+y^2} \leq\]
\[\frac{||-||^4+||-||^3}{||-||} \leq\]
\[2||-||\]

Para diferenciabilidad, queremos ver que:
\[\lim\frac{x^4+x^2y}{(x^2+y^2)||x,y||} = 0\]
\[\lim\frac{x^4+x^2y}{||x,y||^3} =\]

Si tomamos $x = y$, entonces
\[\lim\frac{x^4+x^3}{||x,x||^3} =\]
\[\lim\frac{x^4+x^3}{2^{3/2}x^3} =\]
\[\lim 2^\frac{2}{3}(1+x) \neq 0\]
Entonces no es diferenciable.

\section*{Ejercicio 2}
\[
	f = 
	\begin{cases}
		\frac{x^2(7y+3)+y^2(7x+3)}{7(x^2+y^2)} & (x,y) \neq (0,0)\\
		\frac{3}{7} & (x,y) = (0,0)
	\end{cases}
\]
Ver que es contínua, que existen las direccionales y que no es difernciable
\[\frac{x^2(7y+3)+y^2(7x+3)}{7(x^2+y^2)} - \frac{3}{7}=\]
\[\frac{x^2(7y+3)+y^2(7x+3) - 3x^2 - 3y^2}{7(x^2+y^2)}=\]
\[\frac{7x^2y+7y^2x}{7(x^2+y^2)}=\]
\[\frac{x^2y+y^2x}{x^2+y^2}\leq\]
\[\frac{2||x,y||^3}{||x,y||^2}=\]
\[2||x,y||\]

Ahora la continuidad
\[\frac{f(ta, tb) - f0}{t}=\]
\[\frac{(ta)^2(tb)+(tb)^2(ta)}{t((ta)^2+(tb)^2)}=\]
\[\frac{t^3(a^2b+b^2a)}{t^3(a^2+b^2)}=\]
\[\frac{a^2b+b^2a}{a^2+b^2}=\]
\[a^2b+b^2a=\]
\[ab(a+b)=\]

Pero entonces $\frac{\partial f}{\partial x} = \frac{\partial f}{\partial y} = 0$, pero entonces todas las direccionales son $0$, pero si tomamos $a = b= \frac{1}{\sqrt{2}}$, la cuanta da no nula, entonces $f$ no es diferenciable.

\section*{Ejercicio 3}
\[
	f(x,y) =
	\begin{cases}
		xy\sin\left(\frac{1}{x^2+y^2} \right) & (x,y) \neq 0 \\
		0
	\end{cases}
\]

Las parciales son trivialmente $0$. Queremos ver si es diferenciable.
\[\lim\frac{xy\sin\left(\frac{1}{x^2+y^2} \right)}{||x,y||}=\]
\[\lim\frac{xy}{||x,y||}=\]
\[\lim\frac{||x,y||^2}{||x,y||}=\]
\[\lim||x,y|| = 0\]
Entonces es diferenciable.

Ahora calculemos $\frac{\partial f}{\partial x} $ como función
\[\frac{\partial f}{\partial x}(x,y) = y\sin \left(\frac{1}{x^2+y^2}\right) + xy\cos \left(\frac{1}{x^2+y^2}\right)\frac{-1}{(x^2+y^2)^2}2x\]

Si tomamos $x = y$, queda
\[\frac{\partial f}{\partial x}(x,y) = x\sin \left(\frac{1}{2x^2}\right) + x^2\cos \left(\frac{1}{2x^2}\right)\frac{-1}{4x^4}2x\]
\[\frac{\partial f}{\partial x}(x,y) = x\sin \left(\frac{1}{2x^2}\right) - \frac{1}{2x}\cos \left(\frac{1}{2x^2}\right)\]
Tomemos una sucesión $a_n$ tal que $\cos\frac{1}{2a_n^2} = 1$ y $a_n \to 0$, entonces tendremos que nuestra derivada parcial tiene una subsucesión que tiende a $\infty$. Luego tomamos $\frac{1}{2a_n^2} = 2n\pi$, entonces si tomamos $a_n = \frac{1}{2\sqrt{n\pi}}$ cumple lo que queremos, y está.

\section*{Ejercicio 4}
Tenemos $f$ deferenciable en $(2,3)$, y
\[v_1 = \left(\frac{1}{\sqrt{2}} , \frac{1}{\sqrt{2}} \right)\]
\[v_2 = \left(\frac{3}{5} , \frac{4}{5} \right)\]
Nos dicen $\frac{\partial f}{\partial v_1}(2,3) = 2$, y $\frac{\partial f}{\partial v_2} (2,3) = 1$

si las direccionales son $a, b$, tenemos el sistema de ecuaciones:
\[
\begin{cases}
	\frac{a}{\sqrt{2}} + \frac{b}{\sqrt{2}} = 2 \\
	\\
	\frac{3a}{5} + \frac{4b}{5} = 1 \\
\end{cases}\]
Si lo resolvemos tenemos que $(a,b) = (8\sqrt{2} - 5, 5-6\sqrt{2})$
\end{document}
