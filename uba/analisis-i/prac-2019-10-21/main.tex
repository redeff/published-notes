\documentclass{article}
\usepackage{amssymb}
\usepackage{mathtools}

\everymath{\displaystyle}
\setlength{\parskip}{3mm}
\setlength{\parindent}{0mm}

\def\R{\mathbb{R}}
\def\K{\mathbb{K}}
\def\V{\mathbb{V}}
\def\W{\mathbb{W}}
\def\WW{\mathcal{W}}
\def\PP{\mathcal{P}}
\def\FF{\mathcal{F}}
\def\U{\mathbb{U}}
\def\C{\mathbb{C}}
\def\N{\mathbb{N}}
\def\Q{\mathbb{Q}}
\def\Z{\mathbb{Z}}

\def\BB{\mathcal{B}}
\def\AA{\mathcal{A}}
\def\EE{\mathcal{E}}
\def\CC{\mathcal{C}}
\def\OO{\mathcal{O}}
\def\DD{\mathcal{D}}

\def\e{\varepsilon}
\def\inn{\subseteq}

\def\S{\mathbb{S}}
\def\T{\mathbb{T}}
\def\l{\lambda}

\def\To{\Rightarrow}
\def\from{\leftarrow}
\def\From{\Leftarrow}

\DeclareMathOperator{\cl}{cl}
\DeclareMathOperator{\li}{li}
\DeclareMathOperator{\tl}{tl}
\DeclareMathOperator{\Id}{Id}
\DeclareMathOperator{\tr}{tr}
\DeclareMathOperator{\oo}{o}
\DeclareMathOperator{\spec}{spec}
\DeclareMathOperator{\mult}{mult}

\DeclareMathOperator{\iso}{iso}
\DeclareMathOperator{\mono}{mono}
\DeclareMathOperator{\epi}{epi}
\DeclareMathOperator{\adj}{adj}

\DeclareMathOperator{\Nu}{Nu}
\DeclareMathOperator{\Ima}{Im}
\DeclareMathOperator{\id}{id}
\DeclareMathOperator{\ze}{ze}

\DeclareMathOperator{\rg}{rg}

\DeclareMathOperator{\Hom}{Hom}
\DeclareMathOperator{\GL}{GL}
\DeclareMathOperator{\cont}{cont}
\DeclareMathOperator{\Hs}{H}

\DeclareMathOperator{\D}{D}
\DeclareMathOperator{\lcm}{lcm}

\DeclareMathOperator{\ev}{ev}
\DeclareMathOperator{\sg}{sg}

\date{}
\author{}

\begin{document}
\section*{Parametrización de Curvas de Nivel}
Si tenemos una curva de nivel de la forma $F(x,y) = r$, como $x^2 + y^2 = 1$, luego en todo punto que $\frac{\partial f}{\partial y}p \neq 0$, podemos ``despejar'' la variable $y$ en función de la demás.

En el círculo dado por $x^2 + y^2 = 1$, por ejemplo, $x$ se puede despejar en todos los puntos menos $(0, 1), (0, -1)$, y los opesto con $y$.

\section*{Ejemplo 1}
Si $f(x,y) = y-x^3$, luego siempre podemos despejar $y$.
También podemos despejar $x$ en todo punto, y tendríamos $x = y^\frac{1}{3} $, pero esta función no es derivable en $0$.

\section*{Ejemplo 2}
Si tenemos $f(x,y,z) = x^2 + y^2 - z$, luego su superficie de nivel $0$ es un paraboloide.

Como $\frac{\partial f}{\partial z} = -1 \neq 0$, siempre podemos despejar $z$.

Tenemos $\nabla f(x,y,z) = (2x, 2y, -1)$, luego $x$ se puede despejar siempre que $x \neq 0$.

\section*{Teorema de la función implícita}
Si tenemos $S = \{(x,y,z) : f(x,y,z) = 0\}$, con $f : \R^3 \to \R$, $f \in \CC^1$.
Luego si tenemos $(x_0, y_0, z_0) \in S$, con $\frac{\partial f}{\partial z} (x_0, y_0, z_0)$, luego existe un entorno $U$ de $(x_0, y_0)$ y un $\phi : U \to_{\CC^1} \R$ tal que $\forall (x,y) \in U: f(x, y, \phi \; x y) = 0$.

También sabemos que
\[\frac{\partial \phi}{\partial x}(x,y) = -\frac{\frac{\partial f}{\partial x}(x,y,  \phi \; xy)}{\frac{\partial f}{\partial z}(x, y, \phi \;xy)}\]

\[\frac{\partial \phi}{\partial y}(x,y) = -\frac{\frac{\partial f}{\partial y}(x,y,  \phi \; xy)}{\frac{\partial f}{\partial z}(x, y, \phi \;xy)}\]

\section*{Ejercicio 1}
Tomemos la curva $S = \{(x,y) : f(x,y) = 0\}$ con
\[f(x,y) = x^3-y^3 + 2xy - x + y\]
Probar que se define explícitamente como función $y = \phi x$ al rededor de $p = (0, 1)$. Calcular el polinomio de taylor de orden 2 de $\phi$ al rededor de $x = 0$.

Notemos que $f \in \CC^1$, y que $p \in S$ ya que $fp = 0$.
Además \[\nabla f_y(x,y) = -3y^2 + 2x + 1\]
que no se anula en $p$.
Luego existe el $\phi$ que queremos, con derivada:
\[
	\phi' x = \frac{f_x(x,\phi x)}{f_y(x,\phi x)}
\]
Sabemos que $\phi 0 = 1$, y $\phi 0 = -\frac{f_x(0,1)}{f_y(0,1)} = -\frac{1}{-2} = \frac{1}{2}$.

Además:
\[
	(x \mapsto f_x(x, \phi x))' = \nabla f_x \cdot (1, \phi' x) = f_{xx} + f_{xy} \cdot -\frac{f_x(x, \phi x)}{f_y(x, \phi x)}
\]
\[
(x \mapsto f_x(x, \phi x))'(0, 1) = f_{xx} + \frac{1}{2} \cdot f_{xy}
\]
Similarmente, tenemos
\[
(x \mapsto f_y(x, \phi x))'(0, 1) = f_{yx} + \frac{1}{2} \cdot f_{yy}
\]
Entonces por regla de la cadena tenemos que $\phi'' x$ se puede calcular por regla del cociente en $(0,1)$

Alternativamente, tenemos:
\[
	\phi' x = \frac{f_x(x,\phi x)}{f_y(x,\phi x)}
\]
\[
	\phi' x = \frac{3x^2 + 2\phi x - 1}{-3(\phi x)^2 + 2x + 1}
\]
Luego por derivada de cociente:
\[
	\phi'' x = \frac{
		(3x^2 + 2\phi x - 1)
	(-3(\phi x)^2 + 2x + 1)' -
	(3x^2 + 2\phi x - 1)'
	(-3(\phi x)^2 + 2x + 1)
}{
(-3(\phi x)^2 + 2x + 1)^2
}
\]
\[
	\phi'' x = \frac{
		(3x^2 + 2\phi x - 1)
		(-6(\phi x)(\phi'x) + 2) -
	(6x + 2\phi' x)
	(-3(\phi x)^2 + 2x + 1)
}{
(-3(\phi x)^2 + 2x + 1)^2
}
\]
que evaluado en $0$ da
\[\phi'' 0 = \frac{1}{4}\]

Alternativamente, tenemos la ecuación $f(x, \phi x) = 0$, luego:
\[
	f(x, \phi x) = 0
\]
\[
	\nabla f(x, \phi x) \cdot (1, \phi x) = 0
\]
\[
	\Hs f(x, \phi x) \cdot (1, \phi x) +
	\nabla f(x, \phi x) \cdot (1, \phi x) +
	= 0
\]
\[
	x^3 - (\phi x)^3 + 2x\phi x - x - \phi x = 0
\]

\section*{Ejercicio 3}
Verificar que  la ecuación $x^3z - z^3yx = 0$ define implícitamente una función $z = \phi(x,y)$ alrededor de $p = (1,1,1)$. Calcular el plano tangente al gráfico de $\phi$ en $(1,1,1)$.

Si tomamos $f(x,y,z) = x^3z - z^3yx$, es claro que $f \in \CC^1$, y es claro que $p$ pertenece a la curva.

Además $f_z p$ no se anula, luego existe el $\phi$. Tenemos además:
\[\phi_x (x,y) = \frac{f_x(x,y, \phi xy)}{f_z(x,y, \phi xy)} \]
\[\phi_y (x,y) = \frac{f_y(x,y, \phi xy)}{f_z(x,y, \phi xy)} \]
Y todo eso lo podemos calcular, entonces tenemos las derivadas parciales, y como es diferenciable tenemos el plano tangente.
\end{document}
