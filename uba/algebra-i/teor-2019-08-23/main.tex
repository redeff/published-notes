\documentclass{article}
\usepackage{amssymb}
\usepackage{mathtools}

\everymath{\displaystyle}
\setlength{\parskip}{3mm}
\setlength{\parindent}{0mm}

\def\R{\mathbb{R}}
\def\K{\mathbb{K}}
\def\V{\mathbb{V}}
\def\W{\mathbb{W}}
\def\WW{\mathcal{W}}
\def\PP{\mathcal{P}}
\def\FF{\mathcal{F}}
\def\U{\mathbb{U}}
\def\C{\mathbb{C}}
\def\N{\mathbb{N}}
\def\Q{\mathbb{Q}}
\def\Z{\mathbb{Z}}

\def\BB{\mathcal{B}}
\def\AA{\mathcal{A}}
\def\EE{\mathcal{E}}
\def\CC{\mathcal{C}}
\def\OO{\mathcal{O}}
\def\DD{\mathcal{D}}

\def\e{\varepsilon}
\def\inn{\subseteq}

\def\S{\mathbb{S}}
\def\T{\mathbb{T}}
\def\l{\lambda}

\def\To{\Rightarrow}
\def\from{\leftarrow}
\def\From{\Leftarrow}

\DeclareMathOperator{\cl}{cl}
\DeclareMathOperator{\li}{li}
\DeclareMathOperator{\tl}{tl}
\DeclareMathOperator{\Id}{Id}
\DeclareMathOperator{\tr}{tr}
\DeclareMathOperator{\oo}{o}
\DeclareMathOperator{\spec}{spec}
\DeclareMathOperator{\mult}{mult}

\DeclareMathOperator{\iso}{iso}
\DeclareMathOperator{\mono}{mono}
\DeclareMathOperator{\epi}{epi}
\DeclareMathOperator{\adj}{adj}

\DeclareMathOperator{\Nu}{Nu}
\DeclareMathOperator{\Ima}{Im}
\DeclareMathOperator{\id}{id}
\DeclareMathOperator{\ze}{ze}

\DeclareMathOperator{\rg}{rg}

\DeclareMathOperator{\Hom}{Hom}
\DeclareMathOperator{\GL}{GL}
\DeclareMathOperator{\cont}{cont}
\DeclareMathOperator{\Hs}{H}

\DeclareMathOperator{\D}{D}
\DeclareMathOperator{\lcm}{lcm}

\DeclareMathOperator{\ev}{ev}
\DeclareMathOperator{\sg}{sg}

\date{}
\author{}

\begin{document}
\section{Familias Indexadas}
Dado un conjunto $I$, una familia $\FF$ indexada por $I$ con elementos en $A$ es una
función $\FF : I \to A$, y se define $\FF_i = \FF i = \FF(i)$. Y se nota
$(\FF_i)_{i \in I}$ en lugar de $\FF : I \to A$

Si $A$ es un conjunto de conjuntos, se dice que:
\[\bigcup_{i \in I} \FF_i = \{x : \exists i \in I : x \in \FF_i\}\]
\[\bigcap_{i \in I} \FF_i = \{x : \forall i \in I : x \in \FF_i\}\]

\section{Conjuntos Inductivos}
Un subconjunto $A \inn \R$ es inductivo sii $1 \in A$ y $a \in A \Rightarrow a+1 \in A$.

\section{Propiedades}
Sea $\FF : I \to \PP(\R)$. Si $\FF_i$ es inductivo para todo $i \in I$, tenemos
$X = \bigcap \FF_i$ es inductivo.

Claramente $1 \in X$ ya que $1 \in \FF_i \forall i \in I$. Además, si $a \in X$,
luego $a\in \FF_i \forall i$, por lo que $a+1 \in \FF_i \forall i$, luego $a+1 \in X$.

\section{Naturales}
Definimos a $\N$ como a la intersección de todos los subconjuntos inductivos de $\R$,
que por la propiedad que demostramos es inductivo.

\section{Inducción}
Si $A \in \N$ y $A$ es inductivo, luego $A = \N$.

\section{Suma de Gauss}
Dada una sucesión $f : \{n\} \to \R$, $f n = \lambda \cdot n + c$, la suma
$\sum_i fi = \sum_i \lambda \cdot i + c = c \cdot n + \lambda \sum_i i = c
\cdot n + \lambda \cdot \frac{n \cdot (n+1)}{2}$

Definamos $A = \left\{n \in \N : 1 + 2 + \dots + n = \frac{n \cdot (n+1)}{2} \right\}$.
Notemos que $1 \in A$, y además si $n \in A$ luego $n + 1 \in A$, ya que
$\frac{n \cdot (n+1)}{2} + (n+1) = \frac{(n+1) \cdot (n+2)}{2} $, entonces sabemos que
$A$ es inductivo, y $A \inn \N$, luego $A = \N$

\section{Suma Geométrica}
Por inducción sale que $\sum_{i \leq n} r^i = \frac{r^{n+1}-1}{r-1}$, ya que
\[\frac{r^{n+1}-1}{r-1} + r^{n+1}\]
\[\frac{r^{n+1}-1}{r-1} + \frac{r^{n+1} \cdot (r-1)}{r-1} \]
\[\frac{r^{n+1}-1 + r^{n+1} \cdot (r-1)}{r-1} \]
\[\frac{r^{n+1}-1 + r^{n+1} \cdot r - r^{n+1}}{r-1} \]
\[\frac{r^{n+2}-1}{r-1}\]
\end{document}
